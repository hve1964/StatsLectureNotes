%%%%%%%%%%%%%%%%%%%%%%%%%%%%%%%%%%%%%%%%%%%%%%%%%%%%%%%%%%%%%%%%%%%
%  File name: app1-Stats.tex
%  Title:
%  Version: 17.08.2019
%%%%%%%%%%%%%%%%%%%%%%%%%%%%%%%%%%%%%%%%%%%%%%%%%%%%%%%%%%%%%%%%%%%
%%%%%%%%%%%%%%%%%%%%%%%%%%%%%%%%%%%%%%%%%%%%%%%%%%%%%%%%%%%%%%%%%%%
\appendix
\chapter[Simple principal component analysis]{Principal component 
analysis of a $\boldsymbol{(2 \times 2)}$ correlation matrix}
\lb{app1}
%%%%%%%%%%%%%%%%%%%%%%%%%%%%%%%%%%%%%%%%%%%%%%%%%%%%%%%%%%%%%%%%%%%
Consider a real-valued
$\boldsymbol{(2 \times 2)}$ {\bf correlation matrix} expressed by
%
\be
\boldsymbol{R} =
\left(\begin{array}{cc}
1 & r \\
r & 1
\end{array}\right) \ , \qquad
-1 \leq r \leq +1 \ ,
\ee
%
which, by construction, is symmetric.
Its {\bf trace} amounts to $\mathrm{Tr}(\boldsymbol{R})=2$, while
its {\bf determinant} is $\det(\boldsymbol{R}) = 1-r^{2}$. 
Consequently, $\boldsymbol{R}$ is regular
%and positive definite
as long as $r \neq \pm 1$. We seek to determine the {\bf 
eigenvalues} and corresponding 
{\bf eigenvectors} (or {\bf principal components}) of 
$\boldsymbol{R}$, i.e., real numbers~$\lambda$ and real-valued 
vectors~$\vec{v}$ such that the condition
%
\be
\lb{eq:eigenv}
\boldsymbol{R}\,\vec{v} \stackrel{!}{=} \lambda\,\vec{v}
\qquad\Leftrightarrow\qquad
(\boldsymbol{R}-\lambda\boldsymbol{1})\,\vec{v}
\stackrel{!}{=} \boldsymbol{0}
\ee
%
applies. The determination of non-trivial solutions of this 
algebraic problem leads to the {\bf characteristic equation}
%
\be
0 \stackrel{!}{=} \det(\boldsymbol{R}-\lambda\boldsymbol{1})
= (1-\lambda)^{2} - r^{2} = (\lambda-1)^{2} - r^{2} \ .
\ee
%
Hence, by completing squares, it is clear that $\boldsymbol{R}$ 
possesses the two {\bf eigenvalues}
%
\be
\lambda_{1} = 1+r \qquad\text{and}\qquad
\lambda_{2} = 1-r \ ,
\ee
%
showing that $\boldsymbol{R}$ is {\bf positive-definite} whenever
$|r| < 1$. The normalised {\bf eigenvectors} associated with 
$\lambda_{1}$ and $\lambda_{2}$, obtained from 
Eq.~(\ref{eq:eigenv}), then are
%
\be
\vec{v}_{1} = \frac{1}{\sqrt{2}}
\left(\begin{array}{c}
1 \\ 1
\end{array}\right) \qquad\text{and}\qquad
\vec{v}_{2} = \frac{1}{\sqrt{2}}
\left(\begin{array}{r}
- 1\\ 1
\end{array}\right) \ ,
\ee
%
and constitute a right-handedly oriented basis of the 
two-dimensional {\bf eigenspace} of $\boldsymbol{R}$. 
Note that due to the symmetry of $\boldsymbol{R}$ it holds that 
$\vec{v}_{1}^{T}\cdot\vec{v}_{2}=0$, i.e., the eigenvectors are 
mutually orthogonal.

\medskip
\noindent
The normalised eigenvectors of $\boldsymbol{R}$ define a regular
orthogonal {\bf transformation matrix} $\boldsymbol{M}$, and an
inverse $\boldsymbol{M}^{-1}=\boldsymbol{M}^{T}$, given by resp.
%
\be
\boldsymbol{M} =
\frac{1}{\sqrt{2}}\left(\begin{array}{cr}
1 & -1 \\
1 &  1
\end{array}\right)
\qquad\text{and}\qquad
\boldsymbol{M}^{-1} =
\frac{1}{\sqrt{2}}\left(\begin{array}{rc}
 1 & 1 \\
-1 & 1
\end{array}\right)
= \boldsymbol{M}^{T} \ ,
\ee
%
where $\mathrm{Tr}(\boldsymbol{M})=\sqrt{2}$ and 
$\det(\boldsymbol{M})=1$. The correlation matrix $\boldsymbol{R}$ 
can now be {\bf diagonalised} by means of a rotation with 
$\boldsymbol{M}$ according to\footnote{Alternatively one 
can write
%
\[
\boldsymbol{M} = \left(\begin{array}{cr}
\cos(\pi/4) & -\sin(\pi/4) \\
\sin(\pi/4) & \cos(\pi/4)
\end{array}\right) \ ,
\]
%
thus emphasising the character of a rotation of $\boldsymbol{R}$ 
by an angle $\varphi=\pi/4$.
}
%
\bea
\boldsymbol{R}_\mathrm{diag} & = & \boldsymbol{M}^{-1}\boldsymbol{R}
\boldsymbol{M} \nonumber \\
& = & \frac{1}{\sqrt{2}}\left(\begin{array}{rc}
 1 & 1 \\
-1 & 1
\end{array}\right)
\left(\begin{array}{cc}
1 & r \\
r & 1
\end{array}\right)
\frac{1}{\sqrt{2}}\left(\begin{array}{cr}
1 & -1 \\
1 &  1
\end{array}\right)
= \left(\begin{array}{cc}
1+r &  0 \\
0        &  1-r
\end{array}\right) \ .
\eea
%
Note that $\mathrm{Tr}(\boldsymbol{R}_\mathrm{diag})=2$ and
$\det(\boldsymbol{R}_\mathrm{diag}) = 1-r^{2}$, i.e., the trace 
and determinant of $\boldsymbol{R}$ remain {\bf invariant} under 
the diagonalising transformation.

\medskip
\noindent
The concepts of eigenvalues and 
eigenvectors (principal components), as well as of diagonalisation 
of symmetric matrices, generalise in a straightforward though 
computationally more demanding fashion to arbitrary real-valued 
{\bf correlation matrices} $\boldsymbol{R} \in \mathbb{R}^{m 
\times m}$, with $m \in \mathbb{N}$.

\medskip
\noindent
\underline{\R:} \texttt{prcomp(\textit{data matrix})}

%%%%%%%%%%%%%%%%%%%%%%%%%%%%%%%%%%%%%%%%%%%%%%%%%%%%%%%%%%%%%%%%%%%
%%%%%%%%%%%%%%%%%%%%%%%%%%%%%%%%%%%%%%%%%%%%%%%%%%%%%%%%%%%%%%%%%%%
