%%%%%%%%%%%%%%%%%%%%%%%%%%%%%%%%%%%%%%%%%%%%%%%%%%%%%%%%%%%%%%%%%%%
%  File name: intro.tex
%  Version: 25.08.2019 (hve)
%%%%%%%%%%%%%%%%%%%%%%%%%%%%%%%%%%%%%%%%%%%%%%%%%%%%%%%%%%%%%%%%%%%
\addcontentsline{toc}{chapter}{Introductory remarks}
%%%%%%%%%%%%%%%%%%%%%%%%%%%%%%%%%%%%%%%%%%%%%%%%%%%%%%%%%%%%%%%%%%%
\chapter*{Introductory remarks}
%%%%%%%%%%%%%%%%%%%%%%%%%%%%%%%%%%%%%%%%%%%%%%%%%%%%%%%%%%%%%%%%%%%
\textbf{Statistical methods of data analysis} form the cornerstone
of quantitative--empirical research in the \textbf{Social
Sciences}, \textbf{Humanities}, and \textbf{Economics}.
Historically, the bulk of knowledge available in
\textbf{Statistics} emerged in the context of the analysis of
(nowadays large) data sets from observational and experimental
measurements in the \textbf{ Natural Sciences}. The purpose of the
present lecture notes is to provide its readers with a solid and
thorough, though accessible introduction to the basic concepts of
\textbf{Descriptive and Inferential Statistics}. When discussing
methods relating to the latter subject, we will here take the
perspective of the \textbf{frequentist approach} to
\textbf{Probability Theory}. (See Ref.~\ct{hve2018} for a
methodologically different approach.)

\medskip
\noindent
The concepts to be introduced and the topics to be covered 
have been selected in order to make available a fairly 
self-contained basic statistical tool kit for thorough analysis 
at the \textbf{univariate} and \textbf{bivariate} levels of
complexity of data, gained by means of opinion polls, surveys or
observation.
%In this respect, the present lecture notes are intended to 
%specifically assist the teaching of statistical methods of data 
%analysis in the bachelor degree programmes offered 
%at Karlshochschule International University. In particular, the 
%contents have immediate relevance to solving problems of a 
%quantitative nature in either of %(i)~the year~1 and year~2
%third semester general management modules\\[-0.7cm]
%%
%\begin{itemize}
%\item \textbf{0.1.3 SCIE: Introduction to Scientific Research
%Methods}\\[-0.7cm]
%\item \textbf{0.3.2 RESO: Resources: Financial Resources,
%Human Resources, Organisation}.\\[-0.6cm]
%\end{itemize}
%%
%and (ii)~the year~2
%%third and fourth semester
%special module of the IB study programme\\[-0.8cm]
%%
%\begin{itemize}
%\item \textbf{1.3.2 MARE: Marketing Research}\\[-0.8cm]
%\item \textbf{1.4.2 INOP: International Operations}.\\[-0.8cm]
%\end{itemize}
%

\medskip
\noindent
In the \textbf{Social Sciences}, \textbf{Humanities}, and
\textbf{Economics} there are two broad families of empirical
research tools available for studying behavioural features of and
mutual interactions between human individuals on the one-hand side,
and the social systems and organisations that these form on the
other. \textbf{Qualitative--empirical methods} focus their view 
on the individual with the aim to account for her/his/its 
particular characteristic features, thus probing the ``small 
scale-structure'' of a social system, while
\textbf{quantitative--empirical methods} strive to recognise
patterns and regularities that pertain to a large number of
individuals and so hope to gain insight on the ``large-scale
structure'' of a social system.

\medskip
\noindent
Both approaches are strongly committed to pursuing the principles 
of the \textbf{scientific method}. These entail the systematic
observation and measurement of phenomena of interest on the basis 
of well-defined statistical variables, the structured analysis of 
data so generated, the attempt to provide compelling theoretical 
explanations for effects for which there exists conclusive 
evidence in the data, the derivation from the data of predictions 
which can be tested empirically, and the publication of all 
relevant data and the analytical and interpretational tools 
developed and used, so that the pivotal \textbf{replicability} of a 
researcher's findings and associated conclusions is ensured. By 
complying with these principles, the body of scientific knowledge 
available in any field of research and its practical applications
%and in practical applications of scientific insights,
undergoes a continuing process of updating and expansion.

\medskip
\noindent
Having thoroughly worked through these lecture notes, a reader 
should have obtained a good understanding of the use and 
efficiency of descriptive and frequentist inferential statistical
methods for handling quantitative issues, as they often arise in a
manager's everyday business life. Likewise, a reader should feel
well-prepared for a smooth entry into any Master degree programme
in the \textbf{Social Sciences} or \textbf{Economics} which puts
emphasis on quantitative--empirical methods.

\medskip
\noindent
Following a standard pedagogical concept, these lecture notes are 
split into three main parts: Part~I, comprising
Chapters~\ref{ch1} to \ref{ch5}, covers the basic considerations 
of \textbf{Descriptive Statistics}; Part~II, which consists 
of Chapters~\ref{ch6} to \ref{ch8}, introduces the foundations of 
\textbf{Probability Theory}. Finally, the material of Part~III, 
provided in Chapters~\ref{ch9} to \ref{ch13}, first reviews a
widespread method for operationalising latent statistical 
variables, and then introduces a number of standard uni- and 
bivariate analytical tools of \textbf{Inferential Statistics}
within the \textbf{frequentist framework} that prove valuable in 
applications. As such, the contents of Part~III are the most 
important ones for quantitative--empirical research work. Useful 
mathematical tools and further material have been gathered in 
appendices.

\medskip
\noindent
Recommended introductory textbooks, which may be used for study in 
parallel to these lecture notes, are Levin \textit{et al} 
(2010)~\ct{levetal2009}, Hatzinger and Nagel 
(2013)~\ct{hatnag2013}, Weinberg and Abramowitz
(2008)~\ct{weiabr2008}, Wewel (2014)~\ct{wew2014}, Toutenburg 
(2005) \ct{tou2005}, or Duller (2007)~\ct{dul2007}.
%These textbooks, as well as many of the monographs listed in the
%\textbf{bibliography}, are available in the library of 
%Karlshochschule International University.

\medskip
\noindent
There are \textit{not} included in these lecture notes any explicit 
exercises on the topics to be discussed. These are 
reserved for lectures given throughout term time.
%in any of the modules listed.

\medskip
\noindent
The present lecture notes are designed to be dynamical in 
character. On the one-hand side, this means that they will be 
updated on a regular basis. On the other, that the *.pdf version 
of the notes contains interactive features such as fully 
hyperlinked references to original publications at the websites 
\href{https://doi.org}{\texttt{doi.org}} and 
\href{http://www.jstor.org}{\texttt{jstor.org}}, as well as many 
active links to biographical information on scientists that have 
been influential in the historical development of
\textbf{Probability Theory} and \textbf{Statistics}, hosted by the
websites \href{http://www-history.mcs.st-and.ac.uk/}{The MacTutor
History of Mathematics archive
({\tt www-history.mcs.st-and.ac.uk})} and
\href{http://en.wikipedia.org/wiki/Main_Page}{\texttt{en.wikipedia.org}}.

\medskip
\noindent
Throughout these lecture notes references have been provided to
respective descriptive and inferential statistical functions and
routines that are available in the excellent and widespread
statistical software package \R{}, on a standard graphic display
calculator (GDC), and in the statistical software packages EXCEL,
OpenOffice and SPSS (Statistical Program for the Social Sciences).
\R{}~and its exhaustive documentation are distributed by the
R~Core Team (2019)~\ct{rct2019} via the website
\href{http://cran.r-project.org}{\texttt{cran.r-project.org}}.
\R{}, too, has been employed for generating all the figures
contained in these lecture notes. Useful and easily accessible
textbooks on the application of \R{} for statistical data analysis
are, e.g., Dalgaard (2008)~\ct{dal2008}, or Hatzinger \textit{et
al} (2014)~\ct{hatetal2014}. Further helpful information and 
assistance is available from the website 
\href{http://www.r-tutor.com/}{\texttt{www.r-tutor.com}}. For
active statistical data analysis with \R{}, we strongly recommend
the use of the convenient custom-made work environment \R{}~Studio,
provided free of charge at \href{http://www.rstudio.com}{\texttt{www.rstudio.com}}. Another user friendly statistical software
package is GNU PSPP. This is available as shareware from 
\href{http://www.gnu.org/software/pspp/}{\texttt{www.gnu.org/software/pspp/}}.

\medskip
\noindent
A few examples from the inbuilt \R{} data sets package have
been related to in these lecture notes in the context of the
visualisation of distributional features of statistical data.
Further information on these data sets can be obtained by typing
\texttt{library(help = "datasets")} at the \R{} prompt.

\medskip
\noindent
Lastly, we hope the reader will discover something useful or/and
enjoyable for her/him-self when working through these lecture
notes. Constructive criticism is always welcome.

\vfill
\medskip
\noindent
\textit{Acknowledgments:} I am grateful to Kai Holschuh, Eva 
Kunz and Diane Wilcox for valuable comments on an earlier draft of 
these lecture notes, to Isabel Passin for being a critical sparing
partner in evaluating pedagogical considerations concerning
cocreated accompanying lectures, and to Michael R\"{u}ger for
compiling an initial list of online survey tools for the Social
Sciences.

%%%%%%%%%%%%%%%%%%%%%%%%%%%%%%%%%%%%%%%%%%%%%%%%%%%%%%%%%%%%%%%%%%%
%%%%%%%%%%%%%%%%%%%%%%%%%%%%%%%%%%%%%%%%%%%%%%%%%%%%%%%%%%%%%%%%%%%