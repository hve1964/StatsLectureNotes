%%%%%%%%%%%%%%%%%%%%%%%%%%%%%%%%%%%%%%%%%%%%%%%%%%%%%%%%%%%%%%%%%%%
%  File name: ch11-Stats.tex
%  Title:
%  Version: 12.08.2019 (hve)
%%%%%%%%%%%%%%%%%%%%%%%%%%%%%%%%%%%%%%%%%%%%%%%%%%%%%%%%%%%%%%%%%%%
%%%%%%%%%%%%%%%%%%%%%%%%%%%%%%%%%%%%%%%%%%%%%%%%%%%%%%%%%%%%%%%%%%%
\chapter[Null hypothesis significance testing]{\href{https://www.youtube.com/watch?v=jnLHSyYQ1LE}{Null hypothesis significance testing}}
\lb{ch11}
%%%%%%%%%%%%%%%%%%%%%%%%%%%%%%%%%%%%%%%%%%%%%%%%%%%%%%%%%%%%%%%%%%%
\textbf{Null hypothesis significance testing} by means of
observable quantities is the centrepiece of the current body of
inferential statistical methods in the \textbf{frequentist
framework}. Its logic of an ongoing routine of systematic
\textbf{falsification} of null hypotheses by empirical means is
firmly rooted in the ideas of \textbf{critical rationalism} and
\textbf{logical positivism}. The latter were expressed most
emphatically by the Austro--British 
philosopher \href{http://en.wikipedia.org/wiki/Karl_Popper}{Sir 
Karl Raimund Popper CH FRS FBA (1902--1994)}; see, e.g., Popper 
(2002)~\ct{pop2002}. The systematic procedure for \textbf{null
hypothesis significance testing} on the grounds of observational
\textbf{evidence}, as practiced today within the
\textbf{frequentist framework} as a standardised method of
probability-based \textbf{decision-making}, was developed during
the first half of the $20^\mathrm{th}$ Century, predominantly by
the English statistician, evolutionary biologist, 
eugenicist and geneticist
\href{http://www-history.mcs.st-and.ac.uk/Biographies/Fisher.html}{Sir
Ronald Aylmer Fisher FRS (1890--1962)}, the 
Polish--US-American mathematician and statistician
\href{http://www-history.mcs.st-and.ac.uk/Biographies/Neyman.html}{Jerzy Neyman (1894--1981)}, the English mathematician and statistician
\href{http://www-history.mcs.st-and.ac.uk/Biographies/Pearson.html}{Karl Pearson FRS (1857--1936)}, and his son, the English statistician
\href{http://www-history.mcs.st-and.ac.uk/Biographies/Pearson_Egon.html}{Egon Sharpe Pearson CBE FRS (1895--1980)}; cf. Fisher
(1935)~\ct{fis1935}, Neyman and Pearson (1933)~\ct{neypea1933},
and Pearson (1900)~\ct{pea1900}.
%, and Pearson (1920)~\ct{pea1920}. 
We will describe the main steps of the systematic test procedure 
in the following.

%%%%%%%%%%%%%%%%%%%%%%%%%%%%%%%%%%%%%%%%%%%%%%%%%%%%%%%%%%%%%%%%%%%
\section[General procedure]{General procedure}
\lb{sec:testgen}
%%%%%%%%%%%%%%%%%%%%%%%%%%%%%%%%%%%%%%%%%%%%%%%%%%%%%%%%%%%%%%%%%%%
The central aim of \textbf{null hypothesis significance testing} is
to separate, as reliably as possible, \textbf{true effects} in a
\textbf{target population} $\boldsymbol{\Omega}$ of statistical
units concerning distributional properties of, or relations
between, selected \textbf{statistical variables} $X, Y, \ldots, Z$
from \textbf{chance effects} potentially injected by the sampling
approach to probing the nature of $\boldsymbol{\Omega}$. The
sampling approach results in a, generally unavoidable, state of
\textbf{incomplete information} on the part of the researcher.

\medskip
\noindent
In an inferential statistical context, (null and/or research)
\textbf{hypotheses} are formulated as assumptions on\\[-5mm]
%
\begin{itemize}
\item[(i)] the \textbf{probability distribution function} $F$ of
one or more \textbf{random variables} $X, Y, \ldots, Z$ in 
$\boldsymbol{\Omega}$, or on\\[-5mm]

\item[(ii)] one or more \textbf{parameters} $\theta$ of this
probability distribution function.\\[-5mm]
\end{itemize}
%
Generically, statistical hypotheses need to be viewed as 
probabilistic statements. As such the researcher will always have 
to deal with a fair amount of \textbf{uncertainty} in deciding 
whether an observed, potentially only apparent effect is
\textbf{statistically significant} and/or \textbf{practically
significant} in $\boldsymbol{\Omega}$ or not. Bernstein
(1998)~\ct[p~207]{ber1998} summarises the circumstances relating to
the test of a specific hypothesis as follows:
%
\begin{quotation}
``Under conditions of uncertainty, the choice is not between 
rejecting a hypothesis and accepting it, but between reject and 
not--reject.''
\end{quotation}
%

\medskip
\noindent
The question arises as to \textit{which kinds of quantitative 
problems can be efficiently settled by statistical means}?
With respect to a given target population $\boldsymbol{\Omega}$, 
in the simplest kinds of applications of \textbf{null hypothesis
significance testing}, one may (a)~\textbf{test for differences} in
the distributional properties of a single one-dimensional
statistical variable $X$ between a number of subgroups of
$\boldsymbol{\Omega}$, necessitating \textbf{univariate methods} of
data analysis, or one may (b)~\textbf{test for association} for a
two-dimensional statistical variable $(X,Y)$, thus requiring
\textbf{bivariate methods} of data analysis. The standardised
procedure for \textbf{null hypothesis significance testing},
practiced within the \textbf{frequentist framework} for the purpose
of assessing statistical significance of an observed, potentially
apparent effect, takes the following six steps on the way to making
a \textbf{decision}:

\medskip
\noindent
\textbf{Six-step procedure for null hypothesis significance
testing}%\\[-5mm]
%
\begin{enumerate}

\item Formulation, with respect to the target population 
$\boldsymbol{\Omega}$, of a pair of mutually exclusive
\textbf{hypotheses}:%\\[-5mm]
%
\begin{enumerate}
\item the \textbf{null hypothesis} $H_{0}$ conjectures that ``there 
exists \textit{no} effect in $\boldsymbol{\Omega}$ of the kind 
envisaged by the researcher,'' while%\\[-5mm]

\item the \textbf{research hypothesis} $H_{1}$ conjectures  that 
``there \textit{does} exist a true effect in $\boldsymbol{\Omega}$ 
of the kind envisaged by the researcher.''%\\[-5mm]
\end{enumerate}
%
The starting point of the test procedure is the \textit{assumption 
(!)} that it is the content of the $H_{0}$ conjecture which is 
realised in $\boldsymbol{\Omega}$. The objective is to try to 
refute $H_{0}$ empirically on the basis of random sample data 
drawn from $\boldsymbol{\Omega}$, to a level of significance which 
needs to be specified in advance. In this sense it is $H_{0}$ 
which is being subjected to a statistical test.\footnote{Bernstein 
(1998)~\ct[p~209]{ber1998} refers to the statistical test of a 
(null) hypothesis as a ``mathematical stress test.''} The striking 
\textit{asymmetry} regarding the roles of $H_{0}$ and $H_{1}$ in 
the test procedure embodies the notion of a \textbf{falsification}
of hypotheses, as advocated by critical rationalism.%\\[-5mm]

\item Specification of a \textbf{significance level} $\alpha$ prior
to the performance of the test, where, by convention, $\alpha \in 
[0.01,0.05]$. The parameter $\alpha$ is synonymous with the 
probability of committing a Type~I error (to be defined below) in 
making a test decision.%\\[-5mm]

\item Construction of a suitable continuous real-valued measure 
for quantifying deviations of the data in a random sample 
$\boldsymbol{S_{\Omega}}$:~$(X_{1}, \ldots, X_{n})$ of size~$n$ 
from the initial ``no effect in $\boldsymbol{\Omega}$'' conjecture
of $H_{0}$, a \textbf{test statistic} $T_{n}(X_{1},\ldots, X_{n})$ 
that is perceived as a one-dimensional random variable with (under 
the $H_{0}$ assumption) \textit{known (!)} associated
\textbf{theoretical probability distribution} for computing related
event probabilities. The latter is referred to as the \textbf{test
distribution}.\footnote{Within the frequentist framework of
null hypothesis significance testing the test statistic and its
partner test distribution form an intimate pair of decision-making
devices.}%\\[-5mm]

\item Determination of the \textbf{rejection region}
$B_{\alpha}$ for $H_{0}$ within the spectrum of values of 
the test statistic $T_{n}(X_{1},\ldots, X_{n})$ from re-arranging 
the conditional probability condition
%
\be
P\left(T_{n}(X_{1}, \ldots, X_{n}) \in B_{\alpha}|H_{0}\right)
\stackrel{!}{\leq} \alpha \ ,
\ee
%
where $P(\ldots)$ and the threshold
$\alpha$--quantile(s)~$P^{-1}(\alpha)$ demarking the boundary(ies)
of $B_{\alpha}$ are to be calculated from the assumed (continuous)
test distribution.

\item Computation of a specific \textbf{realisation}
$t_{n}(x_{1}, \ldots, x_{n})$ of the test statistic
$T_{n}(X_{1}, \ldots, X_{n})$ from the data $x_{1}, 
\ldots, x_{n}$ in a \textbf{random sample} 
$\boldsymbol{S_{\Omega}}$:~$(X_{1}, \ldots, X_{n})$, the latter of 
which constitutes the required observational
\textbf{evidence}.%\\[-5mm]

\item Derivation of a \textbf{test decision} on the basis of the 
following alternative criteria: when for the realisation 
$t_{n}(x_{1}, \ldots, x_{n})$ of the test statistic $T_{n}(X_{1}, 
\ldots, X_{n})$, resp.~the $p$--value (to be defined in 
Sec.~\ref{sec:pvalue} below) associated with this 
realisation,\footnote{The statistical software packages \R{}\ and
SPSS provide $p$--values as a means for making decisions in null
hypothesis significance testing.} it holds that%\\[-5mm]
%
\begin{itemize}
\item[(i)]\ $t_{n} \in B_{\alpha}$, resp.
$\boldsymbol{p}\text{{\bf--value}} < \alpha, \text{then}
\quad\Rightarrow\quad
\text{reject}~H_{0}$,\\[-5mm]
\item[(ii)]\ $t_{n} \notin B_{\alpha}$, resp.
$\boldsymbol{p}\text{{\bf--value}} \geq \alpha, \text{then}
\quad\Rightarrow\quad\text{not\ reject}~H_{0}$.%\\[-5mm]
\end{itemize}
%
\end{enumerate}
%

\medskip
\noindent
A fitting metaphor for the six-step procedure for \textbf{null
hypothesis significance testing} just described is that of a
statistical long jump competition. The issue here is to find out
whether actual empirical data deviates sufficiently strongly from
the ``no effect'' reference state conjectured in the
given \textbf{null hypothesis}~$H_{0}$, so as to land in the 
corresponding \textbf{rejection region}~$B_{\alpha}$ within the 
spectrum of values of the \textbf{test 
statistic}~$T_{n}(X_{1}, \ldots, X_{n})$. Steps~1 to~4 prepare the 
long jump facility (the test stage), while the evaluation of the 
outcome of the jump attempt takes place in steps~5 and~6. Step~4 
necessitates the direct application of \textbf{Probability Theory}
within the \textbf{frequentist framework} in that the determination
of the \textbf{rejection region}~$B_{\alpha}$ for~$H_{0}$ entails
the calculation of a conditional event probability from an
\textit{assumed} \textbf{test distribution}.

\medskip
\noindent
When an effect observed on the basis of random sample data proves 
to possess \textbf{statistical signifi\-cance} (to a predetermined 
significance level), this means that most likely it has come about 
\textit{not by chance} due to the sampling methodology. A different 
matter altogether is whether such an effect also possesses
\textbf{practical significance}, so that, for instance, management 
decisions ought to be adapted to it. \textbf{Practical
significance} of an observed effect can be evaluated, e.g., with
the standardised and scale-invariant \textbf{effect size} measures
proposed by Cohen (1992, 2009)~\ct{coh1992,coh2009}. Addressing
the \textbf{practical significance} of an observed effect should be 
commonplace in any report on inferential statistical data analysis;
see also Sullivan and R Feinn (2012)~\ct{sulfei2012}.

\medskip
\noindent
When performing \textbf{null hypothesis significance testing}, the
researcher is always at \textbf{risk} of making a wrong 
decision. Hereby, one distinguishes between the following two 
kinds of potential error:
%
\begin{itemize}
\item \textbf{Type I error:} reject an $H_{0}$ which, however,
is true, with conditional probability $P(H_{1}|H_{0}\ 
\text{true})=\alpha$; this case is also referred to as a ``false 
positive,'' and %\\

\item \textbf{Type II error:} not reject an $H_{0}$ which, however,
is false, with conditional probability $P(H_{0}|H_{1}\ 
\text{true})=\beta$; this case is also referred to as a ``false 
negative.''

\end{itemize}
%
By fixing the significance level $\alpha$ prior to running 
a statistical test, one controls the risk of committing a Type~I 
error in the decision process. We condense the different possible 
outcomes when making a test decision in
%the following table:
Tab.~\ref{tab:decerrors}.
%\pagebreak
%\medskip
%\noindent
%
\begin{table}
%\textbf{Consequences of test decisions:}% \nopagebreak
%
\begin{center}
    \begin{tabular}[!h]{c|ccc}
    	\hline
    	 & & & \\
    	 & $H_{0}$: \textit{no effect} & \fbox{Decision for:} & 
    	 $H_{1}$: \textit{effect} \\
    	 & & & \\
    	\hline
    	 & & & \\
    	$H_{0}$: \textit{no effect} & correct decision: & &
			\textbf{Type I error}: \\
    	true & $P(H_{0}|H_{0}\ \text{true})=1-\alpha$ & & 
    	$P(H_{1}|H_{0}\ \text{true})=\alpha$ \\
    	 & & & \\
    	\fbox{Reality / $\boldsymbol{\Omega}$:} & & & \\
    	 & & & \\
    	$H_{1}$: \textit{effect} & \textbf{Type II error}: & &
			correct decision: \\
    	true & $P(H_{0}|H_{1}\ \text{true})=\beta$ & & 
    	$P(H_{1}|H_{1}\ \text{true})=1-\beta$ \\
    	 & & & \\
      \hline
    \end{tabular}
\end{center}
\caption{Consequences of test decisions in null hypothesis
significance testing.}
\lb{tab:decerrors}
\end{table}
%

\medskip
\noindent
While the probability $\alpha$ is required to be specified
\textit{a priori} to a statistical test, the probability $\beta$ is 
typically computed \textit{a posteriori}. One refers to the 
probability $1-\beta$ associated with the latter as the
\textbf{power} of a statistical test. Its magnitude is determined
in particular by the parameters \textbf{sample size}~$n$,
\textbf{significance level}~$\alpha$, and the \textbf{effect size}
of the phenomenon to be investigated; see, e.g., Cohen 
(2009)~\ct{coh2009} and Hair \textit{et al} 
(2010)~\ct[p~9f]{haietal2010}.

%\textbf{[hve: Zielscheibenmetapher zum Prinzip des Signifikanztests:
%\"au\ss ere 95\% zu treffen ist recht wahrscheinlich (entspricht
%Stichproben bedingten Schwankungen), innere 5\% zu treffen ist
%sehr unwahrscheinlich (entspricht systematischen/signifikanten
%Schwankungen)]}

\medskip
\noindent
As emphasised at the beginning of this chapter, \textbf{null
hypothesis significance testing} is at the heart of
quantitative--empirical research rooted in the
\textbf{frequentist framework}. To foster scientific progress in
this context, it is essential that the scientific community, in an
act of self-control, aims at repeated \textbf{replication} of
specific test results in independent investigations. An interesting
article in this respect was published by the weekly magazine
\texttt{The Economist} on Oct 19, 2013, see Ref.~\ct{eco2013},
which points out that, when subjected to such scrutiny, in general
negative empirical results ($H_{0}$ not rejected) prove much more
reliable than positive ones ($H_{0}$ rejected), though scientific
journals tend to have a bias towards publication of the latter.
A similar viewpoint is expressed in the paper by Nuzzo
(2014)~\ct{nuz2014}. Rather critical accounts of the conceptual
foundations of null hypothesis significance testing are given in
the works by Gill (1999)~\ct{gil1999} and by Kruschke and Liddell
(2017)~\ct{krulid2017}.

\medskip
\noindent
The complementary \textbf{Bayes--Laplace approach} to 
\textbf{statistical data analysis} (cf. Sec. \ref{subsec:bayes}) 
does neither require the prior specification of a significance 
level~$\alpha$, nor the introduction of a test statistic 
$T_{n}(X_{1}, \ldots, X_{n})$ with a partner test distribution for
the empirical testing of a (null) hypothesis. As described in
detail by Jeffreys (1939)~\ct{jef1939}, Jaynes (2003)~\ct{jay2003},
Sivia and Skilling (2006)~\ct{sivski2006}, Gelman \textit{et al}
(2014)~\ct{geletal2014} or McElreath (2016)~\ct{mce2016}, here
\textbf{statistical inference} is practiced entirely on the basis
of a \textbf{posterior probability distribution}
$P(\text{hypothesis}|\text{data}, I)$ for
the (research) hypothesis to be tested, conditional on the
empirical data that was analysed for this purpose, and on the
``relevant background information~$I$'' available to the researcher 
beforehand. By employing \textbf{Bayes' theorem} [cf. 
Eq.~(\ref{eq:bayes})], this \textbf{posterior probability 
distribution} is computed in particular from the product between 
the \textbf{likelihood function} $P(\text{data}|\text{hypothesis}, 
I)$ of the data, given the hypothesis and $I$, and the
\textbf{prior probability distribution} $P(\text{hypothesis}, I)$ 
encoding the researcher's initial reasonable
\textbf{degree-of-belief} in the truth content of the hypothesis on
the backdrop of $I$. That is (see Sivia and Skilling
(2006)~\ct[p~6]{sivski2006}),
%
\be
P(\text{hypothesis}|\text{data}, I) \propto 
P(\text{data}|\text{hypothesis}, I) \times P(\text{hypothesis}, I) 
\ .
\ee
%
%This procedure thus requires information on (i)~the \textbf{joint 
%probability distribution} for distribution parameters and random 
%variables (with parameters treated as random variables) and the 
%distribution of random sample data underlying the probability 
%$P(\text{data}|\text{hypothesis}, I)$, as well as specification of 
%(ii)~a prior probability  $P(\text{hypothesis}, I)$. The latter can
%be interpreted as a representation of a researcher's
%\textbf{state of knowledge} concerning the plausibility of the
%hypothesis to be tested.

\medskip
\noindent
The \textbf{Bayes--Laplace approach} can be viewed as a proposal to 
the formalisation of the process of \textbf{learning}. Note that
the posterior probability distribution of one round of data
generation and analysis can serve as the prior probability
distribution for a subsequent round of generation and analysis of
new data. Further details on the principles within the
\textbf{Bayes--Laplace framework} underlying the estimation of
distribution parameters, the optimal curve-fitting to a given set
of empirical data points, and the related selection of an adequate mathematical model are given in, e.g., Greenberg
(2013)~\ct[Chs.~3~and~4]{gre2013}, Saha (2002)~\ct[p~8ff]{sah2002},
Lupton (1993)~\ct[p~50ff]{lup1993}, and in Ref.~\ct{hve2018}.

%%%%%%%%%%%%%%%%%%%%%%%%%%%%%%%%%%%%%%%%%%%%%%%%%%%%%%%%%%%%%%%%%%%
\section[Definition of a $p$--value]{Definition of a 
$\boldsymbol{p}$--value}
\lb{sec:pvalue}
%%%%%%%%%%%%%%%%%%%%%%%%%%%%%%%%%%%%%%%%%%%%%%%%%%%%%%%%%%%%%%%%%%%
\underline{\textbf{Def.:}} Let $T_{n}(X_{1}, \ldots, X_{n})$ be
the \textbf{test statistic} of a particular \textbf{null hypothesis
significance test} in the \textbf{frequentist framework}. The
\textbf{test distribution} associated with $T_{n}(X_{1}, \ldots,
X_{n})$ \textit{be known} under the assumption that the null
hypothesis $H_{0}$ holds true in the target
population~$\boldsymbol{\Omega}$. 
%If $t_{n}(x_{1}, \ldots, x_{n})$ is the \textbf{realisation} of 
%$T_{n}(X_{1}, \ldots, X_{n})$ computed from the data $x_{1}, 
%\ldots, x_{n}$ of a random sample 
%$\boldsymbol{S_{\Omega}}$:~$(X_{1}, \ldots, X_{n})$, then the 
%$\boldsymbol{p}${\bf--value} associated with $t_{n}(x_{1}, %\ldots, 
%x_{n})$ is defined as the conditional probability of obtaining a 
%value of $T_{n}(X_{1}, \ldots, X_{n})$ which is {\em more 
%extreme\/} than the given $t_{n}(x_{1}, \ldots, x_{n})$, given 
%that the null hypothesis applies.
The $\boldsymbol{p}${\bf--value} associated with a \textbf{
realisation} $t_{n}(x_{1}, \ldots, x_{n})$ of the test statistic 
$T_{n}(X_{1}, \ldots, X_{n})$ is defined as the conditional 
probability of finding a value for $T_{n}(X_{1}, \ldots, X_{n})$  
which is \textit{equal to or more extreme} than the actual 
realisation $t_{n}(x_{1}, \ldots, x_{n})$, given that the null 
hypothesis $H_{0}$ applies in the target 
population~$\boldsymbol{\Omega}$. This conditional probability is
to be computed from the test distribution.

\vspace{5mm}
\noindent
Specifically, using the computational rules 
(\ref{eq:comprulescont2})--(\ref{eq:comprulescont4}), one obtains 
for a
%
\begin{itemize}
\item two-sided statistical test,
%
\begin{eqnarray}
\lb{eq:pvaluetwo}
p & := & P(T_{n}<-\left.|t_{n}|\right|H_{0})
+ P(T_{n}>\left.|t_{n}|\right|H_{0}) \nonumber\\
& = & P(T_{n}<-\left.|t_{n}|\right|H_{0})
+ 1 - P(T_{n}\leq\left.|t_{n}|\right|H_{0}) \nonumber\\
& = & F_{T_{n}}(-|t_{n}|) + 1 - F_{T_{n}}(|t_{n}|) \ .
\end{eqnarray}
%
This result specialises to $p=2\left[1 - F_{T_{n}}(|t_{n}|)\right]$
if the respective \texttt{pdf} of the test distribution exhibits
\textbf{reflection symmetry} with respect to a vertical axis at
$t_{n}=0$, i.e., when 
$F_{T_{n}}(-|t_{n}|)=1 - F_{T_{n}}(|t_{n}|)$ holds.

\item left-sided statistical test,
%
\be
\lb{eq:pvalueleft}
p := P(T_{n}<t_{n}|H_{0})
= F_{T_{n}}(t_{n}) \ ,
\ee
%

\item right-sided statistical test,
%
\be
\lb{eq:pvalueright}
p := P(T_{n}>t_{n}|H_{0})
= 1 - P(T_{n}\leq t_{n}|H_{0})
= 1 - F_{T_{n}}(t_{n}) \ .
\ee
%
\end{itemize}
%
With respect to the \textbf{test decision criterion} of rejecting 
an $H_{0}$ whenever $p < \alpha$, one refers to (i)~cases with 
$p<0.05$ as \textbf{significant} test results, and to (ii)~cases
with $p<0.01$ as \textbf{highly significant} test
results.\footnote{Lakens (2017)~\ct{lak2017} posted a stimulating
blog entry on the potential traps associated with the
interpretation of a $p$--value in statistical data analysis. His
remarks come along with illustrative demonstrations in \R{},
including the underlying codes.}

\medskip
\noindent
\underline{Remark:} User-friendly routines for the computation of 
$p$--values are available in \R{}, SPSS, EXCEL and OpenOffice, and 
also on some GDCs.

\vspace{5mm}
\noindent
In the following two chapters, we will turn to discuss a number of 
standard problems in \textbf{Inferential Statistics} within the 
\textbf{frequentist framework}, in association with the 
quantitative--empirical tools that have been developed in this
context to tackle them. In Ch.~\ref{ch12} we will be concerned with
problems of a \textbf{univariate} nature, in particular,
\textbf{testing for statistical differences} in the distributional
properties of a single one-dimensional statistical variable~$X$
between two of more subgroups of some target
population~$\boldsymbol{\Omega}$, while in Ch.~\ref{ch13} the
problems at hand will be of a \textbf{bivariate} nature,
\textbf{testing for statistical association} in
$\boldsymbol{\Omega}$ for a two-dimensional statistical
variable~$(X,Y)$. An entertaining exhaustive account of the history
of statistical methods of data analysis prior to the year 1900 is
given by Stigler (1986)~\ct{sti1986}.

%%%%%%%%%%%%%%%%%%%%%%%%%%%%%%%%%%%%%%%%%%%%%%%%%%%%%%%%%%%%%%%%%%%
%%%%%%%%%%%%%%%%%%%%%%%%%%%%%%%%%%%%%%%%%%%%%%%%%%%%%%%%%%%%%%%%%%%
