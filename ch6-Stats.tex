%%%%%%%%%%%%%%%%%%%%%%%%%%%%%%%%%%%%%%%%%%%%%%%%%%%%%%%%%%%%%%%%%%%
%  File name: ch6-Stats.tex
%  Title:
%  Version: 10.08.2019 (hve)
%%%%%%%%%%%%%%%%%%%%%%%%%%%%%%%%%%%%%%%%%%%%%%%%%%%%%%%%%%%%%%%%%%%
%%%%%%%%%%%%%%%%%%%%%%%%%%%%%%%%%%%%%%%%%%%%%%%%%%%%%%%%%%%%%%%%%%%
\chapter[Elements of probability theory]{Elements of probability
theory}
\lb{ch6}
%%%%%%%%%%%%%%%%%%%%%%%%%%%%%%%%%%%%%%%%%%%%%%%%%%%%%%%%%%%%%%%%%%%
All examples of \textbf{inferential statistical methods of data 
analysis} to be presented in Chs.~\ref{ch12} and \ref{ch13} have 
been developed in the context of the so-called \textbf{frequentist 
approach} to \textbf{Probability Theory}.\footnote{The origin of
the term ``probability'' is traced back to the Latin word 
\textit{probabilis}, which the Roman philosopher 
\href{https://en.wikipedia.org/wiki/Cicero}{Cicero (106 BC--43 
BC)} used to capture a notion of plausibility or likelihood; see 
Mlodinow (2008)~\ct[p~32]{mlo2008}.} The issue in
\textbf{Inferential Statistics} is to estimate the plausibility or
likelihood of hypotheses given the observational \textbf{evidence}
for them. The \textbf{frequentist approach} was pioneered by the
Italian mathematician, physician, astrologer, philosopher and
gambler 
\href{http://www-history.mcs.st-and.ac.uk/Biographies/Cardan.html}{Girolamo Cardano (1501--1576)}, the French lawyer and amateur mathematician  
\href{http://www-history.mcs.st-and.ac.uk/Biographies/Fermat.html}{Pierre
de Fermat (1601--1665)}, the French mathematician, 
physicist, inventor, writer and Catholic philosopher 
\href{http://www-history.mcs.st-and.ac.uk/Biographies/Pascal.html}{Blaise Pascal (1623--1662)}, the Swiss mathematician
\href{http://www-history.mcs.st-and.ac.uk/Biographies/Bernoulli_Jacob.html}{Jakob Bernoulli (1654--1705)}, and the French 
mathematician and astronomer
\href{http://www-history.mcs.st-and.ac.uk/Biographies/Laplace.html}{Marquis Pierre Simon de Laplace (1749--1827)}. It is deeply 
rooted in the two fundamental assumptions that any particular 
\textbf{random experiment} can be repeated arbitrarily often 
(i)~under the ``same conditions,'' and (ii)~completely 
``independent of one another,'' so that a theoretical basis is 
given for defining allegedly ``\textit{objective probabilities}''
for random events and hypotheses via the \textbf{relative
frequencies} of very long sequences of repetition of the same
random experiment.\footnote{A special role in the context of the
frequentist approach to Probability Theory is assumed by Jakob
Bernoulli's law of large numbers, as well as the concept of
independently and identically distributed 
(in short: ``i.i.d.'') random variables; we will discuss these 
issues in Sec.~\ref{sec:zentrgrenz} below.} This is a highly 
idealised viewpoint, however, which shares only a limited degree 
of similarity with the actual conditions pertaining to an 
observer's resp.~experimentor's reality. Renowned textbooks 
adopting the \textbf{frequentist viewpoint} of \textbf{Probability
Theory} and  \textbf{Inferential Statistics} are, e.g., Cram\'{e}r 
(1946)~\ct{cra1946} and Feller (1968)~\ct{fel1968}.

\medskip
\noindent
Not everyone in \textbf{Statistics} is entirely happy, though, with 
the philosophy underlying the \textbf{frequentist approach} to
introducing the concept of \textbf{probability}, as a number of its
central ideas rely on unobserved data (information). A
complementary viewpoint is taken by the framework which originated
from the work of the English mathematician and Presbyterian
minister \href{http://www-history.mcs.st-and.ac.uk/Biographies/Bayes.html}{Thomas Bayes (1702--1761)}, and later of Laplace, and so is
commonly referred to as the \textbf{Bayes--Laplace approach}; cf. 
Bayes (1763)~\ct{bay1763} and Laplace (1812)~\ct{lap1812}. A 
striking conceptual difference to the \textbf{frequentist approach} 
consists in its use of prior, allegedly ``\textit{subjective
probabilities}'' for random events and hypotheses, quantifying a
persons's individual reasonable \textbf{degree-of-belief} in their 
likelihood, which are subsequently updated by analysing relevant
empirical data.\footnote{Anscombe and Aumann (1963)~\ct{ansaum1963}
in their seminal paper refer to ``objective probabilities'' as
associated with ``roulette lotteries,'' and to ``subjective
probabilities'' as associated with ``horse lotteries.'' Savage
(1954)~\ct{sav1954} employs the alternative terminology of
distinguishing between ``objectivistic probabilities'' and
``personalistic probabilities.''} Renowned textbooks adopting the
\textbf{Bayes--Laplace viewpoint} of \textbf{Probability Theory}
and \textbf{Inferential Statistics} are, e.g., Jeffreys
(1939)~\ct{jef1939} and Jaynes (2003)~\ct{jay2003}, while general
information regarding the  \textbf{Bayes--Laplace approach} is
available from the website \href{http://bayes.wustl.edu/}{\texttt{bayes.wustl.edu}}. More recent textbooks, which assist in the
implementation of advanced computational routines, have been issued
by Gelman \textit{et al} (2014)~\ct{geletal2014} and by McElreath
(2016)~\ct{mce2016}. A discussion of the pros and cons of either of
these two competing approaches to \textbf{Probability Theory} can
be found, e.g., in Sivia and Skilling (2006)
\ct[p~8ff]{sivski2006}, or in Gilboa (2009) \ct[Sec.~5.3]{gil2009}.

\medskip
\noindent
A common denominator of both frameworks, \textbf{frequentist} and 
\textbf{Bayes--Laplace}, is the attempt to quantify a notion of 
\textbf{uncertainty} that can be related to in formal treatments of 
\textbf{decision-making}. In the following we turn to discuss the 
general principles on which \textbf{Probability Theory} is built.

%%%%%%%%%%%%%%%%%%%%%%%%%%%%%%%%%%%%%%%%%%%%%%%%%%%%%%%%%%%%%%%%%%%
\section[Random events]{Random events}
\lb{sec:zufall}
%%%%%%%%%%%%%%%%%%%%%%%%%%%%%%%%%%%%%%%%%%%%%%%%%%%%%%%%%%%%%%%%%%%
We begin by introducing some basic formal constructions and 
corresponding terminology used in the \textbf{frequentist 
approach} to \textbf{Probability Theory}:
%
\begin{itemize}

\item \textbf{Random experiments}: Random experiments are
experiments which can be repeated arbitrarily often under identical 
conditions, with \textbf{events} --- also called \textbf{outcomes}
--- that cannot be predicted with certainty. Well-known simple 
examples are found amongst games of chance such as tossing a coin, 
rolling dice, or playing roulette.

\item \textbf{Sample space} $\boldsymbol{\Omega} =\{\omega_{1},
\omega_{2},\ldots\}$: The sample space associated with a random 
experiment is constituted by the set of all possible
\textbf{elementary events} (or elementary outcomes) $\omega_{i}$
($i=1, 2, \ldots$), which are signified by their property of
\textit{mutual exclusivity}. The sample space~$\boldsymbol{\Omega}$
of a random experiment may contain either
%
\begin{itemize}
\item[(i)] a finite number $n$ of elementary events; then
$|\boldsymbol{\Omega}|=n$, or
\item[(ii)] countably many elementary events in the sense of a 
one-to-one correspondence with the set of natural numbers 
$\mathbb{N}$, or
\item[(iii)] uncountably may elements in the sense of a 
one-to-one correspondence with the set of real numbers 
$\mathbb{R}$, or an open or closed subset thereof.\footnote{For
reasons of definiteness, we will assume in this case
%throughout these  lecture notes
that the sample space $\boldsymbol{\Omega}$ 
associated with a random experiment is compact.}
\end{itemize}
%
The essential concept of the sample space associated with a random 
experiment was introduced to \textbf{Probability Theory} by the 
Italian mathematician 
\href{http://www-history.mcs.st-and.ac.uk/Biographies/Cardan.html}{Girolamo Cardano (1501--1576)}; see Cardano (1564)~\ct{car1564}, 
Mlodinow (2008)~\ct[p~42]{mlo2008}, and Bernstein 
(1998)~\ct[p~47ff]{ber1998}.

\item \textbf{Random events} $A, B, \ldots \subseteq 
\boldsymbol{\Omega}$: Random events are formally defined as all 
kinds of subsets of $\boldsymbol{\Omega}$ that can be formed from 
the elementary events $\omega_{i} \in \boldsymbol{\Omega}$.

\item \textbf{Certain event} $\boldsymbol{\Omega}$: The certain
event is synonymous with the sample space itself. When a particular 
random experiment is conducted, ``something will happen for sure.''

\item \textbf{Impossible event} $\emptyset=\{\}
= \bar{\boldsymbol{\Omega}}$: The impossible event is the natural 
complement to the certain event. When a particular random 
experiment is conducted, ``it is not possible that nothing will 
happen at all.''

\item \textbf{Event space} ${\cal P}(\boldsymbol{\Omega}) := \{A|A 
\subseteq \boldsymbol{\Omega}\}$: The event space, also referred 
to as the \textbf{power set} of~$\boldsymbol{\Omega}$, is the set
of all possible subsets (random events!) that can be formed from 
elementary events $\omega_{i} \in \boldsymbol{\Omega}$. Its size 
(or cardinality) is given by $|{\cal P}(\boldsymbol{\Omega})| = 
2^{|\boldsymbol{\Omega}|}$. The event space 
${\cal P}(\boldsymbol{\Omega})$ constitutes a so-called
\textbf{$\boldsymbol{\sigma}$--algebra} associated with the sample
space $\boldsymbol{\Omega}$; cf. Rinne (2008)~\ct[p~177]{rin2008}.
When $|\boldsymbol{\Omega}|=n$, i.e., when $\boldsymbol{\Omega}$ is 
finite, then $|{\cal P}(\boldsymbol{\Omega})|=2^{n}$.

\end{itemize}
%
In the formulation of probability theoretical laws and 
computational rules, the following set operations and identities 
prove useful.

%------------------------------------------------------------------
\subsection*{Set operations}
%------------------------------------------------------------------
%
\begin{enumerate}
\item $\bar{A} = \boldsymbol{\Omega}\backslash A$
--- \textbf{complementation} of a set (or event) $A$ (``not $A$'')

\item $A\backslash B=A\cap\bar{B}$ --- formation of the
\textbf{difference} of sets (or events) $A$ and $B$ (``$A$, but not
$B$'')

\item $A \cup B$ --- formation of the union of sets (or events) 
$A$ and $B$, otherwise referred to as the \textbf{disjunction} of
$A$ and $B$ (``$A$ or $B$'')

\item $A \cap B$ --- formation of the intersection of sets (or 
events) $A$ and $B$, otherwise referred to as the
\textbf{conjunction} of $A$ and $B$ (``$A$ and $B$'')

\item $A \subseteq B$ --- \textbf{inclusion} of a set (or event)
$A$ in a set (or event) $B$ (``$A$ is a subset of or equal
to~$B$'')

\end{enumerate}
%
%------------------------------------------------------------------
\subsection*{Computational rules and identities}
%------------------------------------------------------------------
%
\begin{enumerate}
\item $A \cup B=B\cup A$ and $A \cap B=B\cap A$
\hfill (\textbf{commutativity})

\item $(A \cup B)\cup C=A\cup (B\cup C)$ and \\
$(A \cap B)\cap C=A\cap (B\cap C)$
\hfill (\textbf{associativity})

\item $(A \cup B)\cap C=(A\cap C)\cup(B\cap C)$ and \\
$(A \cap B)\cup C=(A\cup C)\cap(B\cup C)$
\hfill (\textbf{distributivity})

\item $\overline{A\cup B}=\bar{A}\cap\bar{B}$ and
$\overline{A\cap B}=\bar{A}\cup\bar{B}$
\hfill (\textbf{de Morgan's laws})

\end{enumerate}
%

\medskip
\noindent
Before addressing the central axioms of \textbf{Probability
Theory}, we first provide the following important definition.

\medskip
\noindent
\underline{\textbf{Def.:}} Suppose given a \textit{compact} sample
space $\boldsymbol{\Omega}$ of some random experiment. Then one 
understands by a finite \textbf{complete partition} of 
$\boldsymbol{\Omega}$ a set of $n \in \mathbb{N}$ random events 
$\{A_{1}, \ldots, A_{n}\}$ such that
%
\begin{itemize}
\item[(i)] $A_{i} \cap A_{j} = \emptyset$ for $i \neq j$, i.e., 
they are \textbf{pairwise disjoint} (mutually exclusive), and 
\item[(ii)] $\displaystyle\bigcup_{i=1}^{n}A_{i} = 
\boldsymbol{\Omega}$, i.e., their union is identical to the full 
\textbf{sample space}.
\end{itemize}
%

%%%%%%%%%%%%%%%%%%%%%%%%%%%%%%%%%%%%%%%%%%%%%%%%%%%%%%%%%%%%%%%%%%%
\section[Kolmogorov's axioms of probability theory]{Kolmogorov's
axioms of probability theory}
\lb{sec:kolaxiom}
%%%%%%%%%%%%%%%%%%%%%%%%%%%%%%%%%%%%%%%%%%%%%%%%%%%%%%%%%%%%%%%%%%%
It took a fairly long time until, in 1933, a unanimously accepted 
basis of \textbf{Probability Theory} was established. In part the 
delay was due to problems with providing a unique definition 
of \textbf{probability}, and how it could be measured and
interpreted in practice. The situation was resolved only when the
Russian mathematician
\href{http://www-history.mcs.st-and.ac.uk/Biographies/Kolmogorov.html}{Andrey Nikolaevich Kolmogorov (1903--1987)} proposed to discard 
the intention of providing a unique definition of
\textbf{probability} altogether, and to restrict the issue instead
to merely prescribing in an axiomatic fashion a minimum set of 
properties any \textbf{probability measure} needs to possess
in order to be coherent and consistent. We now recapitulate the 
axioms that Kolmogorov put forward; cf. Kolmogoroff 
(1933)~\ct{kol1933}.
% in 1933.

\medskip
\noindent
For a given \textbf{random experiment}, let $\boldsymbol{\Omega}$
be its \textbf{sample space} and ${\cal P}(\boldsymbol{\Omega})$
the associated \textbf{event space}. Then a mapping
%
\be
P: {\cal P}(\boldsymbol{\Omega}) \rightarrow \mathbb{R}_{\geq 0}
\ee
%
defines a \textbf{probability measure} with the following
properties:
%
\begin{enumerate}
\item for all \textbf{random events} $A \in {\cal 
P}(\boldsymbol{\Omega})$, 
\hfill (\textbf{non-negativity})
%
\be
\lb{eq:axiom1}
P(A) \geq 0 \ ,
\ee
%

\item for the \textbf{certain event} $\boldsymbol{\Omega} \in 
{\cal P}(\boldsymbol{\Omega})$, \hfill (\textbf{normalisability})
%
\be
\lb{eq:axiom2}
P(\boldsymbol{\Omega}) = 1 \ ,
\ee
%

\item for all \textbf{pairwise disjoint random events} $A_{1},
A_{2}, \ldots \in {\cal P}(\boldsymbol{\Omega})$, i.e., $A_{i}
\cap A_{j} = \emptyset$ for all $i \neq j$,\\
\mbox{} \hfill ($\boldsymbol{\sigma}$\textbf{--additivity})
%
\be
\lb{eq:axiom3}
P\left(\bigcup_{i=1}^{\infty}A_{i}\right)
= P(A_{1} \cup A_{2} \cup \ldots)
= P(A_{1}) + P(A_{2}) + \ldots
= \sum_{i=1}^{\infty}P(A_{i}) \ .
\ee
%

\end{enumerate}
%
The first two axioms imply the property
%
\be
\lb{eq:pleqone}
0 \leq P(A) \leq 1 \ ,
\quad\text{for all}\quad
A \in {\cal P}(\boldsymbol{\Omega}) \ ;
\ee
%
the expression $P(A)$ itself is referred to as the
\textbf{probability} of a random event $A \in {\cal 
P}(\boldsymbol{\Omega})$. A less strict version of the third axiom
is given by requiring only \textbf{finite additivity} of a
probability measure. This means it shall possess the property
%
\be
P(A_{1} \cup A_{2}) = P(A_{1}) + P(A_{2}) \ ,
\quad\text{for any two }\quad
A_{1}, A_{2} \in {\cal P}(\boldsymbol{\Omega})
\quad\text{with}\quad A_{1} \cap A_{2} = \emptyset \ .
\ee
%

\medskip
\noindent
The triplet
%
\[
\left(\boldsymbol{\Omega}, {\cal P}, P\right)
\]
%
constitutes a special case of a so-called \textbf{probability
space}.

\medskip
\noindent
The following consequences for random events $A, B, A_{1}, A_{2}, 
\ldots \in {\cal P}(\boldsymbol{\Omega})$ can be derived from 
Kolmogorov's three axioms of probability theory; cf., e.g., 
Toutenburg (2005) \ct[p~19ff]{tou2005}. Their implications can be 
convienently visualised by means of \textbf{Venn diagrams}, 
named in honour of the English logician and philosopher 
\href{http://www-history.mcs.st-and.ac.uk/Biographies/Venn.html}{John
Venn FRS FSA (1834--1923)}; see Venn (1880)~\ct{ven1880}, and also,
e.g., Wewel (2014)~\ct[Ch.~5]{wew2014}.

%\pagebreak
\medskip
\noindent
\textbf{Consequences}
%
\begin{enumerate}

\item $P(\bar{A}) = 1-P(A)$

\item $P(\emptyset) = P(\bar{\boldsymbol{\Omega}}) = 0$

\item If $A \subseteq B$, then $P(A) \leq P(B)$.

\item $P(A_{1} \cup A_{2}) = P(A_{1})+P(A_{2})-P(A_{1} \cap 
A_{2})$.

\item $\displaystyle P(B) = \sum_{i=1}^{n}P(B \cap A_{i})$,
provided the $n \in \mathbb{N}$ random events $A_{i}$ constitute a 
finite \textbf{complete partition} of the sample space 
$\boldsymbol{\Omega}$.

\item $P(A\backslash B) = P(A) - P(A \cap B)$.

\end{enumerate}
%

\vspace{5mm}
\noindent
Employing its \textbf{complementation}~$\bar{A}$ and the first of
the consequences stated above, one defines by the ratio
%
\be
\lb{eq:odds}
O(A) := \frac{P(A)}{P(\bar{A})} = \frac{P(A)}{1-P(A)}
\ee
%
the so-called \textbf{odds} of a random event
$A \in {\cal P}(\boldsymbol{\Omega})$.

\medskip
\noindent
The renowned Israeli--US-American experimental psychologists 
Daniel Kahneman and Amos Tversky (the latter of which deceased in 
1996, aged fifty-nine) refer to the third of the consequences
stated above as the \textbf{extension rule}; see Tversky and
Kahneman (1983)~\ct[p~294]{tvekah1983}. It provides a cornerstone
to their remarkable investigations on the ``intuitive statistics''
applied by Humans in everyday \textbf{decision-making}, which focus
in particular on the \textbf{conjunction rule},
%
\be
\lb{eq:conjrule}
P(A \cap B) \leq P(A)
\quad\text{and}\quad
P(A \cap B) \leq P(B) \ ,
\ee
%
and the associated \textbf{disjunction rule},
%
\be
\lb{eq:disjrule}
P(A \cup B) \geq P(A)
\quad\text{and}\quad
P(A \cup B) \geq P(B) \ .
\ee
%
Both may be perceived as subcases of the fourth law above, which 
is occasionally referred to as the \textbf{convexity} property of a 
probability measure; cf. Gilboa (2009) \ct[p~160]{gil2009}. By 
means of their famous ``Linda the bank teller'' example in 
particular, Tversky and Kahneman (1983)~\ct[p~297ff]{tvekah1983} 
were able to demonstrate the startling empirical fact that the 
conjunction rule is frequently violated in everyday (intuitive)
decision-making; in their view, in consequence of decision-makers 
often resorting to a so-called \textit{representativeness
heuristic} as an aid in corresponding situations; see also Kahneman
(2011)~\ct[Sec.~15]{kah2011}. In recognition of their as much
intriguing as groundbreaking work, which sparked the discipline of
\textbf{Behavioural Economics}, Daniel Kahneman was awarded the  
\href{http://www.nobelprize.org/nobel_prizes/economics/laureates/2002/}{Sveriges
Riksbank Prize in Economic Sciences in Memory of Alfred Nobel}
in 2002.

%%%%%%%%%%%%%%%%%%%%%%%%%%%%%%%%%%%%%%%%%%%%%%%%%%%%%%%%%%%%%%%%%%%
\section[Laplacian random experiments]{Laplacian random
experiments}
\lb{sec:laplace}
%%%%%%%%%%%%%%%%%%%%%%%%%%%%%%%%%%%%%%%%%%%%%%%%%%%%%%%%%%%%%%%%%%%
Games of chance with a \textit{finite} number $n$ of possible 
mutually exclusive elementary outcomes, such as tossing a single 
coin once, rolling a single dye once, or selecting a single 
playing card from a deck of 32, belong to the simplest kinds of 
random experiments. In this context, there exists a clear-cut 
frequentist notion of a unique ``\textit{objective probability}'' 
associated with any kind of possible random event (outcome) that 
may occur.
Such probabilities can be computed according to a straightforward 
prescription due to the French mathematician and astronomer
\href{http://www-history.mcs.st-and.ac.uk/Biographies/Laplace.html}{Marquis Pierre Simon de Laplace (1749--1827)}. The prescription
rests on the assumption that the device generating the random 
events is a ``fair'' (i.e., unbiased) one.

\medskip
\noindent
Consider a random experiment, the $n$ \textbf{elementary events} 
$\omega_{i}$ ($i=1,\ldots,n$) of which that constitute the 
associated sample space $\boldsymbol{\Omega}$ are supposed to 
be ``equally likely,'' meaning they are assigned \textbf{equal 
probability}:
%
\be
P(\omega_{i}) = \frac{1}{|\boldsymbol{\Omega}|}
= \frac{1}{n} \ ,
\quad\text{for all}\quad \omega_{i}\in\boldsymbol{\Omega}
\ (i=1,\ldots,n) \ .
\ee
%
All random experiments of this nature are referred to as 
\textbf{Laplacian random experiments}.

\medskip
\noindent
\underline{\textbf{Def.:}} For a Laplacian random experiment,
the probability of an arbitrary random event $A \in 
{\cal P}(\boldsymbol{\Omega})$ can be computed according to the 
rule
%
\be
\lb{eq:classprob}
\fbox{$\displaystyle
P(A) := \frac{|A|}{|\boldsymbol{\Omega}|}
= \frac{\text{Number of cases favourable to event}\ A}{\text{Number
of all possible cases}} \ .
$}
\ee
%
Any probability measure $P$ which can be constructed in this 
fashion is called a \textbf{Laplacian probability measure}.

\medskip
\noindent
The systematic counting of the numbers of possible outcomes of 
random experiments in general is the central theme of
\textbf{combinatorics}. We now briefly address its main
considerations.

%%%%%%%%%%%%%%%%%%%%%%%%%%%%%%%%%%%%%%%%%%%%%%%%%%%%%%%%%%%%%%%%%%%
\section[Combinatorics]{Combinatorics}
\lb{sec:comb}
%%%%%%%%%%%%%%%%%%%%%%%%%%%%%%%%%%%%%%%%%%%%%%%%%%%%%%%%%%%%%%%%%%%
At the heart of combinatorical considerations is the well-known 
\textbf{urn model}. This supposes given an urn containing $N \in 
\mathbb{N}$ balls that are either
%
\begin{itemize}
\item[(a)] all different, and thus can 
be uniquely distinguished from one another, or
\item[(b)] there are $s \in 
\mathbb{N}$ ($s \leq N$) subsets of indistinguishable
like balls, of sizes $n_{1},\ldots,n_{s}$ resp., such that 
$n_{1}+\ldots+n_{s}=N$.
\end{itemize}
%
The first systematic developments in \textbf{Combinatorics} date
back to the Italian astronomer, physicist, engineer, philosopher,
and mathematician 
\href{http://www-history.mcs.st-and.ac.uk/Biographies/Galileo.html}{Galileo Galilei (1564--1642)} and the French mathematician
\href{http://www-history.mcs.st-and.ac.uk/Biographies/Pascal.html}{Blaise Pascal (1623--1662)}; cf. Mlodinow 
(2008)~\ct[p~62ff]{mlo2008}.

%------------------------------------------------------------------
\subsection{Permutations}
%------------------------------------------------------------------
\textbf{Permutations} relate to the number of distinguishable 
possibilities of arranging $N$ balls in an ordered sequences. 
Altogether, for cases (a) resp.\ (b) one finds that there are a 
total number of

%
\begin{center}
    \begin{tabular}[h!]{c|c}
    (a)~all balls different & (b)~$s$ subsets of like balls \\
    \hline
     & \\
    $N!$ & $\displaystyle \frac{N!}{n_{1}!n_{2}!\cdots n_{s}!}$
    \\
     & \\
    \end{tabular}
\end{center}
%
different possibilities. Remember that the \textbf{factorial} of a 
natural number $N \in \mathbb{N}$ is defined by
%
\be
N! := N\times(N-1)\times(N-2)\times\cdots\times 3 \times 2 
\times 1 \ .
\ee
%
\underline{\R:} \texttt{factorial($N$)}

%------------------------------------------------------------------
\subsection{Combinations and variations}
\lb{subsec:combvar}
%------------------------------------------------------------------
\textbf{Combinations} and \textbf{variations} ask for the total
number of distinguishable possibilities of selecting from a
collection of $N$ balls a sample of size $n \leq N$, while
differentiating between cases when
%
\begin{itemize}
\item[(a)] the order in which balls were selected is either 
neglected or instead accounted for, and

\item[(b)] a ball that was selected once either cannot be selected 
again or indeed can be selected again as often as a ball is being 
drawn.

\end{itemize}
%
These considerations result in the following cases of different 
possibilities:
%
\begin{center}
    \begin{tabular}[h!]{c|c|c}
     & no repetition & with repetition \\
    \hline
     & & \\
    combinations (order neglected) & $\left(\begin{array}{c}
    N \\ n \end{array}\right)$ & $\left(\begin{array}{c}
    N+n-1 \\ n \end{array}\right)$ \\
     & & \\
    \hline
     & & \\
    variations (order accounted for) & $\left(\begin{array}{c}
    N \\ n \end{array}\right)n!$ & $N^{n}$ \\
     & & \\
    \end{tabular}
\end{center}
%
Note that, herein, the \textbf{binomial coefficient} for two
natural numbers $n, N \in \mathbb{N}$, $n \leq N$, introduced by
\href{http://www-history.mcs.st-and.ac.uk/Biographies/Pascal.html}{Blaise Pascal (1623--1662)}, is defined by
%
\be
\lb{eq:binomcoeff}
\left(\begin{array}{c}
N \\ n \end{array}\right) := \frac{N!}{n!(N-n)!} \ .
\ee
%
For fixed value of $N$ and running value of $n \leq N$, it 
generates the positive integer entries of Pascal's well-known 
numerical triangle; see, e.g., Mlodinow 
(2008)~\ct[p~72ff]{mlo2008}. The binomial coefficient satisfies 
the identity
%
\be
\left(\begin{array}{c}
N \\ n \end{array}\right) \equiv \left(\begin{array}{c}
N \\ N-n \end{array}\right) \ .
\ee
%
\underline{\R:} \texttt{choose($N$, $n$)}

\medskip
\noindent
To conclude this chapter, we turn to discuss the essential concept 
of \textbf{conditional probabilities} of random events.

%%%%%%%%%%%%%%%%%%%%%%%%%%%%%%%%%%%%%%%%%%%%%%%%%%%%%%%%%%%%%%%%%%%
\section[Conditional probabilities]{Conditional probabilities}
\lb{sec:bedwahr}
%%%%%%%%%%%%%%%%%%%%%%%%%%%%%%%%%%%%%%%%%%%%%%%%%%%%%%%%%%%%%%%%%%%
Consider some random experiment with sample space 
$\boldsymbol{\Omega}$, event space ${\cal 
P}(\boldsymbol{\Omega})$, and a well-defined, unique probability 
measure $P$ over ${\cal P}(\boldsymbol{\Omega})$.

\medskip
\noindent
\underline{\textbf{Def.:}} For random events $A, B \in
{\cal P}(\boldsymbol{\Omega})$, with $P(B) > 0$,
%
\be
\lb{condprob}
\fbox{$\displaystyle
P(A|B) := \frac{P(A \cap B)}{P(B)}
$}
\ee
%
defines the \textbf{conditional probability} of $A$
to occur, given that it is known that $B$ occurred before.
Analogously, one defines 
a conditional probability  $P(B|A)$ with the roles of random 
events $A$ and $B$ switched, provided $P(A) > 0$. Note that since, 
by Eq.~(\ref{eq:pleqone}), $0 \leq P(A|B), P(B|A) \leq 1$, the 
implication of definition~(\ref{condprob}) is that the conjunction 
rule~(\ref{eq:conjrule}) must \textit{always} be satisfied.

\medskip
\noindent
\underline{\textbf{Def.:}} Random events $A, B \in 
{\cal P}(\boldsymbol{\Omega})$ are called \textbf{mutually 
stochastically independent}, if, simultaneously, the conditions
%
\be
\lb{eq:stochindep1}
\fbox{$\displaystyle
P(A|B) \stackrel{!}{=} P(A) \ ,\quad P(B|A) \stackrel{!}{=} P(B)
\quad\stackrel{\text{Eq.~\ref{condprob}}}{\Leftrightarrow}\quad
P(A \cap B) = P(A)P(B)
$}
\ee
%
are satisfied, i.e., when for both random events $A$ and $B$ the 
{\bf\textit{a posteriori} probabilities} $P(A|B)$ and $P(B|A)$ 
coincide with the respective {\bf\textit{a priori} probabilities} 
$P(A)$ and $P(B)$.

\medskip
\noindent
For applications, the following two prominent laws of
\textbf{Probability Theory} prove essential.

%------------------------------------------------------------------
\subsection[Law of total probability]{Law of total probability}
%------------------------------------------------------------------
For a random experiment with probability space 
$\left(\boldsymbol{\Omega}, {\cal P}, P\right)$, it holds by the 
\textbf{law of total probability} that for any random event $B \in 
{\cal P}(\boldsymbol{\Omega})$
%
\be
\lb{eq:totalprob}
\fbox{$\displaystyle
P(B)=\sum_{i=1}^{m}P(B|A_{i})P(A_{i}) \ ,
$}
\ee
%
provided the random events $A_{1}, \ldots, A_{m} \in 
{\cal P}(\boldsymbol{\Omega})$ constitute a finite \textbf{complete 
partition} of $\boldsymbol{\Omega}$ into $m \in \mathbb{N}$
\textbf{pairwise disjoint events}.

\medskip
\noindent
The content of this law may be conveniently visualised by means of 
a Venn diagram.

%------------------------------------------------------------------
\subsection[Bayes' theorem]{Bayes' theorem}
\lb{subsec:bayes}
%------------------------------------------------------------------
This important result is due to the English mathematician and
Presbyterian minister 
\href{http://www-history.mcs.st-and.ac.uk/Biographies/Bayes.html}{Thomas Bayes (1702--1761)}; see the posthumous publication Bayes 
(1763)~\ct{bay1763}. For a random experiment with probability space
$\left(\boldsymbol{\Omega}, {\cal P}, P\right)$, it states that,
given

%
\begin{itemize}

\item[(i)] random events $A_{1}, \ldots, A_{m} \in 
{\cal P}(\boldsymbol{\Omega})$ which constitute a finite
\textbf{complete partition} of $\boldsymbol{\Omega}$ into $m \in
\mathbb{N}$ \textbf{pairwise disjoint events},

\item[(ii)] $P(A_{i}) > 0$
for all $i = 1, \ldots, m$, with $\displaystyle
\sum_{i=1}^{m}P(A_{i}) = 1$ by Eq.~(\ref{eq:axiom2}), and

\item[(iii)] a random event $B \in {\cal P}(\boldsymbol{\Omega})$
with $\displaystyle 
P(B)\stackrel{\text{Eq.~\ref{eq:totalprob}}}{=}\sum_{i=1}^{m}
P(B|A_{i})P(A_{i}) > 0$ that is known to have occurred,

\end{itemize}
%
the identity
%
\be
\lb{eq:bayes}
\fbox{$\displaystyle
P(A_{i}|B)=\frac{P(B|A_{i})P(A_{i})}{
{\displaystyle\sum_{j=1}^{m}P(B|A_{j})P(A_{j})}}
$}
\ee
%
applies. This form of the theorem was given by Laplace
(1774)~\ct{lap1774}. By Eq.~(\ref{eq:axiom2}), it necessarily
follows that $\displaystyle \sum_{i=1}^{m}P(A_{i}|B) = 1$. Again,
the content of \textbf{Bayes' theorem} may be conveniently
visualised by means of a Venn diagram.

\medskip
\noindent
Some of the different terms appearing in Eq.~(\ref{eq:bayes}) have 
been given names in their own right:
%
\begin{itemize}

\item $P(A_{i})$ is referred to as the \textbf{prior probability}
of random event, or hypothesis, $A_{i}$,

\item $P(B|A_{i})$ is the \textbf{likelihood} of random event,
or empirical evidence, $B$, given random event, or hypothesis,
$A_{i}$, and

\item $P(A_{i}|B)$ is called the \textbf{posterior probability} of 
random event, or hypothesis, $A_{i}$, given random event,
or empirical evidence, $B$.

\end{itemize}
%

\medskip
\noindent
The most common interpretation of \textbf{Bayes' theorem} is 
that it essentially provides a means for computing the
\textbf{posterior probability} of a random event, or hypothesis,
$A_{i}$, given information on the factual realisation of an
associated random event, or evidence, $B$, in terms of the product
of the \textbf{likelihood} of $B$, given $A_{i}$, and the
\textbf{prior probability} of~$A_{i}$,
%
\be
\lb{eq:bayes2}
P(A_{i}|B) \propto P(B|A_{i}) \times P(A_{i}) \ .
\ee
%
This result
is at the heart of the interpretation that \textbf{empirical 
learning} amounts to updating the prior ``\textit{subjective 
probability}'' one has assigned to a specific random 
event, or hypothesis, $A_{i}$, in order to quantify one's initial
reasonable \textbf{degree-of-belief} in its occurrence resp. in its
truth content, by means of adequate experimental or observational
data and corresponding theoretical considerations; see, e.g., Sivia
and Skilling (2006)~\ct[p~5ff]{sivski2006}, Gelman \textit{et al}
(2014)~\ct[p~6ff]{geletal2014}, or McElreath
(2016)~\ct[p~4ff]{mce2016}.
%Caveats concerning the
%\textbf{Bayes--Laplace approach} to evaluating probabilities of
%random events are its often rather high computational costs, and
%the occasional unavailability of useful priors or, more seriously,
%adequate likelihoods.

\medskip
\noindent
The \textbf{Bayes--Laplace approach} to tackling 
quantitative--statistical problems in \textbf{Econometrics} was 
pioneered by Zellner in the early 1970ies; see the 1996 reprint of 
his renowned 1971 monograph~\ct{zel1996}. A recent thorough 
introduction into its main considerations is provided by the 
graduate textbook by Greenberg (2013)~\ct{gre2013}.

\medskip
\noindent
A particularly prominent application of this framework in
\textbf{Econometrics} is given by proposals to the mathematical
modelling of economic agents' \textbf{decision-making} (in the
sense of choice behaviour) under conditions of
\textbf{uncertainty}, which, fundamentally, assume \textit{rational
behaviour} on the part of the agents; see, e.g., the graduate
textbook by Gilboa (2009)~\ct{gil2009}, and the brief reviews by
Svetlova and van Elst (2012, 2014)~\ct{svehve2012,svehve2014}, as
well as references therein. \textit{Psychological dimensions} of 
\textbf{decision-making}, on the other hand, such as the
empirically established existence of reference points, loss
aversion, and distortion of probabilities into corresponding
decision weights, have been accounted for in Kahneman and Tversky's 
(1979)~\ct{kahtve1979} \textbf{Prospect Theory}.

%%%%%%%%%%%%%%%%%%%%%%%%%%%%%%%%%%%%%%%%%%%%%%%%%%%%%%%%%%%%%%%%%%%
%%%%%%%%%%%%%%%%%%%%%%%%%%%%%%%%%%%%%%%%%%%%%%%%%%%%%%%%%%%%%%%%%%%
