%%%%%%%%%%%%%%%%%%%%%%%%%%%%%%%%%%%%%%%%%%%%%%%%%%%%%%%%%%%%%%%%%%%
%  File name: ch7-Stats.tex
%  Title:
%  Version: 29.05.2019 (hve)
%%%%%%%%%%%%%%%%%%%%%%%%%%%%%%%%%%%%%%%%%%%%%%%%%%%%%%%%%%%%%%%%%%%
%%%%%%%%%%%%%%%%%%%%%%%%%%%%%%%%%%%%%%%%%%%%%%%%%%%%%%%%%%%%%%%%%%%
\chapter[Discrete and continuous random variables]{\href{https://www.youtube.com/watch?v=1rEczO3Jh_Y}{Discrete and continuous random
variables}}
\lb{ch7}
%%%%%%%%%%%%%%%%%%%%%%%%%%%%%%%%%%%%%%%%%%%%%%%%%%%%%%%%%%%%%%%%%%%
Applications of \textbf{inferential statistical methods}
rooted in the \textbf{frequentist approach} to \textbf{Probability
Theory}, some of which are to be discussed in Chs.~\ref{ch12} and
\ref{ch13} below, rest fundamentally on the concept of a
probability-dependent quantity arising in the context of
\textbf{random experiments} that is referred to as a \textbf{random
variable}. The present chapter aims to provide a basic introduction
to the general properties and characteristic features of random
variables. We begin by stating the definition of this concept.

\medskip
\noindent
\underline{\textbf{Def.:}} A real-valued one-dimensional
\textbf{random variable} is defined as a one-to-one mapping
%
\be
X: \boldsymbol{\Omega} \rightarrow D \subseteq \mathbb{R}
\ee
%
of the sample space $\boldsymbol{\Omega}$ of some random 
experiment with associated probability space 
$\left(\boldsymbol{\Omega}, {\cal P}, P\right)$ into a subset $D$ 
of the real numbers $\mathbb{R}$.

\medskip
\noindent
Depending on the nature of the \textbf{spectrum of values} of $X$,
we will distinguish in the following between random 
variables of the \textbf{discrete} and \textbf{continuous} kinds.

%%%%%%%%%%%%%%%%%%%%%%%%%%%%%%%%%%%%%%%%%%%%%%%%%%%%%%%%%%%%%%%%%%%
\section[Discrete random variables]{Discrete random variables}
\lb{sec:diskretz}
%%%%%%%%%%%%%%%%%%%%%%%%%%%%%%%%%%%%%%%%%%%%%%%%%%%%%%%%%%%%%%%%%%%
\textbf{Discrete random variables} are signified by the existence
of a finite or countably infinite

\medskip
\noindent
\textbf{Spectrum of values:}
%
\be
X \mapsto x \in \left\{x_{1}, \ldots, x_{n}\right\}
\subset \mathbb{R} \ ,
\quad\quad\text{with}\quad n \in \mathbb{N} \ .
\ee
%
All values $x_{i}$ ($i=1,\ldots,n$) in this spectrum,
referred to as possible \textbf{realisations} of $X$, are assigned 
individual probabilities $p_{i}$ by a real-valued

\medskip
\noindent
\textbf{Probability function:}
%
\be
\fbox{$\displaystyle
P(X=x_{i}) = p_{i} \quad\quad\text{for}\quad i=1, \dots, n \ ,
$}
\ee
%
with properties\\[-5mm]
%
\begin{center}
\begin{itemize}
\item[(i)] $0 \leq p_{i} \leq 1$, and
\hfill (\textbf{non-negativity})\\[-5mm]

\item[(ii)] $\displaystyle\sum_{i=1}^{n}p_{i}=1$.
\hfill (\textbf{normalisability})\\[-5mm]
\end{itemize}
\end{center}
%
Specific distributional features of a discrete random variable $X$ 
deriving from its probability function $P(X=x_{i})$ are encoded in 
the associated theoretical

\medskip
\noindent
\textbf{Cumulative distribution function (\texttt{cdf}):}
%
\be
\fbox{$\displaystyle
F_{X}(x) = \texttt{cdf}(x) := P(X \leq x)
= \sum_{i|x_{i}\leq x}P(X=x_{i}) \ .
$}
\ee
%
The \texttt{cdf} exhibits the asymptotic behaviour
%
\be
\lim_{x\to-\infty}F_{X}(x)=0 \ ,
\qquad
\lim_{x\to+\infty}F_{X}(x)=1 \ .
\ee
%
Information on the central tendency and the variability of a 
discrete random variable~$X$ is quantified in terms of its


\medskip
\noindent
\textbf{Expectation value} and \textbf{variance:}
%
\bea
\lb{eq:evdiscr}
\mathrm{E}(X) & := & \sum_{i=1}^{n}x_{i}P(X=x_{i}) \\
%
\lb{eq:vardiscr}
\mathrm{Var}(X) & := & \sum_{i=1}^{n}\left(x_{i}-\mathrm{
E}(X)\right)^{2}P(X=x_{i}) \ .
\eea
%
One of the first occurrences of the notion of the expectation 
value of a random variable relates to the famous ``wager'' put 
forward by the French mathematician
\href{http://www-history.mcs.st-and.ac.uk/Biographies/Pascal.html}{Blaise Pascal (1623--1662)}; cf. Gilboa 
(2009)~\ct[Sec.~5.2]{gil2009}.

\medskip
\noindent
By the so-called \textbf{shift theorem} it holds that the variance 
may alternatively be obtained from the computationally more 
efficient formula 
%
\be
\mathrm{Var}(X)=\mathrm{E}\left[(X-\mathrm{E}(X))^{2}\right]
=\mathrm{E}(X^{2})-\left[\mathrm{E}(X)\right]^{2} \ .
\ee
%
Specific values of $\mathrm{E}(X)$ and $\mathrm{ Var}(X)$ will be 
denoted throughout by the Greek letters $\mu$ and $\sigma^{2}$, 
respectively. The \textbf{standard deviation} of $X$ amounts to 
$\sqrt{\mathrm{Var}(X)}$; its specific values will be denoted by 
$\sigma$.

\medskip
\noindent
The evaluation of \textbf{event probabilities} for a discrete
random variable~$X$ with known probability function~$P(X=x_{i})$
follows from the

\medskip
\noindent
\textbf{Computational rules:}
%
\bea
\lb{eq:comprulesdiscr1}
P(X \leq d) & = & F_{X}(d) \\
%
P(X < d) & = & F_{X}(d) - P(X=d) \\
%
P(X \geq c) & = & 1 - F_{X}(c) + P(X=c) \\
%
P(X > c) & = &  1- F_{X}(c) \\
%
P(c \leq X \leq d) & = & F_{X}(d) - F_{X}(c) + P(X=c) \\
%
P(c < X \leq d) & = & F_{X}(d) - F_{X}(c) \\
%
P(c \leq X < d) & = & F_{X}(d) - F_{X}(c) - P(X=d) + P(X=c) \\
%
\lb{eq:comprulesdiscr8}
P(c < X < d) & = & F_{X}(d) - F_{X}(c) - P(X=d) \ ,
\eea
%
where $c$ and $d$ denote arbitrary lower and upper cut-off values 
imposed on the spectrum of $X$.

\medskip
\noindent
In applications it is frequently of interest to know the values of 
a discrete \texttt{cdf}'s

\medskip
\noindent
$\boldsymbol{\alpha}$\textbf{--quantiles:}\\
These are realisations $x_{\alpha}$ of $X$ specifically
determined by the condition that $X$ take values $x \leq 
x_{\alpha}$ at least with probability $\alpha$ (for $0<\alpha<1$),
i.e.,
%
\be
F_{X}(x_{\alpha}) = P(X \leq x_{\alpha}) \stackrel{!}{\geq} \alpha
\qquad\text{and}\qquad
F_{X}(x) = P(X \leq x) < \alpha
\quad\quad\text{for}\quad
x < x_{\alpha} \ .
\ee
%
Occasionally, $\alpha$--quantiles of a probability distribution 
are also referred to as \textbf{percentile values}.

%%%%%%%%%%%%%%%%%%%%%%%%%%%%%%%%%%%%%%%%%%%%%%%%%%%%%%%%%%%%%%%%%%%
\section[Continuous random variables]{Continuous random variables}
\lb{sec:stetigz}
%%%%%%%%%%%%%%%%%%%%%%%%%%%%%%%%%%%%%%%%%%%%%%%%%%%%%%%%%%%%%%%%%%%
\textbf{Continuous random variables} possess an uncountably
infinite

\medskip
\noindent
\textbf{Spectrum of values:}
%
\be
X \mapsto x \in D \subseteq \mathbb{R} \ .
\ee
%
It is, therefore, no longer meaningful to assign probabilities to 
individual \textbf{realisations} $x$ of $X$, but only to 
infinitesimally small intervals $\mathrm{d}x \in D$ instead, by
means of a real-valued

\medskip
\noindent
\textbf{Probability density function (\texttt{pdf}):}
%
\be
\fbox{$\displaystyle
f_{X}(x) = \texttt{pdf}(x) \ .
$}
\ee
%
Hence, approximately,
%
\[
P(X \in \mathrm{d}x) \approx f_{X}(\xi)\,\mathrm{d}x \ ,
\]
%
for some representative $\xi \in \mathrm{d}x$. The \texttt{pdf} of
an arbitrary continuous random variable~$X$ has the defining 
properties:\\[-5mm]
%
\begin{center}
\begin{itemize}
\item[(i)] $f_{X}(x) \geq 0$ for all $x \in D$,
\hfill (\textbf{non-negativity})\\[-5mm]

\item[(ii)] ${\displaystyle \int_{-\infty}^{+\infty}f_{X}(x)\,
\mathrm{d}x = 1}$, and \hfill (\textbf{normalisability})\\[-5mm]

\item[(iii)] $f_{X}(x) = F_{X}^{\prime}(x)$. \hfill (\textbf{link
to} \texttt{cdf})\\[-5mm]

\end{itemize}
\end{center}
%
The evaluation of \textbf{event probabilities} for a continuous 
random variable~$X$ rests on the associated theoretical

\medskip
\noindent
\textbf{Cumulative distribution function (\texttt{cdf}):}
%
\be
\fbox{$\displaystyle
F_{X}(x) = \texttt{cdf}(x) := P(X \leq x)
= \int_{-\infty}^{x}f_{X}(t)\,\mathrm{d}t \ .
$}
\ee
%

\pagebreak
\noindent
Event probabilities for~$X$ are then to be obtained from the

\medskip
\noindent
\textbf{Computational rules:}
%
\bea
\lb{eq:comprulescont1}
P(X = d) & = & 0 \\
%
\lb{eq:comprulescont2}
P(X \leq d) & = & F_{X}(d) \\
%
\lb{eq:comprulescont3}
P(X \geq c) & = & 1 - F_{X}(c) \\
%
\lb{eq:comprulescont4}
P(c \leq X \leq d) & = & F_{X}(d) - F_{X}(c) \ ,
\eea
%
where $c$ and $d$ denote arbitrary lower and upper cut-off values 
imposed on the spectrum of $X$. Note that, again, the \texttt{cdf}
exhibits the asymptotic properties
%
\be
\lim_{x\to-\infty}F_{X}(x)=0 \ ,
\qquad
\lim_{x\to+\infty}F_{X}(x)=1 \ .
\ee
%
The central tendency and the variabilty of a continuous random 
variable~$X$ are quantified by its

\medskip
\noindent
\textbf{Expectation value} and \textbf{variance:}
%
\bea
\lb{eq:expectcon}
\mathrm{E}(X) & := & \int_{-\infty}^{+\infty}xf_{X}(x)\,
\mathrm{d}x \\
%
\lb{eq:varcon}
\mathrm{Var}(X) & := & \int_{-\infty}^{+\infty}
\left(x-\mathrm{E}(X)\right)^{2}f_{X}(x)\,\mathrm{d}x \ .
\eea
%
Again, by the \textbf{shift theorem} the variance may alternatively 
be obtained from the computationally more efficient formula 
$\displaystyle \mathrm{Var}(X)=\mathrm{E}
\left[(X-\mathrm{E}(X))^{2}\right]
=\mathrm{E}(X^{2})-\left[\mathrm{E}(X)\right]^{2}$. Specific values
of $\mathrm{E}(X)$ and $\mathrm{Var}(X)$ will be denoted throughout
by $\mu$ and $\sigma^{2}$, respectively. The \textbf{standard
deviation} of $X$ amounts to $\sqrt{\mathrm{Var}(X)}$; its specific
values will be denoted by $\sigma$.

\medskip
\noindent
The construction of interval estimates for unknown distribution 
parameters of continuous one-dimensional random variables~$X$ in 
given target populations~$\boldsymbol{\Omega}$, and null hypothesis
significance testing (to be discussed later in 
Chs.~\ref{ch12} and \ref{ch13}), both require explicit knowledge of
the $\boldsymbol{\alpha}$\textbf{--quantiles} associated with the
\texttt{cdf}s of the~$X$s. Generally, these are defined as follows.

\medskip
\noindent
$\boldsymbol{\alpha}$\textbf{--quantiles:} \\
$X$ take values $x \leq x_{\alpha}$ with probability $\alpha$ 
(for $0<\alpha<1$), i.e.,
%
\be
P(X \leq x_{\alpha}) = F_{X}(x_{\alpha}) \stackrel{!}{=} 
\alpha
\qquad \overbrace{\Leftrightarrow}^{F_{X}(x)\ \text{is strictly
monotonously increasing}}\qquad
\fbox{$\displaystyle
x_{\alpha}=F_{X}^{-1}(\alpha)$} \ .
\ee
%
Hence, $\alpha$--quantiles of the probability distribution for a 
continuous one-dimensional random variable~$X$ are determined by 
the inverse \texttt{cdf}, $F_{X}^{-1}$. For given $\alpha$, the 
spectrum of $X$ is thus naturally partitioned into domains $x \leq 
x_{\alpha}$ and $x \geq x_{\alpha}$. Occasionally, 
$\alpha$--quantiles of a probability distribution are also 
referred to as \textbf{percentile values}.

%%%%%%%%%%%%%%%%%%%%%%%%%%%%%%%%%%%%%%%%%%%%%%%%%%%%%%%%%%%%%%%%%%%
\section[Skewness and excess kurtosis]{Skewness and excess 
kurtosis}
\lb{sec:skewkurt}
%%%%%%%%%%%%%%%%%%%%%%%%%%%%%%%%%%%%%%%%%%%%%%%%%%%%%%%%%%%%%%%%%%%
In analogy to the descriptive case of Sec.~\ref{sec:distortion},
dimensionless \textbf{measures of relative distortion}
characterising the \textbf{shape} of the probability distribution
for a discrete or a continuous one-dimensional random variable~$X$
are defined by the

\medskip
\noindent
\textbf{Skewness} and \textbf{excess kurtosis:}
%
\bea
\lb{eq:skew2}
\mathrm{Skew}(X) & := & \frac{\mathrm{E}\left[(X-\mathrm{
E}(X))^{3}\right]}{\left[\mathrm{Var}(X)\right]^{3/2}} \\
%
\lb{eq:kurt2}
\mathrm{Kurt}(X) & := & \frac{\mathrm{E}\left[(X-\mathrm{
E}(X))^{4}\right]}{\left[\mathrm{Var}(X)\right]^{2}} - 3 \ ,
\eea
%
given $\mathrm{Var}(X) > 0$; cf. Rinne (2008)~\ct[p~196]{rin2008}. 
Specific values of $\mathrm{Skew}(X)$ and $\mathrm{Kurt}(X)$ may be 
denoted by $\gamma_{1}$ and $\gamma_{2}$, respectively.

%%%%%%%%%%%%%%%%%%%%%%%%%%%%%%%%%%%%%%%%%%%%%%%%%%%%%%%%%%%%%%%%%%%
\section[Lorenz curve for continuous random variables]{Lorenz 
curve for continuous random variables}
%%%%%%%%%%%%%%%%%%%%%%%%%%%%%%%%%%%%%%%%%%%%%%%%%%%%%%%%%%%%%%%%%%%
For a continuous one-dimensional random variable~$X$, the
\textbf{Lorenz curve} expressing qualitatively the degree of
concentration involved in its associated probability distribution
of is defined by
%
\be
\lb{lorcurve}
\fbox{$\displaystyle
L(x_{\alpha})
= 
\frac{{\displaystyle\int_{-\infty}^{x_{\alpha}}tf_{X}(t)\,
\mathrm{d}t}}{{\displaystyle\int_{-\infty}^{+\infty}tf_{X}(t)\,
\mathrm{d}t}} \ ,
$}
\ee
%
with $x_{\alpha}$ denoting a particular $\alpha$--quantile
of the distribution in question.

%%%%%%%%%%%%%%%%%%%%%%%%%%%%%%%%%%%%%%%%%%%%%%%%%%%%%%%%%%%%%%%%%%%
\section[Linear transformations of random variables]{Linear
transformations of random variables}
\lb{sec:lintransf}
%%%%%%%%%%%%%%%%%%%%%%%%%%%%%%%%%%%%%%%%%%%%%%%%%%%%%%%%%%%%%%%%%%%
\textbf{Linear transformations} of real-valued one-dimensional
random variables $X$ are determined by the two-parameter relation
%
\be
\lb{lintransf}
\fbox{$\displaystyle
Y = a+bX \quad\text{with}\quad a,b \in \mathbb{R}, b \neq 0 \ ,
$}
\ee
%
where $Y$ denotes the resultant new random variable. 
Transformations of random variables of this kind have 
the following effects on the computation of expectation values and 
variances.

%
%------------------------------------------------------------------
\subsection[Effect on expectation values]{Effect on expectation 
values}
%------------------------------------------------------------------
%
\begin{enumerate}
\item $\mathrm{E}(a) = a$
\item $\mathrm{E}(bX) = b\mathrm{E}(X)$
\item $\mathrm{E}(Y) = \mathrm{E}(a+bX) = \mathrm{E}(a)
+ \mathrm{E}(bX)
= a+b\mathrm{E}(X)$.
\end{enumerate}
%

%------------------------------------------------------------------
\subsection[Effect on variances]{Effect on variances}
\lb{subsec:vartrans}
%------------------------------------------------------------------
%
\begin{enumerate}
\item $\mathrm{Var}(a) = 0$
\item $\mathrm{Var}(bX) = b^{2}\mathrm{Var}(X)$
\item $\mathrm{Var}(Y) = \mathrm{Var}(a+bX) = \mathrm{Var}(a)
+ \mathrm{Var}(bX) = b^{2}\mathrm{Var}(X)$.
\end{enumerate}
%

%------------------------------------------------------------------
\subsection[Standardisation]{Standardisation}
\lb{subsec:standard}
%------------------------------------------------------------------
\textbf{Standardisation} of an arbitrary one-dimensional random 
variable $X$, with $\sqrt{\mathrm{Var}(X)} > 0$, implies the 
determination of a special linear transformation $X \mapsto Z$ 
according to Eq.~(\ref{lintransf}) such that the expectation value 
and variance of $X$ are re-scaled to their simplest values 
possible, i.e., $\mathrm{E}(Z)=0$ and $\mathrm{Var}(Z)=1$. Hence,
the two (in part non-linear) conditions
%
\[
0 \stackrel{!}{=} \mathrm{E}(Z)= a+b\mathrm{E}(X)
\quad\text{and}\quad
1 \stackrel{!}{=} \mathrm{Var}(Z)= b^{2}\mathrm{Var}(X) \ ,
\]
%
for unknowns $a$ and $b$, need to be satisfied simultaneously. 
These are solved by, respectively,
%
\be
a=-\frac{\mathrm{E}(X)}{\sqrt{\mathrm{Var}(X)}}
\quad\text{and}\quad
b=\frac{1}{\sqrt{\mathrm{Var}(X)}} \ ,
\ee
%
and so
%
\be
\lb{eq:standardisation}
\fbox{$\displaystyle
X \rightarrow Z = \frac{X-\mathrm{E}(X)}{\sqrt{\mathrm{Var}(X)}}
\ , \qquad
x \mapsto z = \frac{x-\mu}{\sigma} \in \bar{\mathbb{D}}
\subseteq \mathbb{R} \ ,
$}
\ee
%
irrespective of whether the random variable $X$ is of the discrete 
kind (cf. Sec.~\ref{sec:diskretz}) or of the continuous kind (cf. 
Sec.~\ref{sec:stetigz}). It is essential for applications to 
realise that under the process of standardisation 
the values of event probabilities for a random variable $X$ 
remain \textbf{invariant} (unchanged), i.e.,
%
\be
P(X\leq x)
= P\left(\frac{X-\mathrm{E}(X)}{\sqrt{\mathrm{Var}(X)}}
\leq \frac{x-\mu}{\sigma}\right) = P(Z\leq z) \ .
\ee
%

%%%%%%%%%%%%%%%%%%%%%%%%%%%%%%%%%%%%%%%%%%%%%%%%%%%%%%%%%%%%%%%%%%%
\section[Sums of random variables and reproductivity]{Sums of
random variables and reproductivity}
\lb{sec:sumvar}
%%%%%%%%%%%%%%%%%%%%%%%%%%%%%%%%%%%%%%%%%%%%%%%%%%%%%%%%%%%%%%%%%%%
\underline{\textbf{Def.:}} For a set of $n$ additive
one-dimensional random variables~$X_{1}, \ldots, X_{n}$, one
defines a \textbf{total sum} random variable~$Y_{n}$ and an
associated \textbf{mean} random variable~$\bar{X}_{n}$ according to
%
\be
\lb{eq:sumnmean}
\fbox{$\displaystyle
Y_{n} := \sum_{i=1}^{n}X_{i}
\quad\quad\text{and}\quad\quad
\bar{X}_{n} := \frac{1}{n}\,Y_{n} \ .
$}
\ee
%

\medskip
\noindent
By \textit{linearity} of the expectation value 
operation,\footnote{That is: $\mathrm{E}(X_{1}+X_{2})
= \mathrm{E}(X_{1})+\mathrm{E}(X_{2})$.} it then holds that
%
\be
\lb{eq:sumexpv}
\mathrm{E}(Y_{n})
= \mathrm{E}\left(\sum_{i=1}^{n}X_{i}\right)
= \sum_{i=1}^{n}\mathrm{E}(X_{i})
\quad\quad\text{and}\quad\quad
\mathrm{E}(\bar{X}_{n}) = \frac{1}{n}\,\mathrm{E}(Y_{n}) \ .
\ee
%
If, in addition, the $X_{1}, \ldots, X_{n}$ are \textit{mutually 
stochastically independent} according to 
Eq.~(\ref{eq:stochindep1}) (see also 
Sec.~\ref{subsec:2dcovtheoret} below), it follows from 
Sec.~\ref{subsec:vartrans} that the variances of $Y_{n}$ and 
$\bar{X}_{n}$ are given by
%
\be
\lb{eq:sumvar}
\mathrm{Var}(Y_{n})
= \mathrm{Var}\left(\sum_{i=1}^{n}X_{i}\right)
= \sum_{i=1}^{n}\mathrm{Var}(X_{i})
\quad\quad\text{and}\quad\quad
\mathrm{Var}(\bar{X}_{n})
= \left(\frac{1}{n}\right)^{2}\mathrm{Var}(Y_{n}) \ ,
\ee
%
respectively.

\medskip
\noindent
\underline{\textbf{Def.:}} \textbf{Reproductivity} of a probability 
distribution law (\texttt{cdf}) $F(x)$ is given when the total 
sum~$Y_{n}$ of $n$ independent and identically distributed (in 
short: ``i.i.d.'') additive one-dimensional random 
variables~$X_{1}, \ldots, X_{n}$, which each individually satisfy 
distribution laws $F_{X_{i}}(x) \equiv F(x)$, inherits \textit{this 
very} distribution law $F(x)$ from its underlying $n$ random 
variables. Examples of reproductive distribution laws, to be 
discussed in the following Ch.~\ref{ch8}, are the binomial, the 
Gau\ss ian normal, and the $\chi^{2}$--distributions.

%%%%%%%%%%%%%%%%%%%%%%%%%%%%%%%%%%%%%%%%%%%%%%%%%%%%%%%%%%%%%%%%%%%
\section[Two-dimensional random variables]{Two-dimensional random 
variables}
\lb{sec:2dvar}
%%%%%%%%%%%%%%%%%%%%%%%%%%%%%%%%%%%%%%%%%%%%%%%%%%%%%%%%%%%%%%%%%%%
The \textbf{empirical tests for association} between two 
statistical variables $X$ and $Y$ of Ch.~\ref{ch13} require the 
notions of \textbf{two-dimensional random variables} and their 
bivariate \textbf{joint probability distributions}. Recommended 
introductory literature on these matters are, e.g., Toutenburg 
(2005)~\ct[p~57ff]{tou2005} and Kredler (2003)~\ct[Ch.~2]{kre2003}.

\medskip
\noindent
\underline{\textbf{Def.:}} A real-valued two-dimensional
\textbf{random variable} is defined as a one-to-one mapping
%
\be
\left(X,Y\right): \boldsymbol{\Omega} \rightarrow D \subseteq 
\mathbb{R}^{2}
\ee
%
of the sample space $\boldsymbol{\Omega}$ of some random 
experiment with associated probability space 
$\left(\boldsymbol{\Omega}, {\cal P}, P\right)$ into a subset $D$ 
of the two-dimensional Euclidian space $\mathbb{R}^{2}$.

\medskip
\noindent
We proceed by sketching some important concepts relating to 
two-dimensional random variables.

%------------------------------------------------------------------
\subsection[Joint probability distributions]{Joint probability 
distributions}
%------------------------------------------------------------------
%------------------------------------------------------------------
\subsubsection[Discrete case]{\underline{Discrete case:}}
%------------------------------------------------------------------
Two-dimensional \textbf{discrete random variables} possess a

\medskip
\noindent
\textbf{Spectrum of values:}
%
\be
(X,Y) \mapsto (x,y) \in \left\{x_{1}, \ldots, x_{k}\right\} \times
\left\{y_{1}, \ldots, y_{l}\right\}
\subset \mathbb{R}^{2} \ ,
\quad\quad\text{with}\quad k, l \in \mathbb{N} \ .
\ee
%
All pairs of values $(x_{i},y_{j})_{i=1,\ldots,k; j=1,\ldots,l}$ 
in this spectrum are assigned individual probabilities $p_{ij}$ by 
a real-valued

\medskip
\noindent
\textbf{Joint probability function:}
%
\be
\fbox{$\displaystyle
P(X=x_{i},Y=y_{j}) = p_{ij}
\quad\quad\text{for}\quad i=1, \dots, k;
j=1, \dots, l \ ,
$}
\ee
%
with properties\\[-5mm]
%
\begin{center}
\begin{itemize}
\item[(i)] $0 \leq p_{ij} \leq 1$, and
\hfill (\textbf{non-negativity})\\[-5mm]

\item[(ii)] ${\displaystyle\sum_{i=1}^{k}\sum_{j=1}^{l}p_{ij}=1}$.
\hfill (\textbf{normalisability})\\[-5mm]
\end{itemize}
\end{center}
%
By analogy to the case of one-dimensional random variables, 
specific \textbf{event probabilities} for $(X,Y)$ are obtained from 
the associated

\medskip
\noindent
\textbf{Joint cumulative distribution function (\texttt{cdf}):}
%
\be
\fbox{$\displaystyle
F_{XY}(x,y) = \texttt{cdf}(x,y) := P(X \leq x,Y \leq y)
= \sum_{i|x_{i}\leq x}\sum_{j|y_{j}\leq y}p_{ij} \ .
$}
\ee
%

%------------------------------------------------------------------
\subsubsection[Continous case]{\underline{Continuous case:}}
%------------------------------------------------------------------
For two-dimensional \textbf{continuous random variables} the range 
can be represented by the

\medskip
\noindent
\textbf{Spectrum of values:}
%
\be
(X,Y) \mapsto (x,y) \in D = (x_\mathrm{min},x_\mathrm{max}) \times 
(y_\mathrm{min},y_\mathrm{max}) \subseteq \mathbb{R}^{2} \ .
\ee
%
Probabilities are now assigned to infinitesimally small areas
$\mathrm{d}x \times \mathrm{d}y \in D$ by means of a real-valued

\medskip
\noindent
\textbf{Joint probability density function (\texttt{pdf}):}
%
\be
\fbox{$\displaystyle
f_{XY}(x,y) = \texttt{pdf}(x,y) \ ,
$
}
\ee
%
with properties:\\[-5mm]
%
\begin{center}
\begin{itemize}
\item[(i)] $f_{XY}(x,y) \geq 0$ for all $(x,y) \in D$, and
\hfill (\textbf{non-negativity})\\[-5mm]

\item[(ii)] ${\displaystyle\int_{-\infty}^{+\infty}
\int_{-\infty}^{+\infty}
f_{XY}(x,y)\,\mathrm{d}x\mathrm{d}y=1}$.
\hfill (\textbf{normalisability})\\[-5mm]

\end{itemize}
\end{center}
%
Approximately, one now has
%
\[
P(X \in \mathrm{d}x, Y \in \mathrm{d}y) \approx f_{XY}(\xi, \eta)\,
\mathrm{d}x\mathrm{d}y \ ,
\]
%
for representative $\xi \in \mathrm{d}x$ and $\eta \in
\mathrm{d}y$. Specific \textbf{event probabilities} for $(X,Y)$ are
obtained from the associated

\medskip
\noindent
\textbf{Joint cumulative distribution function (\texttt{cdf}):}
%
\be
\fbox{$\displaystyle
F_{XY}(x,y) = \texttt{cdf}(x,y) := P(X \leq x,Y \leq y)
= \int_{-\infty}^{x}\int_{-\infty}^{y}f_{XY}(t,u)\,
\mathrm{d}t\mathrm{d}u \ .
$}
\ee
%

%------------------------------------------------------------------
\subsection[Marginal and conditional distributions]{Marginal and 
conditional probability distributions}
%------------------------------------------------------------------
%------------------------------------------------------------------
\subsubsection[Discrete case]{\underline{Discrete case:}}
%------------------------------------------------------------------
The univariate \textbf{marginal probability functions} for $X$ and
$Y$ induced by the joint probability function $P(X=x_{i},Y=y_{j}) 
= p_{ij}$ are
%
\be
\lb{eq:margxdiscr}
p_{i+} := \sum_{j=1}^{l}p_{ij} = P(X=x_{i}) 
\quad\quad\text{for}\quad i=1,\ldots,k \ ,
\ee
%
and
%
\be
\lb{eq:margydiscr}
p_{+j} := \sum_{i=1}^{k}p_{ij} = P(Y=y_{j})
\quad\quad\text{for}\quad j=1,\ldots,l \ .
\ee
%
In addition, one defines \textbf{conditional probability functions} 
for $X$ given $Y=y_{j}$, with $p_{+j}>0$, and for $Y$ given 
$X=x_{i}$, with $p_{i+}>0$, by
%
\be
\lb{eq:condxdiscr}
p_{i|j} := \frac{p_{ij}}{p_{+j}} = P(X=x_{i}|Y=y_{j})
\quad\quad\text{for}\quad i=1,\ldots,k \ ,
\ee
%
respectively
%
\be
\lb{eq:condydiscr}
p_{j|i} := \frac{p_{ij}}{p_{i+}} = P(Y=y_{j}|X=x_{i})
\quad\quad\text{for}\quad j=1,\ldots,l \ .
\ee
%

%------------------------------------------------------------------
\subsubsection[Continous case]{\underline{Continuous case:}}
%------------------------------------------------------------------
The univariate \textbf{marginal probability density functions} for
$X$ and $Y$ induced by the joint probability density function 
$f_{XY}(x,y)$ are
%
\be
\lb{eq:margxcont}
f_{X}(x) = \int_{-\infty}^{+\infty}f_{XY}(x,y)\,\mathrm{d}y \ ,
\ee
%
and
%
\be
\lb{eq:margycont}
f_{Y}(y) = \int_{-\infty}^{+\infty}f_{XY}(x,y)\,\mathrm{d}x \ .
\ee
%
Moreover, one defines \textbf{conditional probability density 
functions} for $X$ given $Y$, and for $Y$ given $X$, by
%
\be
\lb{eq:condxcont}
f_{X|Y}(x|y) := \frac{f_{XY}(x,y)}{f_{Y}(y)}
\quad\quad\text{for}\quad f_{Y}(y)>0 \ ,
\ee
%
respectively
%
\be
\lb{eq:condycont}
f_{Y|X}(y|x) := \frac{f_{XY}(x,y)}{f_{X}(x)}
\quad\quad\text{for}\quad f_{X}(x)>0 \ .
\ee
%

%------------------------------------------------------------------
\subsection[Bayes' theorem for two-dimensional random variables]{Bayes' theorem for two-dimensional random variables}
%------------------------------------------------------------------
The concept of a bivariate joint probability distribution is at 
the heart of the formulation of Bayes' theorem, 
Eq.~(\ref{eq:bayes}), for a real-valued two-dimensional random 
variable $(X,Y)$.

%------------------------------------------------------------------
\subsubsection[Discrete case]{\underline{Discrete case:}}
%------------------------------------------------------------------
Let $P(X=x_{i}) = p_{i+} > 0$ be a \textbf{prior probability 
function} for a discrete random variable $X$. Then, on the grounds 
of a joint probability function $P(X=x_{i},Y=y_{j}) = p_{ij}$ and
Eqs.~(\ref{eq:condxdiscr}) and (\ref{eq:condydiscr}), the
\textbf{posterior probability function} for $X$ given $Y=y_{j}$,
with$P(Y=y_{j}) = p_{+j} > 0$, is determined by
%
\be
p_{i|j} = \frac{p_{j|i}}{p_{+j}}\,p_{i+}
\quad\quad\text{for}\quad i=1,\ldots,k \ .
\ee
%
By using Eqs.~(\ref{eq:margydiscr}) and (\ref{eq:condydiscr}) to 
re-expressed the denominator $p_{+j}$, this may be given in the 
standard form
%
\be
\lb{eq:bayesdiscr2drandvars}
\fbox{$\displaystyle
p_{i|j} = \frac{p_{j|i}\,p_{i+}}{\displaystyle\sum_{i=1}^{k}p_{j|i}
\,p_{i+}}
\quad\quad\text{for}\quad i=1,\ldots,k \ .
$}
\ee
%

%------------------------------------------------------------------
\subsubsection[Continous case]{\underline{Continuous case:}}
%------------------------------------------------------------------
Let $f_{X}(x) > 0$ be a \textbf{prior probability density function} 
for a continuous random variable $X$. Then, on the grounds 
of a joint probability density function $f_{XY}(x,y)$ and
Eqs.~(\ref{eq:condxcont}) and (\ref{eq:condycont}), the
\textbf{posterior probability density function} for $X$ given $Y$,
with $f_{Y}(y)>0$, is determined by
%
\be
f_{X|Y}(x|y) = \frac{f_{Y|X}(y|x)}{f_{Y}(y)}\,f_{X}(x) \ .
\ee
%
By using Eqs.~(\ref{eq:margycont}) and (\ref{eq:condycont}) to 
re-expressed the denominator $f_{Y}(y)$, this may be stated in the 
standard form
%
\be
\lb{eq:bayescont2drandvars}
\fbox{$\displaystyle
f_{X|Y}(x|y) = 
\frac{f_{Y|X}(y|x)\,f_{X}(x)}{\displaystyle\int_{-\infty}^{+\infty}
f_{Y|X}(y|x)\,f_{X}(x)\,\mathrm{d}x} \ .
$}
\ee
%

\medskip
\noindent
In practical applications, evaluation of the, at times intricate, 
single and double integrals contained in this representation of 
Bayes' theorem is managed by employing sophisticated numerical 
approxi\-mation techniques; cf. Saha (2002)~\ct{sah2002},
Sivia and Skilling (2006)~\ct{sivski2006}, Greenberg
(2013)~\ct{gre2013}, Gelman \textit{et al} (2014)~\ct{geletal2014},
or McElreath (2016)~\ct{mce2016}.

%------------------------------------------------------------------
\subsection[Covariance and correlation]{Covariance and correlation}
\lb{subsec:2dcovtheoret}
%------------------------------------------------------------------
We conclude this section by reviewing the standard measures for 
characterising the degree of \textbf{stochastic association}
between two random variables $X$ and $Y$.

\medskip
\noindent
The \textbf{covariance} of $X$ and $Y$ is defined by
%
\be
\lb{eq:2dvarcov}
\mathrm{Cov}(X,Y) := \mathrm{E}\left[\left(X-\mathrm{E}(X)\right)
\left(Y-\mathrm{E}(Y)\right)\right] \ .
\ee
%
It constitutes the off-diagonal component of the symmetric 
$\boldsymbol{(2 \times 2)}$ \textbf{covariance matrix}
%
\be
\lb{eq:2dcovmattheoret}
\boldsymbol{\Sigma}(X,Y) :=
\left(\begin{array}{cc}
\mathrm{Var}(X) & \mathrm{Cov}(X,Y) \\
\mathrm{Cov}(X,Y) & \mathrm{Var}(Y)
\end{array}\right) \ ,
\ee
%
which is regular and thus invertible as long as 
$\det[\boldsymbol{\Sigma}(X,Y)] \neq 0$.

\medskip
\noindent
By a suitable normalisation procedure, one defines from 
Eq.~(\ref{eq:2dvarcov}) the \textbf{correlation coefficient} of $X$ 
and $Y$ as
%
\be
\lb{eq:2dvarcorrel}
\rho(X,Y) := \frac{\mathrm{Cov}(X,Y)}{\sqrt{\mathrm{Var}(X)}
\sqrt{\mathrm{Var}(Y)}} \ .
\ee
%
This features as the off-diagonal component in the symmetric 
$\boldsymbol{(2 \times 2)}$ \textbf{correlation matrix}
%
\be
\lb{eq:2dcorrelmattheoret}
\boldsymbol{R}(X,Y) :=
\left(\begin{array}{cc}
1 & \rho(X,Y) \\
\rho(X,Y) & 1
\end{array}\right) \ ,
\ee
%
which is positive definite and thus invertible for $0 < 
\det[\boldsymbol{R}(X,Y)] = 1-\rho^{2} \leq 1$.

\medskip
\noindent
\underline{\textbf{Def.:}} Two random variables $X$ and $Y$ are 
referred to as \textbf{mutually stochastically independent}
provided that
%
\be
\lb{eq:stochindep2}
\mathrm{Cov}(X,Y) = 0
\qquad\Leftrightarrow\qquad
\rho(X,Y) = 0 \ .
\ee
%
It then follows that
%
\be
\lb{eq:stochindep3}
P(X \leq x,Y \leq y) = P(X \leq x) \times P(Y\leq y)
\quad\Leftrightarrow\quad
F_{XY}(x,y) = F_{X}(x) \times F_{Y}(y)
\ee
%
for $(x,y) \in D \subseteq \mathbb{R}^{2}$. Moreover, in this case 
(i)~$\mathrm{E}(X \times Y) = \mathrm{E}(X) \times \mathrm{E}(Y)$,
and (ii)~$\mathrm{Var}(aX+bY) = a^{2}\mathrm{Var}(X)+b^{2}
\mathrm{Var}(Y)$.

\vspace{5mm}
\noindent
In the next chapter we will highlight a number of standard 
univariate probability distributions for discrete and continuous 
one-dimensional random variables.

%%%%%%%%%%%%%%%%%%%%%%%%%%%%%%%%%%%%%%%%%%%%%%%%%%%%%%%%%%%%%%%%%%%
%%%%%%%%%%%%%%%%%%%%%%%%%%%%%%%%%%%%%%%%%%%%%%%%%%%%%%%%%%%%%%%%%%%
