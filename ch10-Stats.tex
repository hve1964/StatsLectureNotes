%%%%%%%%%%%%%%%%%%%%%%%%%%%%%%%%%%%%%%%%%%%%%%%%%%%%%%%%%%%%%%%%%%%
%  File name: ch10-Stats.tex
%  Title:
%  Version: 12.08.2019 (hve)
%%%%%%%%%%%%%%%%%%%%%%%%%%%%%%%%%%%%%%%%%%%%%%%%%%%%%%%%%%%%%%%%%%%
%%%%%%%%%%%%%%%%%%%%%%%%%%%%%%%%%%%%%%%%%%%%%%%%%%%%%%%%%%%%%%%%%%%
\chapter[Random sampling of target populations]{\href{https://www.youtube.com/watch?v=ApHx0GE3zQI}{Random sampling of target populations}}
\lb{ch10}
%%%%%%%%%%%%%%%%%%%%%%%%%%%%%%%%%%%%%%%%%%%%%%%%%%%%%%%%%%%%%%%%%%%
\textbf{Quantitative--empirical research methods} may be employed
for \textbf{exploratory} as well as for \textbf{confirmatory data
analysis}. Here we will focus on the latter, in the context of a
\textbf{frequentist viewpoint} of \textbf{Probability Theory} and
\textbf{statistical inference}. To investigate \textbf{research
questions} systematically by statistical means, with the objective
to make inferences about the distributional properties of a set of
\textbf{statistical variables} in a specific \textbf{target 
population}~$\boldsymbol{\Omega}$ of study objects, on the basis 
of analysis of data from just a few units in a \textbf{sample} 
$\boldsymbol{S_{\Omega}}$, the following three issues have to be 
addressed in a clearcut fashion:
%
\begin{itemize}
\item[(i)] the \textbf{target population} $\boldsymbol{\Omega}$ of 
the research activity needs to be defined in an unambiguous way,

\item[(ii)] an adequate \textbf{random 
sample}~$\boldsymbol{S_{\Omega}}$ needs to be drawn from an 
underlying \textbf{sampling frame}~$\boldsymbol{L_{\Omega}}$ 
associated with $\boldsymbol{\Omega}$, and

\item[(iii)] a reliable mathematical procedure for
\textbf{estimating quantitative population parameters} from random
sample data needs to be employed.
\end{itemize}
%
We will briefly discuss these issues in turn, beginning with a 
review in Tab.~\ref{tab:notation} of conventional \textbf{notation} 
for distinguishing specific statistical measures relating to target
populations $\boldsymbol{\Omega}$ on the one-hand side from the 
corresponding ones relating to random samples 
$\boldsymbol{S_{\Omega}}$ on the other.
%
\begin{table}
\begin{center}
\begin{tabular}[!h]{c|c}
    	\hline \\
      \textbf{Target population} $\boldsymbol{\Omega}$ &
      \textbf{Random sample} $\boldsymbol{S_{\Omega}}$ \\ \\
      \hline \\
      population size $N$ & sample size $n$ \\ \\
%      proportion $\pi$ & sample proportion
%      ${\displaystyle p_{n} = \frac{O_{n}}{n}}$, with \\
%       & standard error
%      ${\displaystyle \sqrt{\frac{p_{n}(1-p_{n})}{n}}}$ \\ \\
      arithmetical mean $\mu$ & sample mean $\bar{X}_{n}$
%, with \\
%      & standard error ${\displaystyle \frac{S_{n}}{\sqrt{n}}}$
      \\ \\
      standard deviation $\sigma$ & sample standard deviation
      $S_{n}$ \\ \\
      median $\tilde{x}_{0.5}$ & sample median
      $\tilde{X}_{0.5,n}$ \\ \\
      correlation coefficient $\rho$ & sample correlation
      coefficient $r$ \\ \\
      rank correlation coefficient $\rho_{S}$ & sample rank
      correl. coefficient $r_{S}$ \\ \\
      regression coefficient (intercept) $\alpha$ & sample 
      regression intercept $a$ %\\
%      & standard error ${\displaystyle n}$
      \\ \\
      regression coefficient (slope) $\beta$ & sample regression 
      slope $b$ %\\
%      & standard error ${\displaystyle n}$
      \\ \\
      \hline
\end{tabular}
\end{center}
\caption{Notation for distinguishing between statistical 
measures relating to a target population~$\boldsymbol{\Omega}$ on 
the one-hand side, and to the corresponding quantities and 
unbiased maximum likelihood point estimator functions obtained 
from a random sample~$\boldsymbol{S_{\Omega}}$ on the other.}
\lb{tab:notation}
\end{table}
%

\medskip
\noindent
One-dimensional \textbf{random variables} in a target population 
$\boldsymbol{\Omega}$ (of size $N$), as what \textbf{statistical 
variables} will be understood to constitute subsequently, will be
denoted by capital Latin letters such as $X$, $Y$, \ldots, $Z$,
while their \textbf{realisations} in random samples
$\boldsymbol{S_{\Omega}}$ (of size $n$) will be denoted by lower
case Latin letters such as $x_{i}$, $y_{i}$, \ldots, $z_{i}$
($i=1,\ldots,n$). In addition, one denotes \textbf{population
parameters} by lower case Greek letters, while for their
corresponding \textbf{point estimator functions} relating to random
samples, which are also perceived as random variables, again
capital Latin letters are used for representation. The ratio $n/N$
will be referred to as the \textbf{sampling fraction}. As is
standard in the statistical literature, we will denote a
particular \textbf{random sample} of size $n$ for a one-dimensional
random variable $X$ by a set $\boldsymbol{S_{\Omega}}$:~$(X_{1},
\ldots, X_{n})$, with $X_{i}$ representing any arbitrary random
variable associated with $X$ in this sample.

\medskip
\noindent
In actual practice, it is often not possible to acquire access for 
the purpose of enquiry to every single statistical unit belonging 
to an identified target population $\boldsymbol{\Omega}$, not 
even in principle. For example, this could be due to the fact that 
$\boldsymbol{\Omega}$'s size $N$ is far too large to be determined 
accurately. In this case, to ensure a reliable investigation, one 
needs to resort to using a \textbf{sampling frame} 
$\boldsymbol{L_{\Omega}}$ for $\boldsymbol{\Omega}$. By this one 
understands a representative list of  elements in 
$\boldsymbol{\Omega}$ to which access can actually be obtained one
way or another. Such a list will have to be compiled by some
authority of scientific integrity. In an attempt to avoid a
notational overflow in the following, we will continue to use~$N$
to denote \textit{both}: the size of the \textbf{target 
population}~$\boldsymbol{\Omega}$ and the  size of its associated 
\textbf{sampling frame}~$\boldsymbol{L_{\Omega}}$ (even though this 
is not entirely accurate). As regards the specific sampling
process, one may distinguish \textbf{cross-sectional} one-off
sampling at a fixed instant from \textbf{longitudinal} multiple
sampling over a finite time interval.\footnote{In a sense,
cross-sectional sampling will yield a ``snapshot'' of a target
population of interest in a particular state, while longitudinal
sampling is the basis for producing a ``film'' featuring a
particular evolutionary aspect of a target population of interest.}

\medskip
\noindent
We now proceed to introduce the three most commonly practiced 
methods of drawing \textbf{random samples} from given fixed target
populations~$\boldsymbol{\Omega}$ of statistical units.

%%%%%%%%%%%%%%%%%%%%%%%%%%%%%%%%%%%%%%%%%%%%%%%%%%%%%%%%%%%%%%%%%%%
\section[Random sampling methods]{Random sampling methods}
\lb{sec:sampling}
%%%%%%%%%%%%%%%%%%%%%%%%%%%%%%%%%%%%%%%%%%%%%%%%%%%%%%%%%%%%%%%%%%%
%------------------------------------------------------------------
\subsection[Simple random sampling]{Simple random sampling}
\lb{subsec:einfzustich}
%------------------------------------------------------------------
The \textbf{simple random sampling} technique can be best
understood in terms of the \textbf{urn model} of
\textbf{combinatorics} introduced in Sec.~\ref{sec:comb}. Given a
target population~$\boldsymbol{\Omega}$ (or sampling frame 
$\boldsymbol{L_{\Omega}}$) of $N$ distinguishable statistical 
units, there is a total of $\left(\begin{array}{c} N \\ n 
\end{array}\right)$ distinct possibilities of drawing samples of 
size $n$ from $\boldsymbol{\Omega}$ (or 
$\boldsymbol{L_{\Omega}}$), given the order of selection is 
\textit{not} being accounted for and \textit{excluding
repetitions}, see Sec.~\ref{subsec:combvar}. A \textbf{simple
random sample} is then defined by the property that its
probability of selection is equal to
%
\be
\frac{1}{\left(\begin{array}{c} N \\ n \end{array}\right)} \ ,
\ee
%
according to the Laplacian principle of Eq.~(\ref{eq:classprob}). 
This has the immediate consequence that the \textit{a priori} 
probability of selection of any single statistical unit is given 
by\footnote{In the statistical literature this particular property 
of a random sample is referred to as ``epsem'': equal probability 
of selection method.}
%
\be
1 - \frac{\left(\begin{array}{c} N-1 \\ n \end{array}\right)}{
\left(\begin{array}{c} N \\ n \end{array}\right)}
= 1 - \frac{N-n}{N} = \frac{n}{N} \ .
\ee
%
On the other hand, the probability that two statistical units $i$ 
and $j$ will be members of the \textit{same} sample of size~$n$
amounts to
%
\be
\frac{n}{N} \times \frac{n-1}{N-1} \ .
\ee
%
As such, by Eq.~(\ref{eq:stochindep1}), this type of a selection 
procedure of two statistical units proves \textit{not} to yield two 
stochastically independent units (in which case the joint 
probability of selection would be $n/N \times n/N$). However, for 
sampling fractions $n/N \leq 0.05$, stochastic independence of the 
selection of statistical units generally holds to a reasonably 
good approximation. When, in addition, the sample size is $n
\geq 50$, the conditions for the \textbf{central limit theorem} in
the variant of Lindeberg and L\'{e}vy (cf.
Sec.~\ref{sec:zentrgrenz}) to apply often hold to a fairly good
degree.

%------------------------------------------------------------------
\subsection[Stratified random sampling]{Stratified random sampling}
\lb{subsec:stratsampling}
%------------------------------------------------------------------
\textbf{Stratified random sampling} adapts the sampling process to
a known intrinsic structure of the target 
population~$\boldsymbol{\Omega}$ (and its associated sampling 
frame $\boldsymbol{L_{\Omega}}$), as provided by the $k$ mutually 
exclusive and exhaustive categories of some qualitative (nominal 
or ordinal) variable; these thus define a set of $k$
\textbf{strata} (layers) of $\boldsymbol{\Omega}$ (or
$\boldsymbol{L_{\Omega}}$). By construction, there are $N_{i}$
statistical units belonging to the $i$th stratum ($i=1, \ldots,
k$). Simple random samples of sizes $n_{i}$ are drawn from each
stratum according to the principles outlined in
Sec.~\ref{subsec:einfzustich}, yielding a total sample of size
$n=n_{1}+\ldots+n_{k}$. Frequently applied variants of this
sampling technique are (i)~\textbf{proportionate allocation} of
statistical units, defined by the condition\footnote{Note that,
thus, this also has the ``epsem'' property.}
%
\be
\frac{n_{i}}{n} \stackrel{!}{=} \frac{N_{i}}{N}
\qquad\Rightarrow\qquad
\frac{n_{i}}{N_{i}} = \frac{n}{N} \ ;
\ee
%
in particular, this allows for a fair representation of minorities 
in $\boldsymbol{\Omega}$, and (ii)~\textbf{optimal allocation} of 
statistical units which aims at a minimisation of the resultant 
sampling errors of the variables investigated. Further details on 
the stratified random sampling technique can be found, e.g., in 
Bortz and D\"{o}ring (2006)~\ct[p~425ff]{bordoe2006}.

%------------------------------------------------------------------
\subsection[Cluster random sampling]{Cluster random sampling}
\lb{subsec:clustersampling}
%------------------------------------------------------------------
When the target population~$\boldsymbol{\Omega}$ (and its 
associated sampling frame $\boldsymbol{L_{\Omega}}$) naturally 
subdivides into an exhaustive set of $K$ mutually exclusive
\textbf{clusters} of statistical units, a convenient sampling
strategy is given by selecting $k < K$ clusters from this set at
random and perform complete surveys within each of the chosen
clusters. The probability of selection of any particular
statistical unit from $\boldsymbol{\Omega}$ (or
$\boldsymbol{L_{\Omega}}$) thus amounts to $k/K$. This
\textbf{cluster random sampling} method has the practical advantage
of being less contrived. However, in general it entails sampling
errors that are greater than for the previous two sampling methods.
Further details on the cluster random sampling technique can be
found, e.g., in Bortz and D\"{o}ring
(2006)~\ct[p~435ff]{bordoe2006}.

\vspace{5mm}
\noindent
We emphasise at this point that empirical data gained from
\textbf{convenience samples} (in contrast to random samples) is
\textit{not} amenable to \textbf{statistical inference}, in that
its information content \textit{cannot} be generalised to the
target population~$\boldsymbol{\Omega}$ from which it was drawn;
see, e.g., Bryson (1976)~\ct[p~185]{bry1976}, or Schnell \textit{et
al} (2013)~\ct[p~289]{schetal2013}.

%%%%%%%%%%%%%%%%%%%%%%%%%%%%%%%%%%%%%%%%%%%%%%%%%%%%%%%%%%%%%%%%%%%
\section[Point estimator functions]{Point estimator functions}
%%%%%%%%%%%%%%%%%%%%%%%%%%%%%%%%%%%%%%%%%%%%%%%%%%%%%%%%%%%%%%%%%%%
Many \textbf{inferential statistical methods of data analysis} in
the \textbf{frequentist framework} revolve around the
\textbf{estimation} of unknown \textbf{distribution parameters}
$\theta$ with respect to some target
population~$\boldsymbol{\Omega}$ by means of corresponding
\textbf{maximum likelihood point estimator functions}
$\hat{\theta}_{n}(X_{1},\ldots,X_{n})$ (or: 
\textbf{statistics}), the values of which are computed from the
data of \textbf{random samples}~$\boldsymbol{S_{\Omega}}$:~$(X_{1}, 
\ldots, X_{n})$. Owing to the stochastic nature of the random 
sampling process, any point estimator function 
$\hat{\theta}_{n}(X_{1},\ldots,X_{n})$ is subject to a
\textbf{random sampling error}. One can show that this estimation
procedure becomes reliable provided that a point estimator function 
satisfies the following two important criteria of quality:
%
\begin{itemize}
\item[(i)] \textbf{Unbiasedness:}
$\mathrm{E}(\hat{\theta}_{n})=\theta$, and

\item[(ii)] \textbf{Consistency:}
$\displaystyle\lim_{n\to\infty}\mathrm{Var}(\hat{\theta}_{n})=0$.
\end{itemize}
%
For metrically scaled one-dimensional random variables $X$, 
defining for a given random sample 
$\boldsymbol{S_{\Omega}}$:~$(X_{1}, \ldots, X_{n})$ of size $n$ a 
\textbf{sample total sum} by
%
\be
\lb{eq:samptot}
Y_{n} := \sum_{i=1}^{n}X_{i} \ ,
\ee
%
the two most prominent \textbf{maximum likelihood point estimator 
functions} satisfying the \textbf{unbiasedness} and
\textbf{consistency} conditions are the \textbf{sample mean} and
\textbf{sample variance}, defined by
%
\bea
\lb{eq:sampmean}
\bar{X}_{n} & := & \frac{1}{n}\,Y_{n} \\
%
\lb{eq:sampvar}
S_{n}^{2} & := & 
\frac{1}{n-1}\,\sum_{i=1}^{n}(X_{i}-\bar{X}_{n})^{2} \ .
\eea
%
These will be frequently employed in subsequent considerations in 
Ch.~\ref{ch12} for point-estimating the values of the
\textbf{location} and \textbf{scale parameters}~$\mu$ and 
$\sigma^{2}$ of the distribution for a one-dimensional random 
variable~$X$ in a target population~$\boldsymbol{\Omega}$.
\textbf{Sampling theory} in the \textbf{frequentist framework}
holds it that the \textbf{standard errors (SE)} associated with the 
maximum likelihood point estimator functions $\bar{X}_{n}$ and 
$S_{n}^{2}$, defined in Eqs.~(\ref{eq:sampmean}) and 
(\ref{eq:sampvar}), amount to the standard deviations of the 
underlying theoretical \textbf{sampling distributions} for these 
functions; see, e.g., Cram\'{e}r (1946)~\ct[Chs.~27 to 
29]{cra1946}. For a given target population~$\boldsymbol{\Omega}$ 
(or sampling frame $\boldsymbol{L_{\Omega}}$) of size $N$, imagine 
drawing all possible $\left(\begin{array}{c} N \\ n 
\end{array}\right)$ mutually independent random samples of a fixed 
size $n$ (no order accounted for and repetitions excluded), from 
each of which individual realisations of $\bar{X}_{n}$ and 
$S_{n}^{2}$ are obtained. The theoretical distributions 
for all such realisations of $\bar{X}_{n}$ resp.~$S_{n}^{2}$ for 
given $N$ and $n$ are referred to as their corresponding
\textbf{sampling distributions}. A useful simulation illustrating
the concept of a sampling distribution is available at the website 
\href{http://onlinestatbook.com/stat_sim/sampling_dist/index.html}
{\texttt{onlinestatbook.com}}. In the limit that $N \to \infty$
while keeping $n$ fixed, the theoretical \textbf{sampling
distributions} of $\bar{X}_{n}$ and $S_{n}^{2}$ become normal (cf. 
Sec.~\ref{sec:normverteil}) resp.~$\chi^{2}$ with $n-1$ degrees of 
freedom (cf. Sec.~\ref{sec:chi2verteil}), with standard deviations
%
\bea
\lb{eq:sesammean}
\text{SE}\bar{X}_{n} & := & \frac{S_{n}}{\sqrt{n}} \\
%
\lb{eq:sesamvar}
\text{SE}S_{n}^{2} & := & \sqrt{\frac{2}{n-1}}\,S_{n}^{2} \ ;
\eea
%
cf., e.g., Lehman and Casella (1998)~\ct[p~91ff]{lehcas1998},
and Levin \textit{et al} (2010)~\ct[Ch.~6]{levetal2009}. 
Thus, for a \textit{finite} sample standard deviation $S_{n}$, 
these two \textbf{standard errors} decrease with the sample 
size~$n$ in proportion to the inverse of $\sqrt{n}$ resp.~the 
inverse of $\sqrt{n-1}$. It is a main criticism of proponents of 
the \textbf{Bayes--Laplace approach} to \textbf{Probability Theory}
and \textbf{statistical inference} that the concept of a
\textbf{sampling distribution} for a maximum likelihood point
estimator function is based on \textit{unobserved data}; cf.
Greenberg (2013)~\ct[p~31f]{gre2013}.

\medskip
\noindent
There are likewise unbiased maximum likelihood point estimators 
for the \textbf{shape} parameters $\gamma_{1}$ and $\gamma_{2}$ of 
the probability distribution for a one-dimensional random 
variable~$X$ in a target population~$\boldsymbol{\Omega}$, as 
given in Eqs.~(\ref{eq:skew2}) and (\ref{eq:kurt2}). For $n>2$ 
resp.~$n>3$, the \textbf{sample skewness} and \textbf{sample excess 
kurtosis} in, e.g., their implementation in the software packages 
\R\ (package: \texttt{e1071}, by Meyer \textit{et al}
(2019)~\ct{meyetal2019}) or SPSS are defined by (see, e.g.,
Joanes and Gill (1998)~\ct[p~184]{joagil1998})
%
\bea
\lb{eq:sampskew}
G_{1} & := & 
\frac{\sqrt{(n-1)n}}{n-2}\,\frac{\frac{1}{n}\,\sum_{i=1}^{n}(X_{i}
-\bar{X}_{n})^{3}}{\left(\frac{1}{n}\,\sum_{j=1}^{n}(X_{j}
-\bar{X}_{n})^{2}\right)^{3/2}}\\
%
\lb{eq:sampkurt}
G_{2} & := & 
\frac{n-1}{(n-2)(n-3)}\,\left[(n+1)\left(\frac{\frac{1}{n}
\,\sum_{i=1}^{n}(X_{i}-\bar{X}_{n})^{4}}{\left(\frac{1}{n}
\,\sum_{j=1}^{n}(X_{j}-\bar{X}_{n})^{2}\right)^{2}}-3\right)
+6\right] \ ,
\eea
%
with associated standard errors (cf. Joanes and Gill 
(1998)~\ct[p~185f]{joagil1998})
%
\bea
\lb{eq:sesamskew}
\text{SE}G_{1} & := & \sqrt{\frac{6(n-1)n}{(n-2)(n+1)(n+3)}} \\
%
\lb{eq:sesamkurt}
\text{SE}G_{2} & := & 
2\,\sqrt{\frac{6(n-1)^{2}n}{(n-3)(n-2)(n+3)(n+5)}} \ .
\eea
%

%A major strength of inferential statistical tools of data 
%analysis, based on the random sampling of populations, is that 
%they (i)~make explicit \textbf{sampling errors}, and (ii)~provide 
%appropriate quantitative means to \textbf{estimate their size};
%the basis of their derivation is the respective \textbf{sampling 
%distribution}.

%%%%%%%%%%%%%%%%%%%%%%%%%%%%%%%%%%%%%%%%%%%%%%%%%%%%%%%%%%%%%%%%%%%
%%%%%%%%%%%%%%%%%%%%%%%%%%%%%%%%%%%%%%%%%%%%%%%%%%%%%%%%%%%%%%%%%%%
