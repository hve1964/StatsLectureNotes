%%%%%%%%%%%%%%%%%%%%%%%%%%%%%%%%%%%%%%%%%%%%%%%%%%%%%%%%%%%%%%%%%%%
%  File name: ch13-Stats.tex
%  Title:
%  Version: 20.06.2019 (hve)
%%%%%%%%%%%%%%%%%%%%%%%%%%%%%%%%%%%%%%%%%%%%%%%%%%%%%%%%%%%%%%%%%%%
%%%%%%%%%%%%%%%%%%%%%%%%%%%%%%%%%%%%%%%%%%%%%%%%%%%%%%%%%%%%%%%%%%%
\chapter[Bivariate methods of statistical data analysis]{Bivariate
methods of statistical data analysis: testing for association}
\lb{ch13}
%%%%%%%%%%%%%%%%%%%%%%%%%%%%%%%%%%%%%%%%%%%%%%%%%%%%%%%%%%%%%%%%%%%
Recognising patterns of regularity in the variability of data sets 
for given (observable) statistical variables, and explaining them 
in terms of \textbf{causal relationships} in the context of a 
suitable \textbf{theoretical model}, is one of the main objectives
of any empirical scientific discipline, and thus motivation for 
corresponding \textbf{research}; see, e.g., Penrose 
(2004)~\ct{pen2004}. Causal relationships are intimately related 
to \textbf{interactions} between objects or agents of the physical 
or/and of the social kind. A \textit{necessary} (though not 
sufficient) \textit{condition} on the way to theoretically
fathoming causal relationships is to establish empirically the
existence of significant \textbf{statistical associations} between
the variables in question. \textbf{Replication} of positive
observational or experimental results of this kind, when
accomplished, yields strong support in favour of this idea.
Regrettably, however, the existence of causal relationships between
two statistical variables \textit{cannot} be established with
absolute certainty by empirical means; compelling theoretical
arguments need to stand in. Causal relationships between
statistical variables imply an unambiguous distinction between
\textbf{independent variables} and \textbf{dependent variables}. In
the following, we will discuss the principles of the simplest three inferential statistical methods within the \textbf{frequentist
framework}, each associated with specific \textbf{null hypothesis
significance tests}, that provide empirical checks of the
aforementioned necessary condition in the \textbf{bivariate case}.

%%%%%%%%%%%%%%%%%%%%%%%%%%%%%%%%%%%%%%%%%%%%%%%%%%%%%%%%%%%%%%%%%%%
\section[Correlation analysis and linear regression]{\href{https://www.youtube.com/watch?v=aLcTqycIXgU}{Correlation analysis and
linear regression}}
\lb{sec:correl}
%%%%%%%%%%%%%%%%%%%%%%%%%%%%%%%%%%%%%%%%%%%%%%%%%%%%%%%%%%%%%%%%%%%
%------------------------------------------------------------------
\subsection[$t$--test for a correlation]{$\boldsymbol{t}$--test 
for a correlation}
\lb{subsec:correlttest}
%------------------------------------------------------------------
The parametric \textbf{correlation analysis} presupposes a
metrically scaled two-dimensional statistical variable $(X,Y)$ 
that can be assumed to satisfy a \textbf{bivariate normal 
distribution} in some target population~$\boldsymbol{\Omega}$. Its 
aim is to investigate whether or not the components $X$ and $Y$ 
feature a quantitative--statistical association of a
\textit{linear} nature, given a data matrix $\boldsymbol{X} \in 
\mathbb{R}^{n \times 2}$ obtained from a random sample of 
size~$n$. Formulated in terms of the \textbf{population correlation 
coefficient} $\rho$ according to 
\href{http://en.wikipedia.org/wiki/Auguste_Bravais}{Auguste
Bravais (1811--1863)} and
\href{http://www-history.mcs.st-and.ac.uk/Biographies/Pearson.html}
{Karl Pearson FRS (1857--1936)}, the method tests $H_{0}$ against 
$H_{1}$ in one of the alternative pairs of

\medskip
\noindent
\textbf{Hypotheses:} \hfill (test for association)
%
\be
\lb{eq:correlanahypo}
\begin{cases}
H_{0}: \rho = 0
\quad\text{or}\quad
\rho \geq 0
\quad\text{or}\quad
\rho \leq 0 \\
H_{1}: \rho \neq 0
\quad\text{or}\quad
\rho < 0
\quad\text{or}\quad
\rho > 0
\end{cases} \ ,
\ee
%
with $-1 \leq \rho \leq +1$.

\medskip
\noindent
For sample sizes~$n \geq 50$, the assumption of normality of the 
marginal $X$- and $Y$-distributions in a given random sample 
$\boldsymbol{S_{\Omega}}$:~$(X_{1}, \ldots, X_{n};\-Y_{1}, \ldots, 
Y_{n})$ drawn from $\boldsymbol{\Omega}$ can again be tested by 
means of the \textbf{Kolmogorov--Smirnov--test}; cf. 
Sec.~\ref{sec:onesampttest}. For sample sizes $n < 50$, on the 
other hand, the magnitudes of the \textbf{standardised skewness and
excess kurtosis measures}, Eqs.~(\ref{eq:g1g2ratios}), can be
considered instead. A \textbf{scatter plot} of the bivariate raw
sample data $\{(x_{i},y_{i})\}_{i=1,\ldots,n}$ displays
characteristic features of the \textbf{joint
$\boldsymbol{(X,Y)}$-distribution}.

\medskip
\noindent
\underline{\R:} \texttt{ks.test(\textit{variable}, "pnorm")} \\
\underline{SPSS:} Analyze $\rightarrow$ Nonparametric Tests
$\rightarrow$ Legacy Dialogs $\rightarrow$ 1-Sample K-S \ldots: 
Normal

\medskip
\noindent
Normalising the \textbf{sample correlation coefficient} $r$ of 
Eq.~(\ref{eq:correl}) by its \textbf{standard error},
%
\be
\text{SE}r :=\sqrt{\frac{1-r^{2}}{n-2}} \ ,
\ee
%
the latter of which can be derived from the corresponding 
theoretical \textbf{sampling distribution} for~$r$, presently
yields the (see, e.g., Toutenburg (2005)~\ct[Eq.~(7.18)]{tou2005})

\medskip
\noindent
\textbf{Test statistic:}
%
\be
\lb{eq:correlteststat}
\fbox{$\displaystyle
T_{n} := \frac{r}{\text{SE}r} \ \stackrel{H_{0}}{\sim}\ t(n-2) \ ,
$}
\ee
%
which, under $H_{0}$, satisfies a $\boldsymbol{t}$\textbf{--test
distribution} with $df=n-2$ degrees of freedom; cf.
Sec.~\ref{sec:tverteil}.

\medskip
\noindent
\textbf{Test decision:} Depending on the kind of test to be 
performed, the rejection region for $H_{0}$ at significance level 
$\alpha$ is given by
%
\begin{center}
\begin{tabular}[h]{c|c|c|c}
 & & & \\
\textbf{Kind of test} & $\boldsymbol{H_{0}}$ &
$\boldsymbol{H_{1}}$ &
\textbf{Rejection region for} $\boldsymbol{H_{0}}$ \\
 & & & \\
\hline
 & & & \\
(a)~two-sided & $\rho=0$ & $\rho \neq 0$ &
$|t_{n}|>t_{n-2;1-\alpha/2}$ \\
 & & & \\
\hline
 & & & \\
(b)~left-sided & $\rho \geq 0$ & $\rho<0$ &
$t_{n}<t_{n-2;\alpha}=-t_{n-2;1-\alpha}$ \\
 & & & \\
\hline
 & & & \\
(c)~right-sided & $\rho \leq 0$ & $\rho>0$ &
$t_{n}>t_{n-2;1-\alpha}$ \\
 & & &
\end{tabular}
\end{center}
%
$p$--values associated with realisations $t_{n}$ of 
(\ref{eq:correlteststat}), which are to be calculated from the
$\boldsymbol{t}$\textbf{--test distribution}, can  be obtained from 
Eqs.~(\ref{eq:pvaluetwo})--(\ref{eq:pvalueright}).

\medskip
\noindent
\underline{\R:} \texttt{cor.test(\textit{variable1},
\textit{variable2})}, \\
\texttt{cor.test(\textit{variable1}, \textit{variable2}, 
alternative = "less")}, \\
\texttt{cor.test(\textit{variable1}, \textit{variable2}, 
alternative = "greater")} \\
\underline{SPSS:} Analyze $\rightarrow$ Correlate
$\rightarrow$ Bivariate \ldots: Pearson

\medskip
\noindent
\textbf{Effect size:} The practical significance of the phenomenon investigated can be estimated directly from the absolute value of
the scale-invariant sample correlation coefficient~$r$ according to
Cohen's (1992)~\ct[Tab.~1]{coh1992}

\medskip
\noindent
\underline{\textbf{Rule of thumb:}}\\
$0.10 \leq |r| < 0.30$: small effect\\
$0.30 \leq |r| < 0.50$: medium effect\\
$0.50 \leq |r|$: large effect.

\medskip
\noindent
It is generally recommended to handle significant test results of 
\textbf{correlation analyses} for metrically scaled two-dimensional
statistical variables $(X,Y)$ with some care, due to the 
possibility of \textbf{spurious correlations} induced by additional 
\textbf{control variables} $Z, \ldots$, acting hidden in the 
background. To exclude this possibility, a correlation analysis 
should, e.g., be repeated for homogeneous subgroups of the 
sample~$\boldsymbol{S_{\Omega}}$. Some rather curious and 
startling cases of spurious correlations have been collected at 
the website
\href{http://www.tylervigen.com/}{\texttt{www.tylervigen.com}}.

%------------------------------------------------------------------
\subsection[$F$--test of a regression 
model]{$\boldsymbol{F}$--test of a regression model}
\lb{subsec:correlFtest}
%------------------------------------------------------------------
When a correlation in the joint distribution of a metrically 
scaled two-dimensional statistical variable~$(X,Y)$, significant 
in~$\boldsymbol{\Omega}$ at level~$\alpha$, proves to be 
\textit{strong}, i.e., when the magnitude of $\rho$ takes a value
in the interval
%
\[
0.71 \leq |\rho| \leq 1.0 \ ,
\]
%
it is meaningful to ask which \textit{linear} quantitative model 
best represents the detected linear statistical association; 
cf.~Pearson (1903)~\ct{pea1903}. To this end, \textbf{simple linear 
regression} seeks to devise a \textbf{linear stochastic regression 
model} for the target population~$\boldsymbol{\Omega}$ of the form
%
\be
\lb{eq:linregmod}
\text{in}\ \boldsymbol{\Omega}: \quad
Y_{i} = \alpha + \beta x_{i} + \varepsilon_{i}
\qquad (i=1, \ldots,n) \ ,
\ee
%
which, for instance, assigns $X$ the role of an \textbf{independent 
variable} (and so its values $x_{i}$ can be considered prescribed 
by the modeller) and $Y$ the role of a \textbf{dependent variable}; 
such a model is essentially \textbf{univariate} in nature. The
\textbf{regression coefficients} $\alpha$ and $\beta$ denote the 
unknown $\boldsymbol{y}$\textbf{--intercept} and \textbf{slope} of
the model in $\boldsymbol{\Omega}$. For the \textbf{random errors} 
$\varepsilon_{i}$ it is assumed that
%
\be
\lb{eq:linregassump}
\varepsilon_{i} \stackrel{\text{i.i.d.}}{\sim} N(0;\sigma^{2}) \ ,
\ee
%
meaning they are identically normally distributed (with zero mean 
and constant variance~$\sigma^{2}$) and mutually stochastically 
independent. With respect to the bivariate random sample 
$\boldsymbol{S_{\Omega}}$:~$(X_{1}, \ldots, X_{n};\-Y_{1}, \ldots, 
Y_{n})$, the supposed linear relationship between $X$ and $Y$ is 
expressed by
%
\be
\text{in}\ \boldsymbol{S_{\Omega}}: \quad
y_{i} = a + b x_{i} + e_{i} \qquad (i=1, \ldots,n) \ .
\ee
%
So-called \textbf{residuals} are then defined according to
%
\be
e_{i} := y_{i}-\hat{y}_{i} = y_{i} - a - b x_{i} \qquad (i=1, 
\ldots,n) \ ,
\ee
%
which, for given values of $x_{i}$, encode the differences between 
the observed realisations $y_{i}$ of $Y$ and the corresponding (by
the linear regression model) predicted values $\hat{y}_{i}$
of $Y$. Given the assumption expressed in
Eq.~(\ref{eq:linregassump}), the residuals must satisfy the
condition $\displaystyle\sum_{i=1}^{n}e_{i}=0$.

\medskip
\noindent
Next, introduce \textbf{sums of squared deviations} for the
$Y$-data, in line with the ANOVA procedure of Sec.~\ref{sec:anova},
i.e.,
%
\bea
\lb{eq:tssreg}
\text{TSS} & := & \sum_{i=1}^{n}(y_{i}-\bar{y})^{2} \\
%
\lb{eq:rssreg}
\text{RSS} & := & \sum_{i=1}^{n}(y_{i}-\hat{y}_{i})^{2}
\ = \ \sum_{i=1}^{n}e_{i}^{2} \ .
\eea
%
In terms of these quantities, the \textbf{coefficient of 
determination} of Eq.~(\ref{eq:linregcoeffdetdescr}) for assessing 
the \textbf{goodness-of-the-fit} of a regression model can be
expressed by
%
\be
\lb{eq:linregcoeffdet}
B = \frac{\text{TSS}-\text{RSS}}{\text{TSS}}
= \frac{(\text{total variance of}\ Y)-(\text{unexplained variance 
of}\ Y)}{(\text{total variance of}\ Y)} \ .
\ee
%
This normalised measure expresses the proportion of variability in 
a data set of $Y$ which can be explained by the corresponding 
variability of $X$ through the \textbf{best-fit regression model}. 
The range of $B$ is $0 \leq B \leq 1$.

\medskip
\noindent
In the methodology of a \textbf{regression analysis} within the
\textbf{frequentist framework}, the first issue to be addressed is
to test the significance of the overall \textbf{simple linear
regression model}~(\ref{eq:linregmod}), i.e., to test $H_{0}$
against $H_{1}$ in the set of

\medskip
\noindent
\textbf{Hypotheses:} \hfill (test for differences)
%
\be
\lb{linreganovahyp1}
\begin{cases}
H_{0}: \beta = 0 \\
H_{1}: \beta \neq 0
\end{cases} \ .
\ee
%
Exploiting the goodness-of-the-fit aspect of the regression model 
as quantified by $B$ in Eq.~(\ref{eq:linregcoeffdet}), one 
arrives via division by the \textbf{standard error} of $B$,
%
\be
\lb{eq:sersq}
\text{SE}B := \frac{1-B}{n-2} \ ,
\ee
%
which derives from the theoretical \textbf{sampling distribution}
for $B$, at the (see, e.g., Hatzinger and Nagel 
(2013)~\ct[Eq.~(7.8)]{hatnag2013})

\medskip
\noindent
\textbf{Test statistic:}\footnote{Note that with the identity 
$B=r^{2}$ of Eq.~(\ref{eq:linregrsq}), which applies in simple 
linear regression, this is just the square of the test 
statistic~(\ref{eq:correlteststat}).}
%
\be
\lb{eq:linregteststat}
\fbox{$\displaystyle
T_{n} := \frac{B}{\text{SE}B}\ \stackrel{H_{0}}{\sim}\ F(1,n-2) \ .
$}
\ee
%
Under $H_{0}$, this satisfies an $\boldsymbol{F}$\textbf{--test
distribution} with $df_{1}=1$ and $df_{2}=n-2$ degrees of freedom;
cf. Sec.~\ref{sec:fverteil}.

\medskip
\noindent
\textbf{Test decision:} The rejection region for $H_{0}$ at 
significance level $\alpha$ is given by (right-sided test)
%
\be
t_{n} > f_{1,n-2;1-\alpha} \ .
\ee
%
With Eq.~(\ref{eq:pvalueright}), the $p$--value associated with a 
specific realisation $t_{n}$ of (\ref{eq:linregteststat}), which is
to be calculated from the $\boldsymbol{F}$\textbf{--test
distribution}, amounts 
to
%
\be
p = P(T_{n}>t_{n}|H_{0}) = 1-P(T_{n}\leq t_{n}|H_{0})
= 1-F\texttt{cdf}(0,t_{n},1,n-2) \ .
\ee
%

%------------------------------------------------------------------
\subsection[$t$--test for the regression 
coefficients]{$\boldsymbol{t}$--test for the regression 
coefficients}
%------------------------------------------------------------------
The second issue to be addressed in a systematic \textbf{regression 
analysis} within the \textbf{frequentist framework} is to test
statistically which of the regression coefficients in the
model~(\ref{eq:linregmod}) are significantly different from zero.
In the case of simple linear regression, though, the matter for the coefficient $\beta$ is settled already 
by the \textbf{$\boldsymbol{F}$--test} of the regression model just 
outlined, resp.~the \textbf{$\boldsymbol{t}$--test} for $\rho$ 
described in Sec.~\ref{subsec:correlttest}; see, e.g., Levin 
\textit{et al} (2010)~\ct[p~389f]{levetal2009}. In this sense, a 
further test of statistical significance is redundant in the case 
of simple linear regression. However, when extending the concept 
of \textbf{regression analysis} to the more involved case of
\textbf{multivariate data}, a quantitative approach frequently
employed in the research literature of the \textbf{Social Sciences}
and \textbf{Economics}, this question attains relevance in its own
right. In this context, the \textbf{linear stochastic regression
model} for the dependent variable $Y$ to be assessed is of the
general form (cf. Yule (1897)~\ct{yul1897})
%
\be
\lb{eq:multlinregmod}
\text{in}\ \boldsymbol{\Omega}: \quad
Y_{i} = \alpha + \beta_{1}x_{i1} + \ldots
+ \beta_{k}x_{ik} + \varepsilon_{i}
\qquad (i=1, \ldots,n) \ ,
\ee
%
containing a total of $k$ uncorrelated independent variables and 
$k+1$ regression coefficients, as well as a random error term. A 
\textbf{multiple linear regression model} to be estimated from data 
of a corresponding random sample from $\boldsymbol{\Omega}$ of 
size~$n$ thus entails $n-k-1$ degrees of freedom; cf. 
Hair \textit{et al} (2010)~\ct[p~176]{haietal2010}. In view of this 
prospect, we continue with our methodological considerations.

\medskip
\noindent
First of all, \textbf{unbiased maximum likelihood point estimators} 
for the regression coefficients $\alpha$ and $\beta$ in 
Eq.~(\ref{eq:linregmod}) are obtained from application to the data 
of Gau\ss' method of \textbf{minimising the sum of squared
residuals} (RSS) (cf. Gau\ss~(1809)~\ct{gau1809} and
Ch.~\ref{ch5}),
%
\[
\text{minimise}\left(\text{RSS}=\sum_{i=1}^{n}e_{i}^{2}\right) \ ,
\]
%
yielding solutions
%
\be
\lb{eq:linregab}
b = \frac{S_{Y}}{s_{X}}\,r
\qquad\text{and}\qquad
a=\bar{Y}-b\bar{x} \ .
\ee
%
The equation of the \textbf{best-fit simple linear regression
model} is thus given by
%
\be
\fbox{$\displaystyle
\lb{eq:linregmodelstand}
\hat{y}=\bar{Y}+\frac{S_{Y}}{s_{X}}\,r\,(x-\bar{x}) \ ,
$}
\ee
%
and can be employed for purposes of predicting values of~$Y$ from 
given values of $X$ in the empirical interval $[x_{(1)},x_{(n)}]$.

\medskip
\noindent
Next, the \textbf{standard errors} associated with the values of
the maximum likelihood point estimators $a$ and $b$ in 
Eq.~(\ref{eq:linregab}) are derived from the corresponding 
theoretical \textbf{sampling distributions} and amount to (cf.,
e.g., Hartung \textit{et al} (2005)~\ct[p~576ff]{haretal2005})
%
\bea
\lb{eq:seregcoeffa}
\text{SE}a & := & 
\sqrt{\frac{1}{n}+\frac{\bar{x}}{(n-1)s_{X}^{2}}}\,\text{SE}e \\
%
\lb{eq:seregcoeffb}
\text{SE}b & := & 
\frac{\text{SE}e}{\sqrt{n-1}\,s_{X}} \ ,
\eea
%
where the \textbf{standard error of the residuals} $e_{i}$ is
defined by
%
\be
\text{SE}e := 
\sqrt{\frac{{\displaystyle\sum_{i=1}^{n}
(Y_{i}-\hat{Y}_{i})^{2}}}{n-2}} \ .
\ee
%

\medskip
\noindent
We now describe the test procedure for the \textbf{regression 
coefficient}~$\beta$. To be tested is $H_{0}$ against $H_{1}$ in 
one of the alternative pairs of

\medskip
\noindent
\textbf{Hypotheses:} \hfill (test for differences)
%
\be
\begin{cases}
H_{0}: \beta=0
\quad\text{or}\quad
\beta \geq 0
\quad\text{or}\quad
\beta \leq 0 \\
H_{1}: \beta \neq 0
\quad\text{or}\quad
\beta < 0
\quad\text{or}\quad
\beta > 0
\end{cases} \ .
\ee
%
Dividing the \textbf{sample regression slope} $b$ by its
\textbf{standard error}~(\ref{eq:seregcoeffb}) yields the

\medskip
\noindent
\textbf{Test statistic:}
%
\be
\lb{eq:linregbteststat}
\fbox{$\displaystyle
T_{n} := \frac{b}{\text{SE}b} \ \stackrel{H_{0}}{\sim}\ t(n-2) \ ,
$}
\ee
%
which, under $H_{0}$, satisfies a $\boldsymbol{t}$\textbf{--test
distribution} with $df=n-2$ degrees of freedom; cf.
Sec.~\ref{sec:tverteil}.

\medskip
\noindent
\textbf{Test decision:} Depending on the kind of test to be 
performed, the rejection region for $H_{0}$ at significance level 
$\alpha$ is given by
%
\begin{center}
\begin{tabular}[h]{c|c|c|c}
 & & & \\
\textbf{Kind of test} & $\boldsymbol{H_{0}}$ &
$\boldsymbol{H_{1}}$ &
\textbf{Rejection region for} $\boldsymbol{H_{0}}$ \\
 & & & \\
\hline
 & & & \\
(a)~two-sided & $\beta = 0$ & $\beta \neq 0$ &
$|t_{n}|>t_{n-2;1-\alpha/2}$ \\
 & & & \\
\hline
 & & & \\
(b)~left-sided & $\beta \geq 0$ & $\beta < 0$ &
$t_{n}<t_{n-2;\alpha}=-t_{n-2;1-\alpha}$ \\
 & & & \\
\hline
 & & & \\
(c)~right-sided & $\beta \leq 0$ & $\beta > 0$ &
$t_{n}>t_{n-2;1-\alpha}$ \\
 & & &
\end{tabular}
\end{center}
%
$p$--values associated with realisations $t_{n}$ of 
(\ref{eq:linregbteststat}), which are to be calculated from the
$\boldsymbol{t}$\textbf{--test distribution}, can be obtained from 
Eqs.~(\ref{eq:pvaluetwo})--(\ref{eq:pvalueright}). We emphasise 
once more that for simple linear regression the test procedure 
just described is equivalent to the \textbf{correlation analysis}
of Sec.~\ref{subsec:correlttest}.

\medskip
\noindent
An analogous \textbf{$\boldsymbol{t}$--test} needs to be run to
check whether the \textbf{regression coefficient} $\alpha$ is
non-zero, too, using the ratio $\displaystyle\frac{a}{\text{SE}a}$
as a test statistic. However, in particular when the origin of $X$
is \textit{not} contained in the empirical interval 
$[x_{(1)},x_{(n)}]$, the null hypothesis $H_{0}: \alpha=0$ is a 
meaningless statement.

\medskip
\noindent
\underline{\R:}
\texttt{regMod <- lm(\textit{variable:y}~\texttildelow~\textit{variable:x})} \\
\texttt{summary(regMod)} \\
\underline{GDC:} mode \texttt{STAT} $\rightarrow$ \texttt{TESTS}
$\rightarrow$ \texttt{LinRegTTest\ldots} \\
\underline{SPSS:} Analyze $\rightarrow$ Regression
$\rightarrow$ Linear \ldots\ldots

\medskip
\noindent
\underline{Note:} Regrettably, SPSS provides no option for 
selecting between a one-sided and a two-sided $t$--test. The 
default setting is for a two-sided test. For the purpose of 
one-sided tests the $p$--value output of SPSS needs to be divided 
by $2$.

\medskip
\noindent
The extent to which the prerequisites of a regression analysis as 
stated in Eq.~(\ref{eq:linregassump}) are satisfied can be 
assessed by means of an \textbf{analysis of the residuals}:
%
\begin{itemize}

\item[(i)] for $n \geq 50$, \textbf{normality} of the distribution
of \textbf{residuals} $e_{i}$ ($i=1, \ldots, n$) can be checked by
means of a \textbf{Kolmogorov--Smirnov--test}; cf. 
Sec.~\ref{sec:onesampttest}; otherwise, when $n < 50$, resort to a 
consideration of the magnitudes of the \textbf{standardised
skewness and excess kurtosis measures}, Eqs.~(\ref{eq:g1g2ratios});

\item[(ii)] \textbf{homoscedasticity} of the $e_{i}$ ($i=1, \ldots, 
n$), i.e., whether or not they can be assumed to have constant 
variance, can be investigated qualitatively in terms of a
\textbf{scatter plot} that marks the standardised~$e_{i}$ (along
the vertical axis) against the corresponding predicted 
$Y$-values~$\hat{y}_{i}$ ($i=1, \ldots, n$) (along the horizontal 
axis). An elliptically shaped envelope of the cloud of data points 
thus obtained indicates that homoscedasticity applies.
\end{itemize}
%

\medskip
\noindent
Simple linear regression analysis can be easily modified to 
provide a tool to test bivariate empirical data 
$\{(x_{i},y_{i})\}_{i=1,\ldots,n}$ for positive metrically scaled 
statistical variables~$(X,Y)$ for an association in the form of a 
\textbf{Pareto distribution}; cf. Sec.~\ref{sec:paretodistr}. To
begin with, the original data is subjected to logarithmic 
transformations in order to obtain data for the \textbf{logarithmic 
quantities} $\ln(y_{i})$ resp.~$\ln(x_{i})$. Subsequently, a 
correlation analysis can be performed on the transformed data. 
Given there exists a functional relationship between the original 
$Y$ and $X$ of the form $y=Kx^{-(\gamma+1)}$, the logarithmic 
quantities are related by
%
\be
\ln(y) = \ln(K) - (\gamma+1)\times\ln(x) \ ,
\ee
%
i.e., one finds a \textit{straight line relationship} between 
$\ln(y)$ and $\ln(x)$ with negative slope equal to $-(\gamma+1)$.

\medskip
\noindent
To conclude this section, we like to draw the reader's attention 
to a remarkable statistical phenomenon that was discovered, and 
emphatically addressed, by the English empiricist 
\href{http://www-history.mcs.st-and.ac.uk/Biographies/Galton.html}{Sir
Francis Galton FRS (1822--1911)}, following years of intense 
research during the late $19^\mathrm{th}$ Century; see Galton 
(1886)~\ct{gal1886}, and also Kahneman (2011)~\ct[Ch.~17]{kah2011}.
\textbf{Regression toward the mean} is best demonstrated on the
basis of the standardised version of the best-fit simple linear 
regression model of Eq.~(\ref{eq:linregmodelstand}), namely
%
\be
\lb{eq:regmean}
\hat{z}_{Y} = rz_{X} \ .
\ee
%
For bivariate metrically scaled random sample data that exhibits a 
non-perfect positive correlation (i.e., $0 < r < 1$), one observes
that, on average, large (small) $z_{X}$-values (i.e., values that
are far from their mean; that are, perhaps, even outliers) pair
with smaller (larger) $z_{Y}$-values (i.e., values that are closer
to their mean; that are more mediocre). Since this phenomenon 
persists after the roles of $X$ and $Y$ in the regression model 
have been switched, this is clear evidence that \textbf{regression 
toward the mean} is a manifestation of \textbf{randomness}, and 
\textit{not} of \textbf{causality} (which requires an unambiguous 
temporal order between a cause and an effect). Incidently,
\textbf{regression toward the mean} ensures that many physical and
social processes cannot become unstable.

\medskip
\noindent
Ending this section we point out that in reality a lot of the
processes studied in the \textbf{Natural Sciences} and in the
\textbf{Social Sciences} prove to be of an inherently
\textbf{non-linear nature};
see e.g. Gleick (1987)~\ct{gle1987}, Penrose (2004)~\ct{pen2004},
and Smith (2007)~\ct{smi2007}. On the one hand, this increases the
level of complexity involved in the analysis of data, on the other, 
non-linear processes offer the reward of a plethora of interesting 
and intriguing (dynamical) phenomena.

%%%%%%%%%%%%%%%%%%%%%%%%%%%%%%%%%%%%%%%%%%%%%%%%%%%%%%%%%%%%%%%%%%%
\section[Rank correlation analysis]{Rank correlation analysis}
%%%%%%%%%%%%%%%%%%%%%%%%%%%%%%%%%%%%%%%%%%%%%%%%%%%%%%%%%%%%%%%%%%%
When the two-dimensional statistical variable~$(X,Y)$ is 
metrically scaled but may \textit{not} be assumed bivariate
normally distributed in the target
population~$\boldsymbol{\Omega}$, or when
$(X,Y)$ is ordinally scaled in the first place, the standard tool 
for testing for a statistical association between the 
components~$X$ and $Y$ is the parametric \textbf{rank correlation 
analysis} developed by the English psychologist and statistician
\href{http://en.wikipedia.org/wiki/Charles_Edward_Spearman}{Charles
Edward Spearman FRS (1863--1945)} in 1904~\ct{spe1904}. This 
approach, like the univariate test procedures of Mann and Whitney, 
Wilcoxon, and Kruskal and Wallis discussed in Ch.~\ref{ch12}, is 
again fundamentally rooted in the concept of \textbf{rank numbers} 
representing statistical data which possess a natural order, 
introduced in Sec.~\ref{sec:2Dord}.

\medskip
\noindent
Following the translation of the original data pairs into 
corresponding \textbf{rank number pairs},
%
\be
(x_{i},y_{i}) \mapsto [R(x_{i}),R(y_{i})] \qquad (i=1,\ldots,n) \ ,
\ee
%
the objective is to subject $H_{0}$ in the alternative sets of

\medskip
\noindent
\textbf{Hypotheses:} \hfill (test for association)
%
\be
\begin{cases}
H_{0}: \rho_{S} = 0
\quad\text{or}\quad
\rho_{S} \geq 0
\quad\text{or}\quad
\rho_{S} \leq 0 \\
H_{1}: \rho_{S} \neq 0
\quad\text{or}\quad
\rho_{S} < 0
\quad\text{or}\quad
\rho_{S} > 0
\end{cases} \ ,
\ee
%
with $\rho_{S}$ ($-1 \leq \rho_{S} \leq +1$) the \textbf{population 
rank correlation coefficient}, to a test of statistical 
significance at level $\alpha$. Provided the size of the random 
sample is such that $n \geq 30$ (see, e.g., Bortz 
(2005)~\ct[p~233]{bor2005}), by dividing the \textbf{sample rank 
correlation coefficient}~$r_{S}$ of Eq.~(\ref{eq:rankcorrelcoeff}) 
by its \textbf{standard error}
%
\be
\lb{eq:sers}
\text{SE}r_{S} := \sqrt{\frac{1-r_{S}^{2}}{n-2}}
\ee
%
derived from the theoretical \textbf{sampling distribution}
for~$r_{S}$, one obtains a suitable

\medskip
\noindent
\textbf{Test statistic:}
%
\be
\lb{eq:rankcorrelteststat}
\fbox{$\displaystyle
T_{n} := \frac{r_{S}}{\text{SE}r_{S}} \ \stackrel{H_{0}}{\approx}\ 
t(n-2) \ .
$}
\ee
%
Under $H_{0}$, this approximately satisfies a
$\boldsymbol{t}$\textbf{--test distribution} 
with $df=n-2$ degrees of freedom; cf. Sec.~\ref{sec:tverteil}.

\medskip
\noindent
\textbf{Test decision:} Depending on the kind of test to be 
performed, the rejection region for $H_{0}$ at significance level 
$\alpha$ is given by
%
\begin{center}
\begin{tabular}[h]{c|c|c|c}
 & & & \\
\textbf{Kind of test} & $\boldsymbol{H_{0}}$ &
$\boldsymbol{H_{1}}$ &
\textbf{Rejection region for} $\boldsymbol{H_{0}}$ \\
 & & & \\
\hline
 & & & \\
(a)~two-sided & $\rho_{S}=0$ & $\rho_{S} \neq 0$ &
$|t_{n}|>t_{n-2;1-\alpha/2}$ \\
 & & & \\
\hline
 & & & \\
(b)~left-sided & $\rho_{S} \geq 0$ & $\rho_{S}<0$ &
$t_{n}<t_{n-2;\alpha}=-t_{n-2;1-\alpha}$ \\
 & & & \\
\hline
 & & & \\
(c)~right-sided & $\rho_{S} \leq 0$ & $\rho_{S}>0$ &
$t_{n}>t_{n-2;1-\alpha}$ \\
 & & &
\end{tabular}
\end{center}
%
$p$--values associated with realisations $t_{n}$ of 
(\ref{eq:rankcorrelteststat}), which are to be calculated from the
$\boldsymbol{t}$\textbf{--test distribution}, can be obtained from 
Eqs.~(\ref{eq:pvaluetwo})--(\ref{eq:pvalueright}).

\medskip
\noindent
\underline{\R:} \texttt{cor.test(\textit{variable1},
\textit{variable2}, method = "spearman")}, \\
\texttt{cor.test(\textit{variable1}, \textit{variable2}, 
method = "spearman", alternative = "less")}, \\
\texttt{cor.test(\textit{variable1}, \textit{variable2}, 
method = "spearman", alternative = "greater")} \\
\underline{SPSS:} Analyze $\rightarrow$ Correlate
$\rightarrow$ Bivariate \ldots: Spearman

\medskip
\noindent
\textbf{Effect size:} The practical significance of the phenomenon investigated can be estimated directly from the absolute value of
the scale-invariant sample rank correlation coefficient~$r_{S}$
according to (cf. Cohen (1992)~\ct[Tab.~1]{coh1992})

\medskip
\noindent
\underline{\textbf{Rule of thumb:}}\\
$0.10 \leq |r_{S}| < 0.30$: small effect\\
$0.30 \leq |r_{S}| < 0.50$: medium effect\\
$0.50 \leq |r_{S}|$: large effect.


%%%%%%%%%%%%%%%%%%%%%%%%%%%%%%%%%%%%%%%%%%%%%%%%%%%%%%%%%%%%%%%%%%%
\section[$\chi^{2}$--test for
independence]{\href{https://www.youtube.com/watch?v=uo2kjAPkYXQ}{$\boldsymbol{\chi}^{2}$--test for independence}}
%%%%%%%%%%%%%%%%%%%%%%%%%%%%%%%%%%%%%%%%%%%%%%%%%%%%%%%%%%%%%%%%%%%
The non-parametric \textbf{$\boldsymbol{\chi}^{2}$--test for 
independence} constitutes the most generally applicable 
significance test for bivariate statistical associations. Due to 
its formal indifference to the scale level of measurement of the
two-dimensional statistical variable~$(X,Y)$ involved in an
investigation, it may be used for statistical analysis of any kind
of pairwise combinations between nominally, ordinally and
metrically scaled components. The advantage of generality of the
method is paid for at the price of a generally weaker test power.

\medskip
\noindent
Given qualitative and/or quantitative statistical variables~$X$ 
and $Y$ that take values in a spectrum of $k$ mutually exclusive 
categories $a_{1}, \ldots, a_{k}$ resp.~$l$ mutually exclusive 
categories $b_{1}, \ldots, b_{l}$, the intention is to subject 
$H_{0}$ in the pair of alternative

\medskip
\noindent
\textbf{Hypotheses:} \hfill (test for association)
%
\be
\begin{cases}
H_{0}: \text{There does not exist a statistical association 
between}\ X\ \text{and}\ Y\ \text{in}\ \boldsymbol{\Omega} \\
H_{1}: \text{There does exist a statistical association between}\ 
X\ \text{and}\ Y\ \text{in}\ \boldsymbol{\Omega}
\end{cases}
\ee
%
to a convenient empirical significance test at level $\alpha$.

\medskip
\noindent
A conceptual issue that requires special attention along the way
%in testing for association between variables
is the definition of a reasonable \textbf{zero point} on the
\textbf{scale of statistical dependence} of statistical variables
$X$ and $Y$ (which one aims to establish). This problem is solved
by recognising that a common feature of sample data for statistical 
variables of all scale levels of measurement is the information
residing in the distribution of (relative) frequencies over (all
possible combinations of) categories, and drawing an analogy to the
concept of stochastic independence of two events as expressed in
\textbf{Probability Theory} by Eq.~(\ref{eq:stochindep2}). In this
way, by definition, we refer to variables $X$ and $Y$ as being
mutually \textbf{statistically independent} provided that the
bivariate relative frequencies $h_{ij}$ of \textit{all}
combinations of categories $(a_{i},b_{j})$ are numerically equal to
the products of the univariate marginal relative frequencies
$h_{i+}$ of $a_{i}$ and $h_{+j}$ of $b_{j}$ (cf.
Sec.~\ref{sec:konttaf}), i.e.,
%
\be
h_{ij} = h_{i+}h_{+j} \ .
\ee
%
Translated into the language of random sample variables, 
viz.~introducing \textbf{sample observed frequencies}, this 
operational \textbf{independence condition} is re-expressed by 
$O_{ij} = E_{ij}$, where the $O_{ij}$ denote the bivariate
\textbf{observed frequencies} of the category combinations
$(a_{i},b_{j})$ in a \textbf{cross tabulation} underlying a
specific random sample of size $n$, and the quantities $E_{ij}$,
which are defined in terms of (i)~the univariate sum $O_{i+}$ of
observed frequencies in row $i$, see Eq.~(\ref{eq:margfreq1}),
(ii)~the univariate sum $O_{+j}$ of observed frequencies in column
$j$, see Eq.~(\ref{eq:margfreq2}), and (iii)~the sample size $n$ by 
$\displaystyle E_{ij}:=\frac{O_{i+}O_{+j}}{n}$, are interpreted as 
the \textbf{expected frequencies} of $(a_{i},b_{j})$, given that
$X$ and $Y$ are statistically independent. Expressing differences 
between observed and (under independence) expected frequencies via 
the \textbf{residuals} $O_{ij} -  E_{ij}$, the hypotheses may be 
reformulated as

\medskip
\noindent
\textbf{Hypotheses:} \hfill (test for association)
%
\be
\begin{cases}
H_{0}: O_{ij} - E_{ij} = 0 \qquad\text{for all}
\ i = 1, \ldots, k \ \text{and}
\ j = 1, \ldots, l \\
H_{1}: O_{ij} - E_{ij} \neq 0 \qquad\text{for at least one}
\ i \ \text{and}\ j
\end{cases} \ .
\ee
%

\medskip
\noindent
For the subsequent test procedure to be reliable, it is
\textit{very important (!)} that the empirical prere\-quisite
%
\be
E_{ij} \stackrel{!}{\geq} 5
\ee
%  
holds for all values of $i=1\ldots,k$ and $j=1,\ldots,l$,
such that one avoids the possibility of individual rescaled 
squared residuals $\displaystyle 
\frac{(O_{ij}-E_{ij})^{2}}{E_{ij}}$ becoming artificially 
magnified. The latter constitute the core of the

\medskip
\noindent
\textbf{Test statistic:}
%
\be
\lb{eq:chisqindepteststat}
\fbox{$\displaystyle
T_{n} := \sum_{i=1}^{k}\sum_{j=1}^{l}\frac{(O_{ij}-E_{ij}
)^{2}}{E_{ij}}
\ \stackrel{H_{0}}{\approx}\ \chi^{2}[(k-1)\times(l-1)] \ ,
$}
\ee
%
which, under $H_{0}$, approximately satisfies a 
$\boldsymbol{\chi^{2}}$\textbf{--test distribution} with $df=(k-1)
\times (l-1)$ degrees of freedom; cf. Sec.~\ref{sec:chi2verteil}.

\medskip
\noindent
\textbf{Test decision:} The rejection region for $H_{0}$ at 
significance level $\alpha$ is given by (right-sided test)
%
\be
t_{n}>\chi^{2}_{(k-1)\times(l-1);1-\alpha} \ .
\ee
%
By Eq.~(\ref{eq:pvalueright}), the $p$--value associated with a 
realisation $t_{n}$ of (\ref{eq:chisqindepteststat}), which is to
be calculated from the $\boldsymbol{\chi^{2}}$\textbf{--test
distribution}, amounts to
%
\be
p = P(T_{n}>t_{n}|H_{0}) = 1-P(T_{n}\leq t_{n}|H_{0})
= 1-\chi^{2}\texttt{cdf}\left(0,t_{n},(k-1)\times(l-1)\right) \ .
\ee
%

\medskip
\noindent
\underline{\R:} \texttt{chisq.test(\textit{row variable},
\textit{column variable})} \\
\underline{GDC:} mode \texttt{STAT} $\rightarrow$ \texttt{TESTS}
$\rightarrow$ \texttt{$\chi^{2}$-Test\ldots} \\
\underline{SPSS:} Analyze $\rightarrow$ Descriptive Statistics
$\rightarrow$ Crosstabs \ldots $\rightarrow$ Statistics \ldots:
Chi-square

\medskip
\noindent
The $\chi^{2}$--test for independence can establish the 
\textbf{existence} of a significant association in the joint
distribution of a two-dimensional statistical variable~$(X,Y)$. 
The \textbf{strength} of the association, on the other hand, 
may be measured in terms of \textbf{Cram\'{e}r's} $\boldsymbol{V}$
(Cram\'{e}r (1946)~\ct{cra1946}), which has a normalised range of 
values given by $0 \leq V \leq 1$; cf. Eq.~(\ref{eq:cramv}) and 
Sec.~\ref{sec:2Dnom}. Low values of $V$ in the case of significant 
associations between components $X$ and $Y$ typically indicate the 
statistical influence of additional \textbf{control variables}.

\medskip
\noindent
\underline{\R:} \texttt{assocstats(\textit{contingency table})}
(package: \texttt{vcd}, by Meyer \textit{et al}
(2017)~\ct{meyetal2017}) \\
\underline{SPSS:} Analyze $\rightarrow$ Descriptive Statistics
$\rightarrow$ Crosstabs \ldots $\rightarrow$ Statistics \ldots:
Phi and Cramer's V

\medskip
\noindent
\textbf{Effect size:} The practical significance of the phenomenon investigated can be estimated and interpreted by means of the
effect size measure~$w$ defined in Eq.~(\ref{eq:eschisq});
cf. Cohen (1992)~\ct[Tab.~1]{coh1992}.

%%%%%%%%%%%%%%%%%%%%%%%%%%%%%%%%%%%%%%%%%%%%%%%%%%%%%%%%%%%%%%%%%%%
%%%%%%%%%%%%%%%%%%%%%%%%%%%%%%%%%%%%%%%%%%%%%%%%%%%%%%%%%%%%%%%%%%%
