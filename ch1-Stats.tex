%%%%%%%%%%%%%%%%%%%%%%%%%%%%%%%%%%%%%%%%%%%%%%%%%%%%%%%%%%%%%%%%%%%
%  File name: ch1-Stats.tex
%  Title:
%  Version: 27.05.2019 (hve)
%%%%%%%%%%%%%%%%%%%%%%%%%%%%%%%%%%%%%%%%%%%%%%%%%%%%%%%%%%%%%%%%%%%
%%%%%%%%%%%%%%%%%%%%%%%%%%%%%%%%%%%%%%%%%%%%%%%%%%%%%%%%%%%%%%%%%%%
\chapter[Statistical variables]{\href{https://www.youtube.com/watch?v=DXkHcaiRcd4}{Statistical variables}}
\lb{ch1}
%%%%%%%%%%%%%%%%%%%%%%%%%%%%%%%%%%%%%%%%%%%%%%%%%%%%%%%%%%%%%%%%%%%
A central task of an empirical scientific discipline is the
\textbf{observation} of a  finite set of characteristic
\textbf{variable features} of a given  \textbf{system of objects}
chosen for study. The hope is to be able to recognise in a sea of
data, typically guided by \textbf{randomness}, meaningful patterns
and regularities that provide evidence for possible
\textbf{associations}, or, stronger still, \textbf{causal
relationships} between these variable features. Based on a 
combination of \textbf{inductive} and \textbf{deductive methods of
data analysis}, one aims at gaining insights of a qualitative
and/or quantitative nature into the intricate and often complex 
interdependencies of such variable features for the purpose of 
(i)~obtaining explanations for phenomena that have been observed, 
and (ii)~making predictions which, subsequently, can be tested. 
The acceptance of the validity of a particular empirical 
scientific framework generally increases with the number of 
successful \textbf{replications} of its predictions.\footnote{A 
particularly sceptical view on the ability of making reliable
predictions in certain empirical scientific disciplines is voiced 
in Taleb (2007)~\ct[pp~135--211]{tal2007}.} It is the interplay of 
observation, experimentation and theoretical modelling, 
systematically coupled to one another by a number of 
feedback loops, which gives rise to progress in learning and 
understanding in all empirical scientific activities. This 
procedure is referred to as the \textbf{scientific method}.

\medskip
\noindent
More specifically, the general intention of empirical 
scientific activities is to modify or strengthen the
\textbf{theoretical foundations} of an empirical scientific
discipline by means of observational and/or experimental
\textbf{testing} of sets of \textbf{hypotheses}; see
Ch.~\ref{ch11}. This is generally 
achieved by employing the quantitative--empirical techniques that 
have been developed in \textbf{Statistics}, in particular in the 
course of the $20^\mathrm{th}$ Century. At the heart of these 
techniques is the concept of a \textbf{statistical variable} $X$ as 
an entity which represents a single common aspect of the system of 
objects selected for analysis --- the \textbf{target population} 
$\boldsymbol{\Omega}$ of a \textbf{statistical investigation}. In
the ideal case, a variable entertains a one-to-one correspondence
with an \textbf{observable}, and thus is directly amenable to
\textbf{measurement}. In the \textbf{Social Sciences},
\textbf{Humanities}, and \textbf{Economics}, however, one needs to
carefully distinguish between \textbf{manifest variables}
corresponding to observables on the one-hand side, and
\textbf{latent variables} representing in general unobservable
``social constructs'' on the other. It is this latter kind of
variables which is commonplace in the fields mentioned. Hence, it
becomes an unavoidable task to thoroughly address the issue of a
reliable, valid and objective \textbf{operationalisation} of any
given latent variable one has identified as providing essential
information on the objects under investigation. A standard approach
to dealing with the important matter of rendering latent variables
measurable is reviewed in Ch.~\ref{ch9}.

\medskip
\noindent
In \textbf{Statistics}, it has proven useful to classify variables
on the basis of their intrinsic information content into one of
three hierachically ordered categories, referred to as
\textbf{scale levels of measurement}; cf. Stevens
(1946)~\ct{ste1946}. We provide the definition of these scale
levels next.

%%%%%%%%%%%%%%%%%%%%%%%%%%%%%%%%%%%%%%%%%%%%%%%%%%%%%%%%%%%%%%%%%%%
\section[Scale levels of measurement]{Scale levels of measurement}
\lb{merkskal}
%%%%%%%%%%%%%%%%%%%%%%%%%%%%%%%%%%%%%%%%%%%%%%%%%%%%%%%%%%%%%%%%%%%
\underline{\textbf{Def.:}} Let $X$ be a one-dimensional
\textbf{statistical variable} with $k \in \mathbb{N}$ (countably
many) resp.\ $k \in \mathbb{R}$ (uncountably many) possible
\textbf{values}, \textbf{attributes}, or \textbf{categories}
$a_{j}$ ($j=1,\ldots,k$). Statistical variables
are classified as belonging into one of three hierachically ordered 
\textbf{scale levels of measurement}. This is done on the basis of 
three criteria for distinguishing information contained in the 
values of actual \textbf{data} for these variables. One thus
defines:

\begin{itemize}

\item \textbf{Metrically scaled variables} $X$ \hfill
\textbf{(quantitative/numerical)}\\
Possible values can be distinguished by
%
\begin{itemize}
\item[(i)] their \textit{names}, $a_{i} \neq a_{j}$,
\item[(ii)] they allow for a \textit{natural rank order}, $a_{i} < 
a_{j}$, and
\item[(iii)] \textit{distances} between them, $a_{i}-a_{j}$,
are uniquely determined.
\end{itemize}
%
	\begin{itemize}
	\item \textbf{Ratio scale}: $X$ has an \textit{absolute zero
	point} and otherwise only non-negative values;	analysis of both 
	differences	$a_{i}-a_{j}$ and ratios $a_{i}/a_{j}$ is meaningful.
	
	\underline{Examples:} body height, monthly net income, \ldots.
	
	\item \textbf{Interval scale}: $X$ has no \textit{absolute
	zero point}; only differences $a_{i}-a_{j}$
	are meaningful.
	
	\underline{Examples:} year of birth, temperature in centigrades,
	Likert scales (cf. Ch.~\ref{ch9}), \ldots.
	\end{itemize}
	
Note that the values obtained for a metrically scaled variable 
(e.g. in a survey) always constitute definite numerical 
multiples of a specific \textbf{unit of measurement}.

\item \textbf{Ordinally scaled variables} $X$ \hfill
\textbf{(qualitative/categorical)}\\
Possible values, attributes, or categories can be distinguished by
%
\begin{itemize}
\item[(i)] their \textit{names}, $a_{i} \neq a_{j}$, and
\item[(ii)] they allow for a \textit{natural rank order}, $a_{i} 
< a_{j}$.
\end{itemize}
%
\underline{Examples:} Likert item rating scales (cf. 
Ch.~\ref{ch9}), grading of commodities, \ldots.

\item \textbf{Nominally scaled variables} $X$ \hfill
\textbf{(qualitative/categorical)}\\
Possible values, attributes, or categories can be distinguished 
only by
%
\begin{itemize}
\item[(i)] their \textit{names}, $a_{i} \neq a_{j}$.
\end{itemize}
%
\underline{Examples:} first name, location of birth, \ldots.

\end{itemize}

\noindent
\underline{\textbf{Remark:}} As we will see later in Ch.~\ref{ch12}
and~\ref{ch13}, the applicability of specific methods of
\textbf{statistical data analysis} crucially depends on the
\textbf{scale level of measurement} of the variables involved in
the respective procedures. Metrically scaled data offers the
largest variety of powerful methods for this purpose!

%%%%%%%%%%%%%%%%%%%%%%%%%%%%%%%%%%%%%%%%%%%%%%%%%%%%%%%%%%%%%%%%%%%
\section[Raw data sets and data matrices]{Raw data sets and data
matrices}
\lb{urlist}
%%%%%%%%%%%%%%%%%%%%%%%%%%%%%%%%%%%%%%%%%%%%%%%%%%%%%%%%%%%%%%%%%%%
To set the stage for subsequent considerations, we here introduce
some formal representations of entities which assume central roles
in statistical data analyses.

\medskip
\noindent
Let $\boldsymbol{\Omega}$ denote the \textbf{target population} of 
study objects of interest (e.g., human individuals forming a 
particular social system) relating to some \textbf{statistical 
investigation}. This set~$\boldsymbol{\Omega}$ shall comprise a 
total of $N \in \mathbb{N}$ \textbf{statistical units}, i.e., its 
size be $|\boldsymbol{\Omega}|=N$.

\medskip
\noindent
Suppose one intends to determine the \textbf{frequency
distributional properties} in $\boldsymbol{\Omega}$ of a portfolio
of $m \in \mathbb{N}$ \textbf{statistical variables} $X$, $Y$,
\ldots, and $Z$, with \textbf{spectra of values} $a_{1}, a_{2},
\ldots, a_{k}$, $b_{1}, b_{2}, \ldots, b_{l}$, \ldots, and $c_{1},
c_{2}, \ldots, c_{p}$, respectively ($k,l,p \in \mathbb{N}$). A
\textbf{survey} typically obtains from~$\boldsymbol{\Omega}$ a
\textbf{statistical sample} $\boldsymbol{S_{\Omega}}$ of size 
$|\boldsymbol{S_{\Omega}}|=n$ ($n \in \mathbb{N}$, $n < N$), 
unless one is given the rare opportunity to conduct a proper
\textbf{census} on $\boldsymbol{\Omega}$ (in which case $n=N$).
The \textbf{data} thus generated consists of \textbf{observed 
values}~$\{x_{i}\}_{i=1,\ldots,n}$, 
$\{y_{i}\}_{i=1,\ldots,n}$, \ldots, and $\{z_{i}\}_{i=1,\ldots,n}$.
It constitutes the \textbf{raw data set} $\{(x_{i}, y_{i}, \ldots, 
z_{i})\}_{i=1,\ldots,n}$ of a statistical investigation and may be 
conveniently assembled in the form of an $\boldsymbol{(n \times 
m)}$ \textbf{data matrix} $\boldsymbol{X}$ given by
%
\begin{center}
\begin{tabular}[h]{|c||c|c|c|c|}
\hline
 & & & & \\
\textbf{sampling} & \textbf{variable} & \textbf{variable} &
\ldots & \textbf{variable} \\
\textbf{unit} & $X$ & $Y$ & & $Z$ \\
 & & & & \\
\hline\hline
 & & & & \\
$1$ & $x_{1}=a_{5}$ & $y_{1}=b_{9}$ & \ldots & $z_{1}=c_{3}$ \\
 & & & & \\
\hline
 & & & & \\
$2$ & $x_{2}=a_{2}$ & $y_{2}=b_{12}$ & \ldots & $z_{2}=c_{8}$ \\
 & & & & \\
\hline
 & & & & \\
\vdots & \vdots & \vdots & \vdots & \vdots \\
 & & & & \\
\hline
 & & & & \\
$n$ & $x_{n}=a_{8}$ & $y_{n}=b_{9}$ & \ldots & $z_{n}=c_{15}$ \\
 & & & & \\
\hline
\end{tabular}
\end{center}
%
To systematically record the information obtained from measuring 
the values of a portfolio of statistical variables in a 
statistical sample $\boldsymbol{S_{\Omega}}$, in the 
$\boldsymbol{(n \times m)}$~\textbf{data matrix} 
$\boldsymbol{X}$ every one of the $n$~\textbf{sampling units} 
investigated is assigned a particular \textit{row}, while every one 
of the $m$~\textbf{statistical variables} measured is assigned a 
particular \textit{column}. In the following, $X_{ij}$ denotes the 
data entry in the $i$th row ($i=1,\ldots,n$) and the $j$th column 
($i=1,\ldots,m$) of $\boldsymbol{X}$. To clarify standard 
terminology used in \textbf{Statistics}, a \textbf{raw data set} is 
referred to as
%
\begin{itemize}
\item[(i)] \textbf{univariate}, when $m=1$,
\item[(ii)] \textbf{bivariate}, when $m=2$, and
\item[(iii)] \textbf{multivariate}, when $m \geq 3$.
\end{itemize}
%
According to Hair \textit{et al} (2010) 
\ct[pp~102, 175]{haietal2010}, a rough rule of thumb concerning an 
adequate \textbf{sample size} $|\boldsymbol{S_{\Omega}}|=n$ for 
\textbf{multivariate data analysis} is given 
by
%
\be
\lb{eq:samplesize}
n \geq 10m \ .
\ee
%
Considerations of \textbf{statistical power} of particular methods
of data analysis lead to more refined recommendations; cf. 
Sec.~\ref{sec:testgen}.

\medskip
\noindent
\textbf{``Big data''} scenarios apply when $n, m 
\gg 1$ (i.e., when $n$ is typically on the order of $10^{4}$, or
very much larger still, and $m$ is on the order of $10^{2}$, or
larger).

\medskip
\noindent
In general, an $(n \times m)$ data matrix $\boldsymbol{X}$ is the 
starting point for the application of a \textbf{statistical
software package} such as \R{}, SPSS, GNU PSPP, or other for the
purpose of systematic data analysis. When the sample comprises
exclusively \textbf{metrically scaled data}, the data matrix is
real-valued, i.e.,
%
\be
\lb{eq:metrdatamatrix}
\boldsymbol{X} \in \mathbb{R}^{n \times m} \ ;
\ee
%
cf. Ref.~\ct[Sec.~2.1]{hve2009}. Then the information contained in 
$\boldsymbol{X}$ uniquely positions a collection of $n$ sampling 
units according to $m$ quantitative characteristic variable 
features in (a subset of) an $m$-dimensional \textbf{Euclidian
space}~$\mathbb{R}^{m}$.

\medskip
\noindent
\underline{\R:} \texttt{datMat <- data.frame(x = c($x_{1}$,\ldots,$x_{n}$),
y = c($y_{1}$,\ldots,$y_{n}$), \ldots, \\
z = c($z_{1}$,\ldots,$z_{n}$))} 

\vspace{5mm}
\noindent
We next turn to describe phenomenologically the \textbf{univariate 
frequency distribution} of a single one-dimensional statistical 
variable $X$ in a specific statistical sample 
$\boldsymbol{S_{\Omega}}$ of size $n$,  drawn in the context of a 
survey from some target population of study objects 
$\boldsymbol{\Omega}$ of size $N$.

%%%%%%%%%%%%%%%%%%%%%%%%%%%%%%%%%%%%%%%%%%%%%%%%%%%%%%%%%%%%%%%%%%%
%%%%%%%%%%%%%%%%%%%%%%%%%%%%%%%%%%%%%%%%%%%%%%%%%%%%%%%%%%%%%%%%%%%
