%%%%%%%%%%%%%%%%%%%%%%%%%%%%%%%%%%%%%%%%%%%%%%%%%%%%%%%%%%%%%%%%%%%
%  File name: ch9-WSt.tex
%  Title:
%  Version: 04.06.2019 (hve)
%%%%%%%%%%%%%%%%%%%%%%%%%%%%%%%%%%%%%%%%%%%%%%%%%%%%%%%%%%%%%%%%%%%
%%%%%%%%%%%%%%%%%%%%%%%%%%%%%%%%%%%%%%%%%%%%%%%%%%%%%%%%%%%%%%%%%%%
\chapter[Likert's scaling method of summated item 
ratings]{\href{https://www.youtube.com/watch?v=_IJJ_D6UOaQ}{Operationalisation 
of latent variables: Likert's scaling method of summated item
ratings}}
\lb{ch9}
%%%%%%%%%%%%%%%%%%%%%%%%%%%%%%%%%%%%%%%%%%%%%%%%%%%%%%%%%%%%%%%%%%%
%%%%%%%%%%%%%%%%%%%%%%%%%%%%%%%%%%%%%%%%%%%%%%%%%%%%%%%%%%%%%%%%%%%
%\section[Likert's scaling method of summated item 
%ratings]{Likert's scaling method of summated item ratings}
%\lb{sec:likertscale}
%%%%%%%%%%%%%%%%%%%%%%%%%%%%%%%%%%%%%%%%%%%%%%%%%%%%%%%%%%%%%%%%%%%
A sound \textbf{operationalisation} of ones's portfolio of
\textbf{statistical variables} in quantitative--empirical research
is key to a successful and effective application of
\textbf{statistical methods of data analysis}, particularly in the
\textbf{Social Sciences} and \textbf{Humanities}. The most
frequently practiced method to date for operationalising
\textbf{latent variables} (such as unobservable ``social
constructs'') is due to the US-American psychologist 
\href{http://en.wikipedia.org/wiki/Rensis_Likert}{Rensis Likert's 
(1903--1981)}. In his 1932 paper \ct{lik1932}, which completed his 
thesis work for a Ph.D., he expressed the idea that \textbf{latent 
statistical variables}~$X_{L}$, when they may be perceived as 
\textit{one-dimensional} in nature, can be rendered measurable in a 
\textit{quasi-metrical} fashion by means of the \textbf{summated 
ratings} over an extended set of suitable and observable
\textbf{indicator items} $X_{i}$ ($i=1,2,\ldots$), which, in order
to ensure effectiveness, ought to be (i)~\textit{highly
interdependent} and possess (ii)~\textit{high discriminatory
power}. Such indicator items are often formulated as specific
statements relating to the theoretical concept a particular
one-dimensional latent variable~$X_{L}$ is supposed to capture,
with respect to which test persons need to express their subjective
level of agreement or, in different settings, indicate a specific
subjective degree of intensity. A typical \textbf{item rating
scale} for the indicator items~$X_{i}$, 
providing the necessary item ratings, is given for instance by the 
5--level ordinally ranked attributes of agreement\\[-5mm]
%
\begin{itemize}
\item[1:] strongly disagree/strongly unfavourable\\[-5mm]
\item[2:] disagree/unfavourable\\[-5mm]
\item[3:] undecided\\[-5mm]
\item[4:] agree/favourable\\[-5mm]
\item[5:] strongly agree/strongly favourable.\\[-5mm]
\end{itemize}
%
In the research literature, one also encounters 7--level or 
10--level item rating scales, which offer more flexibility. Note 
that it is \textit{assumed} fom the outset that the items~$X_{i}$, 
and thus their ratings, can be treated as \textbf{additive}, so
that the conceptual principles of Sec.~\ref{sec:sumvar} relating to 
sums of random variables can be relied upon. When forming the sum 
over the ratings of all the indicator items one selected, it is 
essential to carefully pay attention to the \textbf{polarity} of
the items involved. For the resultant \textbf{total sum} 
${\displaystyle\sum_{i}X_{i}}$ to be consistent, the polarity of 
all items used needs to be uniform.\footnote{For a questionnaire, 
however, it is strongly recommended to include also indicator 
items of reversed polarity. This will improve the overall 
construct validity of the measurement tool.}

\medskip
\noindent
The construction of a consistent and coherent \textbf{Likert scale} 
for a one-dimensional latent statistical variable~$X_{L}$ involves 
four basic steps (see, e.g., Trochim (2006)~\ct{tro2006}):\\[-5mm]
%
\begin{itemize}
\item[(i)] the compilation of an initial list of 80 to 100 
potential \textbf{indicator items} $X_{i}$ for the one-dimensional 
latent variable of interest,\\[-5mm]

\item[(ii)] the draw of a \textbf{gauge random sample} from the 
target population~$\boldsymbol{\Omega}$,\\[-5mm]

\item[(iii)] the computation of the \textbf{total sum} 
${\displaystyle\sum_{i}X_{i}}$ of item ratings, and, most 
importantly, \\[-5mm]

\item[(iv)] the performance of an \textbf{item analysis} based on
the sample data and the associated total sum 
${\displaystyle\sum_{i}X_{i}}$ of item ratings.\\[-5mm]
\end{itemize}
%
The item analysis, in particular, consists of the consequential 
application of two exclusion criteria, which aim at establishing 
the scientific quality of the final \textbf{Likert scale}. Items
are being discarded from the list when either\\[-6mm]
%
\begin{itemize}
\item[(a)] they show a weak \textbf{item-to-total 
correlation} with the total sum ${\displaystyle\sum_{i}X_{i}}$ (a 
rule of thumb is to exclude items with correlations less than 
$0.5$), or\\[-6mm]
  
\item[(b)] it is possible to increase the value of
\textbf{Cronbach's}\footnote{Named 
after the US-American educational psychologist
\href{http://en.wikipedia.org/wiki/Lee_Cronbach}{Lee Joseph 
Cronbach (1916--2001)}. The range of the normalised real-valued
$\alpha$--coefficient is the interval $[0,1]$.} 
$\boldsymbol{\alpha}$\textbf{--coefficient} (see Cronbach (1951) 
\ct{cro1951}), a measure of the scale's \textbf{internal
consistency reliability}, by excluding a particular item from the
list (the objective being to attain $\alpha$-values greater than
$0.8$).\\[-6mm]
\end{itemize}
%
For a set of $m \in \mathbb{N}$ indicator items $X_{i}$, 
Cronbach's $\alpha$--coefficient is defined by
%
\be
\lb{eq:cronbachalpha}
\fbox{$\displaystyle
\alpha := \left(\frac{m}{m-1}\right)\left(1 - 
\frac{{\displaystyle\sum_{i=1}^{m}S_{i}^{2}}}{S_\mathrm{total}^{2}}
\right) \ ,
$}
\ee
%
where $S_{i}^{2}$ denotes the sample variance associated with the 
$i$th indicator item,
%which here is {\em assumed\/} to be measured on a 5--point 
%interval scale,
and $S_\mathrm{total}^{2}$ is the sample variance of the total sum 
${\displaystyle\sum_{i}X_{i}}$.

\medskip
\noindent
\underline{\R:} \texttt{alpha({\it items\/})} (package:
\texttt{psych}, by Revelle (2019)~\ct{rev2019}) \\
\underline{SPSS:} Analyze $\rightarrow$ Scale $\rightarrow$ 
Reliability Analysis \ldots (Model: Alpha) $\rightarrow$ 
Statistics \ldots: Scale if item deleted

\medskip
\noindent
The outcome of the item analysis is a drastic reduction of the 
initial list to a set of just $k \in \mathbb{N}$ indicator items 
$X_{i}$ ($i=1,\ldots,k$) of high discriminatory power, where $k$ 
is typically in the range of $10$ to $15$.\footnote{However, in 
many research papers one finds Likert scales with a minimum of 
just four indicator items.} The associated \textbf{total sum}
%
\be 
X_{L} := \sum_{i=1}^{k}X_{i}
\ee
%
thus operationalises the one-dimensional latent statistical 
variable~$X_{L}$ in a quasi-metrical fashion, since it is to be 
measured on an \textbf{interval scale} with a \textit{discrete} 
spectrum of values given (for a 5--level item rating scale) by
%
\be
X_{L} \mapsto \sum_{i=1}^{k}x_{i} \in \left[1k,5k\right] \ .
\ee
%
The structure of a finalised discrete $k$-indicator-item Likert 
scale for some one-dimensional latent statistical variable~$X_{L}$ 
with an equidistant graphical 5--level item rating scale is 
displayed in Tab.~\ref{tab:likert}.
%
\begin{table}

\underline{\textbf{One-dimensional latent statistical
variable}~$\boldsymbol{X_{L}}$:}
%
\begin{center}
\begin{tabular}[!htb]{cccccccc}
% & \textbf{strongly disagree} & & & & \textbf{strongly agree} \\ \\
$\bullet$\ \textbf{Item} $X_{1}$: &
strongly disagree & $\bigcirc$\quad\quad\mbox{} 
& $\bigcirc$\quad\quad\mbox{} & $\bigcirc$\quad\quad\mbox{} & 
$\bigcirc$\quad\quad\mbox{} & $\bigcirc$ & strongly agree \\ \\
$\bullet$\ \textbf{Item} $X_{2}$: &
strongly disagree & $\bigcirc$\quad\quad\mbox{} 
& $\bigcirc$\quad\quad\mbox{} & $\bigcirc$\quad\quad\mbox{} & 
$\bigcirc$\quad\quad\mbox{} & $\bigcirc$ & strongly agree \\ \\
\vdots & \vdots & \vdots\quad\quad\mbox{} & 
\vdots\quad\quad\mbox{} & \vdots\quad\quad\mbox{} & 
\vdots\quad\quad\mbox{} & \vdots & \vdots\\ \\
$\bullet$\ \textbf{Item} $X_{k}$: & 
strongly disagree & $\bigcirc$\quad\quad\mbox{} 
& $\bigcirc$\quad\quad\mbox{} & $\bigcirc$\quad\quad\mbox{} & 
$\bigcirc$\quad\quad\mbox{} & $\bigcirc$ & strongly agree
\end{tabular}
\end{center}
\caption{Structure of a discrete $k$-indicator-item Likert scale 
for some one-dimensional latent statistical variable~$X_{L}$, 
based on a visualised equidistant 5--level item rating scale.}
\lb{tab:likert}
\end{table}
%

\medskip
\noindent
Likert's scaling method of aggregating information from a set of  
$k$ highly interdependent ordinally scaled items to form an 
effectively quasi-metrical, one-dimensional total sum 
${\displaystyle X_{L}=\sum_{i}X_{i}}$ draws its legitimisation to 
a large extent from a generalised version of the \textbf{central 
limit theorem} (cf. Sec.~\ref{sec:zentrgrenz}), wherein the 
precondition of mutually stochastically independent variables 
contributing to the sum is relaxed. In practice it is found that 
for many cases of interest in the samples one has available for 
research the total sum  ${\displaystyle X_{L}=\sum_{i}X_{i}}$ is 
normally distributed in to a very good approximation. 
Nevertheless, the normality property of Likert scale data needs to 
be established on a case-by-case basis. The main shortcoming of 
Likert's approach is its dependency of the gauging process of the 
scale on the target population.

\medskip
\noindent
In the \textbf{Social Sciences} there is available a broad variety
of operationalisation procedures alternative to the discrete
\textbf{Likert scale}. We restrict ourselves here to mention but
one example, namely the \textit{continuous} psychometric
\textbf{visual analogue scale (VAS)} developed by Hayes and
Paterson (1921)~\ct{haypat1921} and by Freyd (1923)~\ct{fre1923}.
Further measurement scales for latent statistical variables can be 
obtained from the websites 
\href{http://zis.gesis.org/ZisApplication/}{\texttt{zis.gesis.org}},
German Social Sciences measurement scales (ZIS), 
and \href{http://www.ssrn.com}{\texttt{ssrn.com}}, Social Science 
Research Network (SSRN). On a historical note: one of the first 
systematically designed \textbf{questionnaires} as a measurement
tool for collecting socio-economic data (from workers on strike at
the time in Britain) was published by the Statistical Society of 
London in 1838; see Ref.~\ct{ssl1838}.

%%%%%%%%%%%%%%%%%%%%%%%%%%%%%%%%%%%%%%%%%%%%%%%%%%%%%%%%%%%%%%%%%%%
%%%%%%%%%%%%%%%%%%%%%%%%%%%%%%%%%%%%%%%%%%%%%%%%%%%%%%%%%%%%%%%%%%%
