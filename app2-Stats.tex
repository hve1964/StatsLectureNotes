%%%%%%%%%%%%%%%%%%%%%%%%%%%%%%%%%%%%%%%%%%%%%%%%%%%%%%%%%%%%%%%%%%%
%  File name: app2-Stats.tex
%  Title:
%  Version: 17.08.2019
%%%%%%%%%%%%%%%%%%%%%%%%%%%%%%%%%%%%%%%%%%%%%%%%%%%%%%%%%%%%%%%%%%%
%%%%%%%%%%%%%%%%%%%%%%%%%%%%%%%%%%%%%%%%%%%%%%%%%%%%%%%%%%%%%%%%%%%
%\appendix
\chapter[Distance measures in Statistics]{Distance measures in 
Statistics}
\lb{app2}
%%%%%%%%%%%%%%%%%%%%%%%%%%%%%%%%%%%%%%%%%%%%%%%%%%%%%%%%%%%%%%%%%%%
{\bf Statistics} employs a number of different measures of {\bf 
distance} $d_{ij}$ to quantify the separation in an $m$--D space 
of metrically scaled statistical variables $X, Y, \ldots, Z$ of
two statistical units $i$ and $j$ ($i, j = 1, \ldots, n$). Note 
that, by construction, these measures $d_{ij}$ exhibit the 
properties $d_{ij} \geq 0$, $d_{ij} = d_{ji}$ and $d_{ii} = 0$. In 
the following, $X_{ik}$ is the entry of the data matrix 
$\boldsymbol{X} \in \mathbb{R}^{n \times m}$ 
relating to the $i$th statistical unit and the $k$th statistical 
variable, etc. The $d_{ij}$ define the elements of a
$\boldsymbol{(n \times n)}$ {\bf proximity matrix} $\boldsymbol{D} 
\in \mathbb{R}^{n \times n}$.

%------------------------------------------------------------------
\subsection*{Euclidian distance \hfill (dimensionful)}
%------------------------------------------------------------------
This most straightforward, dimensionful distance measure is named 
after the ancient Greek (?) mathematician 
\href{http://www-history.mcs.st-and.ac.uk/Biographies/Euclid.html}{Euclid of Alexandria (ca.~325BC--ca.~265BC)}. It is defined by
%
\be
\lb{eq:eucliddist}
\fbox{$\displaystyle
d_{ij}^{E} := 
\sqrt{\sum_{k=1}^{m}\sum_{l=1}^{m}(X_{ik}-X_{jk})\delta_{kl}
(X_{il}-X_{jl})} \ ,
$}
\ee
%
where $\delta_{kl}$ denotes the elements of the unit matrix 
$\boldsymbol{1} \in \mathbb{R}^{m \times m}$; cf. 
Ref.~\ct[Eq.~(2.2)]{hve2009}.

%------------------------------------------------------------------
\subsection*{Mahalanobis distance \hfill (dimensionless)}
%------------------------------------------------------------------
A more sophisticated, {\bf scale-invariant} distance measure in 
{\bf Statistics} was devised by the Indian applied statistician 
\href{http://www-history.mcs.st-and.ac.uk/Biographies/Mahalanobis.html}{Prasanta Chandra Mahalanobis (1893--1972)}; cf. Mahalanobis 
(1936)~\ct{mah1936}. It is defined by
%
\be
\lb{eq:mahaladist}
\fbox{$\displaystyle
d_{ij}^{M} := 
\sqrt{\sum_{k=1}^{m}\sum_{l=1}^{m}(X_{ik}-X_{jk})(S^{2})^{-1}_{kl}
(X_{il}-X_{jl})} \ ,
$}
\ee
%
where $(S^{2})^{-1}_{kl}$ denotes the elements of the inverse
covariance matrix $(\boldsymbol{S^{2}})^{-1} \in
\mathbb{R}^{m \times m}$ relating to $X, Y, \ldots, Z$; cf.
Sec.~\ref{subsec:covar}. The Mahalanobis distance thus accounts for
inter-variable correlations and so eliminates a potential source of
bias.

\medskip
\noindent
\underline{\R:} \texttt{mahalanobis(\textit{data matrix})}

%%%%%%%%%%%%%%%%%%%%%%%%%%%%%%%%%%%%%%%%%%%%%%%%%%%%%%%%%%%%%%%%%%%
%%%%%%%%%%%%%%%%%%%%%%%%%%%%%%%%%%%%%%%%%%%%%%%%%%%%%%%%%%%%%%%%%%%
