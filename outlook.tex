%%%%%%%%%%%%%%%%%%%%%%%%%%%%%%%%%%%%%%%%%%%%%%%%%%%%%%%%%%%%%%%%%%%
%  File name: outlook.tex
%  Version: 16.08.2019 (hve)
%%%%%%%%%%%%%%%%%%%%%%%%%%%%%%%%%%%%%%%%%%%%%%%%%%%%%%%%%%%%%%%%%%%
\addcontentsline{toc}{chapter}{Outlook}
%%%%%%%%%%%%%%%%%%%%%%%%%%%%%%%%%%%%%%%%%%%%%%%%%%%%%%%%%%%%%%%%%%%
\chapter*{Outlook}
%%%%%%%%%%%%%%%%%%%%%%%%%%%%%%%%%%%%%%%%%%%%%%%%%%%%%%%%%%%%%%%%%%%
Our discussion on the foundations and the application of
statistical methods of data analysis ends here. We have focused on
the description of uni- and bivariate data sets and making
inferences from corresponding random samples within the frequentist
approach to Probability Theory. At this stage, the attentive
reader should feel well-equipped for confronting problems
concerning more complex, multivariate data sets, and adequate
methods for tackling them by statistical means. Many modules at the
Master degree level review a broad spectrum of advanced topics such
as multiple linear regression, generalised linear models, principal
component analysis, or cluster analysis, which in turn relate to
computational techniques presently employed in the context of
machine learning. The ambitious reader might even think of getting
involved with proper research and work towards a Ph.D. degree in an
empirical scientific discipline. To gain additional data analytical
flexibility, and to increase chances on obtaining transparent
and satisfactory research results, it is strongly recommended to
consult the conceptually compelling inductive Bayes--Laplace
approach to statistical inference. In order to leave behind the
methodological shortcomings uncovered by the recent replication
crisis (cf. Refs.~\ct{eco2013} and~\ct{nuz2014}), strict adherence
to accepted scientific standards cannot be compromised with. 

\medskip
\noindent
Beyond activities within the scientific community, the dedicated
reader may feel encouraged to use her/his solid topical foundation
in statistical methods of data analysis for careers in either
fields of higher education, business management, marketing, or the
financial services, amongst many other possibilities.

supply, sustainability.

Reflection and introspection; critical evaluation of the data
(tangible facts) available and, even more interesting,
the data and information that is *not* available (non-knowledge),
and inferences made therefrom; *not* blind faith in quantitative
methods. Looking beyond horizons; anticipation.

\medskip
\noindent
Luckily, not every single matter of human life is amenable to
quantification, or, acknowledging an individual freedom to make
choices, needs to be quantified in the first place.

%%%%%%%%%%%%%%%%%%%%%%%%%%%%%%%%%%%%%%%%%%%%%%%%%%%%%%%%%%%%%%%%%%%
%%%%%%%%%%%%%%%%%%%%%%%%%%%%%%%%%%%%%%%%%%%%%%%%%%%%%%%%%%%%%%%%%%%