%%%%%%%%%%%%%%%%%%%%%%%%%%%%%%%%%%%%%%%%%%%%%%%%%%%%%%%%%%%%%%%%%%%
%  File name: ch12-Stats.tex
%  Title:
%  Version: 20.06.2019 (hve)
%%%%%%%%%%%%%%%%%%%%%%%%%%%%%%%%%%%%%%%%%%%%%%%%%%%%%%%%%%%%%%%%%%%
%%%%%%%%%%%%%%%%%%%%%%%%%%%%%%%%%%%%%%%%%%%%%%%%%%%%%%%%%%%%%%%%%%%
\chapter[Univariate methods of statistical data
analysis]{Univariate methods of statistical data analysis: 
confidence intervals and testing for differences}
\lb{ch12}
%%%%%%%%%%%%%%%%%%%%%%%%%%%%%%%%%%%%%%%%%%%%%%%%%%%%%%%%%%%%%%%%%%%
In this chapter we present a selection of standard inferential 
statistical techniques within the \textbf{frequentist framework}
that, based upon the random sampling of some target 
population~$\boldsymbol{\Omega}$, were developed for the purpose 
of (a)~range-estimating unknown distribution parameters by means 
of \textbf{confidence intervals}, (b)~\textbf{testing for
differences} between a given empirical distribution of a
one-dimensional statistical variable and its \textit{a priori}
assumed theoretical distribution, and (c)~\textbf{comparing}
distributional properties and parameters of a one-dimensional
statistical variable between two or more subgroups of
$\boldsymbol{\Omega}$. Since the methods to be introduced relate to considerations on distributions of a single one-dimensional
statistical variable only, they are thus referred to as
\textbf{univariate}.

%%%%%%%%%%%%%%%%%%%%%%%%%%%%%%%%%%%%%%%%%%%%%%%%%%%%%%%%%%%%%%%%%%%
\section[Confidence intervals]{Confidence intervals}
\lb{sec:konfintv}
%%%%%%%%%%%%%%%%%%%%%%%%%%%%%%%%%%%%%%%%%%%%%%%%%%%%%%%%%%%%%%%%%%%
Assume given a continuous one-dimensional statistical variable~$X$ 
which satisfies in some target population~$\boldsymbol{\Omega}$ a 
\textbf{Gau\ss ian normal distribution} with \textit{unknown}
\textbf{distribution parameters} $\theta \in \{\mu, \sigma^{2}\}$
(cf. Sec.~\ref{sec:normverteil}). The issue is to determine, using 
empirical data from a random sample 
$\boldsymbol{S_{\Omega}}$:~$(X_{1}, \ldots, X_{n})$, a two-sided 
 \textbf{confidence interval} estimate for any one of these unknown distribution parameters~$\theta$
%which can be relied on
at (as one says) a \textbf{confidence level} $1-\alpha$, where, by
convention, $\alpha \in [0.01,0.05]$.

\medskip
\noindent
Centred on a suitable unbiased and consistent maximum likelihood 
point estimator function~$\hat{\theta}_{n}(X_{1},\ldots,X_{n})$
for $\theta$, the aim of the estimation process is to explicitly
account for the \textbf{sampling error} $\delta_{K}$ arising due to
the random selection process. This approach yields a two-sided
confidence interval
%
\be
K_{1-\alpha}(\theta) = \left[\hat{\theta}_{n}-\delta_{K},
\hat{\theta}_{n}+\delta_{K}\right] \ ,
\ee
%
such that $P(\theta \in K_{1-\alpha}(\theta))=1-\alpha$ applies.
The interpretation of the confidence interval $K_{1-\alpha}$ is
that upon arbitrarily many independent repetitions of the
random sampling process, in $(1-\alpha)\times$100\% of all cases
the unknown distribution parameter~$\theta$ will fall inside the
boundaries of $K_{1-\alpha}$ and in $\alpha\times$100\% of all
cases it will not.\footnote{In actual reality, for a given
fixed confidence interval $K_{1-\alpha}$, the unknown distribution
parameter~$\theta$ either takes its value inside $K_{1-\alpha}$, or
not, but the researcher cannot say which case applies.} In the
following we will consider the two cases which result when choosing
$\theta \in \{\mu, \sigma^{2}\}$.

%------------------------------------------------------------------
\subsection[Confidence intervals for a mean]{Confidence
intervals for a population mean}
%------------------------------------------------------------------
When $\theta=\mu$, and $\hat{\theta}_{n}=\bar{X}_{n}$ by 
Eq.~(\ref{eq:sampmean}), the \textbf{two-sided confidence interval 
for a population mean} $\mu$ at significance level $1-\alpha$ 
becomes
%
\be
K_{1-\alpha}(\mu) = \left[\bar{X}_{n}-\delta_{K},
\bar{X}_{n}+\delta_{K}\right] \ ,
\ee
%
with a \textbf{sampling error} amounting to
%
\be
\lb{eq:samplerrcimu}
\delta_{K} = t_{n-1;1-\alpha/2}\,\frac{S_{n}}{\sqrt{n}} \ ,
\ee
%
where $S_{n}$ is the positive square root of the \textbf{sample 
variance} $S_{n}^{2}$ according to Eq.~(\ref{eq:sampvar}), and
$t_{n-1;1-\alpha/2}$ denotes the value of the 
$(1-\alpha/2)$--quantile of a $t$--distribution with $df=n-1$ 
degrees of freedom; cf. Sec.~\ref{sec:tverteil}. The ratio 
$\displaystyle\frac{S_{n}}{\sqrt{n}}$ represents the
\textbf{standard error}~$\text{SE}\bar{X}_{n}$ associated with
$\bar{X}_{n}$; cf. Eq.~(\ref{eq:sesammean}).

\medskip
\noindent
\underline{GDC:} mode \texttt{STAT} $\rightarrow$ \texttt{TESTS} 
$\rightarrow$ \texttt{TInterval}

\medskip
\noindent
Equation~(\ref{eq:samplerrcimu}) may be inverted to obtain the 
\textbf{minimum sample size} necessary to construct a two-sided 
confidence interval for $\mu$ to a prescribed accuracy 
$\delta_\mathrm{max}$, maximal sample variance
$\sigma_\mathrm{max}^{2}$, and fixed confidence level $1-\alpha$.
Thus,
%
\be
n \geq \left(\frac{t_{n-1;1-\alpha/2}}{\delta_\mathrm{max}}
\right)^{2}\sigma_\mathrm{max}^{2} \ .
\ee
%

%------------------------------------------------------------------
\subsection[Confidence intervals for a variance]{Confidence
intervals for a population variance}
%------------------------------------------------------------------
When $\theta=\sigma^{2}$, and $\hat{\theta}_{n}=S_{n}^{2}$ by 
Eq.~(\ref{eq:sampvar}), the associated point estimator function
%
\be
\frac{(n-1)S_{n}^{2}}{\sigma^{2}} \sim
\chi^{2}(n-1) \ ,
\quad\text{with}\quad n \in \mathbb{N} \ ,
\ee
%
satisfies a $\chi^{2}$--distribution with $df=n-1$ degrees of 
freedom; cf. Sec.~\ref{sec:chi2verteil}. By inverting the 
condition
%
\be
P\left(\chi^{2}_{n-1;\alpha/2} \leq
\frac{(n-1)S_{n}^{2}}{\sigma^{2}} \leq
\chi^{2}_{n-1;1-\alpha/2}\right) \stackrel{!}{=} 1-\alpha \ ,
\ee
%
one derives a \textbf{two-sided confidence interval for a
population variance} $\sigma^{2}$ at significance level $1-\alpha$
given by
%
\be
\left[\frac{(n-1)S_{n}^{2}}{\chi^{2}_{n-1;1-\alpha/2}},
\frac{(n-1)S_{n}^{2}}{\chi^{2}_{n-1;\alpha/2}}\right] \ .
\ee
%
$\chi^{2}_{n-1;\alpha/2}$ and $\chi^{2}_{n-1;1-\alpha/2}$ again 
denote the values of particular quantiles of a 
$\chi^{2}$--distribution.

%%%%%%%%%%%%%%%%%%%%%%%%%%%%%%%%%%%%%%%%%%%%%%%%%%%%%%%%%%%%%%%%%%%
\section[One-sample $\chi^{2}$--goodness--of--fit--test]{\href{https://www.youtube.com/watch?v=EjMZdii62Fk}{One-sample
$\boldsymbol{\chi}^{2}$--goodness--of--fit--test}}
%%%%%%%%%%%%%%%%%%%%%%%%%%%%%%%%%%%%%%%%%%%%%%%%%%%%%%%%%%%%%%%%%%%
A standard research question in quantitative--empirical 
investigations deals with the issue whether or not, with respect 
to some target population~$\boldsymbol{\Omega}$ of sample units, 
the \textbf{distribution law} for a specific one-dimensional 
statistical variable~$X$ may be assumed to comply with a 
particular theoretical reference distribution. This question can 
be formulated in terms of the corresponding \texttt{cdf}s,
$F_{X}(x)$ and $F_{0}(x)$, presupposing that for practical reasons
the spectrum of values of $X$ is subdivided into a set of $k$
mutually exclusive \textbf{categories} (or \textbf{bins}), with $k$
a judiciously chosen positive integer which depends in the first
place on the size $n$ of the random sample
$\boldsymbol{S_{\Omega}}$:~$(X_{1}, \ldots, X_{n})$ to be
investigated.

\medskip
\noindent
The non-parametric \textbf{one-sample 
$\boldsymbol{\chi}^{2}$--goodness--of--fit--test} takes 
as its starting point the pair of

\medskip
\noindent
\textbf{Hypotheses:}
%
\be
\begin{cases}
H_{0}: F_{X}(x) = F_{0}(x)
\quad\Leftrightarrow\quad
O_{i}-E_{i} = 0 \\
H_{1}: F_{X}(x) \neq F_{0}(x)
\quad\Leftrightarrow\quad
O_{i}-E_{i} \neq 0
\end{cases} \ ,
\ee
%
where $O_{i}$ ($i=1,\ldots,k$) denotes the actually
\textbf{observed frequency} of category $i$ in a random sample of
size $n$, $E_{i}:=np_{i}$ denotes the, under $H_{0}$ (and so
$F_{0}(x)$), theoretically \textbf{expected frequency} of
category~$i$ in the same random sample, and $p_{i}$ is the
\textbf{probability} of finding a value of $X$ in category~$i$
under $F_{0}(x)$.

\medskip
\noindent
The present procedure, devised by Pearson (1900)~\ct{pea1900}, 
employs the \textbf{residuals} $O_{i}-E_{i}$ ($i=1\ldots,k$) to 
construct a suitable 

\medskip
\noindent
\textbf{Test statistic:}
%
\be
\lb{eq:chisqgofteststat}
\fbox{$\displaystyle
T_{n}(X_{1}, \ldots, X_{n})
= \sum_{i=1}^{k}\frac{(O_{i}-E_{i})^{2}}{E_{i}}
\ \stackrel{H_{0}}{\approx}\ \chi^{2}(k-1-r)
$}
\ee
%
in terms of a sum of rescaled squared residuals 
$\displaystyle\frac{(O_{i}-E_{i})^{2}}{E_{i}}$, which, under 
$H_{0}$, approximately follows a
$\boldsymbol{\chi^{2}}$\textbf{--test distribution} with 
$df=k-1-r$ degrees of freedom (cf. Sec.~\ref{sec:chi2verteil}); 
$r$ denotes the number of free parameters of the reference 
distribution $F_{0}(x)$ which need to be estimated from the random 
sample data. For this test procedure to be reliable, it is 
\textit{important (!)} that the size $n$ of the random sample be 
chosen such that the condition
%
\be
E_{i} \stackrel{!}{\geq} 5
\ee
%
holds for all categories $i=1,\ldots,k$, due to the fact that the 
$E_{i}$ appear in the denominator of the test statistic in 
Eq.~(\ref{eq:chisqgofteststat}) (and so would artifically inflate 
the magnitudes of the summed ratios when the denominators become 
too small).

\medskip
\noindent
\textbf{Test decision:} The rejection region for $H_{0}$ at 
significance level $\alpha$ is given by (right-sided test)
%
\be
t_{n}>\chi^{2}_{k-1-r;1-\alpha} \ .
\ee
%
By Eq.~(\ref{eq:pvalueright}), the $p$--value associated with a 
realisation $t_{n}$ of (\ref{eq:chisqgofteststat}), which is to be
calculated from the
$\boldsymbol{\chi^{2}}$\textbf{--test distribution}, amounts to
%
\be
p = P(T_{n}>t_{n}|H_{0}) = 1-P(T_{n}\leq t_{n}|H_{0})
= 1-\chi^{2}\texttt{cdf}(0,t_{n},k-1-r) \ .
\ee
%

\medskip
\noindent
\underline{\R:} \texttt{chisq.test(table(\textit{variable}))} \\
\underline{SPSS:} Analyze $\rightarrow$ Nonparametric Tests
$\rightarrow$ Legacy Dialogs $\rightarrow$ Chi-square \ldots

\medskip
\noindent
\textbf{Effect size:} In the present context, the practical
significance of the phenomenon investigated can be estimated
from the realisation $t_{n}$ and the sample size~$n$ by
%
\be
\fbox{$\displaystyle
\lb{eq:eschisq}
w := \sqrt{\frac{t_{n}}{n}} \ .
$}
\ee
%
For the interpretation of its strength Cohen 
(1992)~\ct[Tab.~1]{coh1992} recommends the

\medskip
\noindent
\underline{\textbf{Rule of thumb:}}\\
$0.10 \leq w < 0.30$: small effect\\
$0.30 \leq w < 0.50$: medium effect\\
$0.50 \leq w$: large effect.

\medskip
\noindent
Note that in the spirit of \textbf{critical rationalism} the 
one-sample $\chi^{2}$--goodness--of--fit--test provides a tool for 
empirically \textit{excluding} possibilities of distribution laws 
for $X$.

%%%%%%%%%%%%%%%%%%%%%%%%%%%%%%%%%%%%%%%%%%%%%%%%%%%%%%%%%%%%%%%%%%%
\section[One-sample $t$-- and $Z$--tests for a population 
mean]{One-sample $\boldsymbol{t}$-- and $\boldsymbol{Z}$--tests 
for a population mean}
\lb{sec:onesampttest}
%%%%%%%%%%%%%%%%%%%%%%%%%%%%%%%%%%%%%%%%%%%%%%%%%%%%%%%%%%%%%%%%%%%
The idea here is to test whether the unknown population mean $\mu$ 
of some continuous one-dimensional statistical variable~$X$ is 
equal to, less than, or greater than some reference value 
$\mu_{0}$, to a given significance level~$\alpha$. To this end, it 
is required that $X$ satisfy in the target 
population~$\boldsymbol{\Omega}$ a \textbf{Gau\ss ian normal 
distribution}, i.e., $X \sim N(\mu;\sigma^{2})$; cf. 
Sec.~\ref{sec:normverteil}. The quantitative--analytical tool to 
be employed in this case is the parametric \textbf{one-sample 
$\boldsymbol{t}$--test for a population mean} developed by Student 
[Gosset] (1908)~\ct{stu1908}, or, when the sample size $n \geq 
50$, in consequence of the \textbf{central limit theorem} discussed 
in Sec.~\ref{sec:zentrgrenz}, the corresponding
\textbf{one-sample $\boldsymbol{Z}$--test}.

\medskip
\noindent
For a random sample~$\boldsymbol{S_{\Omega}}$:~$(X_{1}, \ldots, 
X_{n})$ of size $n \geq 50$, the validity of the \textit{assumption 
(!)} of \textbf{normality} for the $X$-distribution can be tested
by a procedure due to the Russian mathematicians
\href{http://www-history.mcs.st-and.ac.uk/Biographies/Kolmogorov.html}{Andrey Nikolaevich Kolmogorov (1903--1987)} 
and 
\href{http://en.wikipedia.org/wiki/Nikolai_Smirnov_(mathematician)}
{Nikolai Vasilyevich Smirnov (1900--1966)}. This tests the null 
hypothesis $H_{0}$: ``There is no difference between the 
distribution of the sample data and the associated reference 
normal distribution'' against the alternative $H_{1}$: ``There is 
a difference between the distribution of the sample data and the 
associated reference normal distribution;''
cf. Kolmogorov (1933)~\ct{kol1933b} and Smirnov 
(1939)~\ct{smi1939}. This procedure is referred to as the
\textbf{Kolmogorov--Smirnov--test} (or, for short, the KS--test).
The associated test statistic evaluates the strength of the
deviation of the empirical cumulative distribution function [cf. 
Eq.~(\ref{klempvert})] of given random sample data, with sample 
mean $\bar{x}_{n}$ and sample variance $s_{n}^{2}$, from the
\texttt{cdf} of a reference Gau\ss ian normal distribution with
parameters $\mu$ and $\sigma^{2}$ equal to these sample values [cf. 
Eq.~(\ref{eq:gaussiancdf})].

\medskip
\noindent
\underline{\R:} \texttt{ks.test(\textit{variable}, "pnorm")} \\
\underline{SPSS:} Analyze $\rightarrow$ Nonparametric Tests
$\rightarrow$ Legacy Dialogs $\rightarrow$ 1-Sample K-S \ldots: 
Normal

\medskip
\noindent
For sample sizes $n < 50$, however, the validity of the normality 
assumption for the $X$-distribution may be estimated in terms of 
the magnitudes of the \textbf{standardised skewness and excess
kurtosis measures},
%
\be
\lb{eq:g1g2ratios}
\left|\frac{G_{1}}{\text{SE}G_{1}}\right|
\quad\quad\text{and}\quad\quad
\left|\frac{G_{2}}{\text{SE}G_{2}}\right| \ ,
\ee
%
which are constructed from the quantities defined in
Eqs.~(\ref{eq:sampskew})--(\ref{eq:sesamkurt}).
At a significance level $\alpha = 0.05$, the normality assumption 
may be maintained as long as \textit{both} measures are smaller
than the \textbf{critical value} of~$1.96$; cf. Hair \textit{et al} 
(2010)~\ct[p~72f]{haietal2010}.

\medskip
\noindent
Formulated in a non-directed or a directed fashion, the starting 
point of the $t$--test resp.~$Z$--test procedures are the

\medskip
\noindent
\textbf{Hypotheses:}
%
\be
\begin{cases}
H_{0}: \mu=\mu_{0}
\quad\text{or}\quad
\mu \geq \mu_{0}
\quad\text{or}\quad
\mu \leq \mu_{0} \\
H_{1}: \mu \neq \mu_{0}
\quad\text{or}\quad
\mu < \mu_{0}
\quad\text{or}\quad
\mu > \mu_{0}
\end{cases} \ .
\ee
%
To measure the deviation of the sample data from the state 
conjectured to hold in the null hypothesis $H_{0}$, the
difference between the sample mean $\bar{X}_{n}$ and the 
hypothesised population mean $\mu_{0}$, normalised in analogy to 
Eq.~(\ref{eq:standardisation}) by the \textbf{standard error}
%
\be
\text{SE}\bar{X}_{n} := \frac{S_{n}}{\sqrt{n}}
\ee
%
of $\bar{X}_{n}$ given in Eq.~(\ref{eq:sesammean}), serves as the 
$\mu_{0}$--dependent

\medskip
\noindent
\textbf{Test statistic:}
%
\be
\lb{eq:ztteststat}
\fbox{$\displaystyle
T_{n}(X_{1}, \ldots, X_{n})
= \frac{\bar{X}_{n}-\mu_{0}}{\text{SE}\bar{X}_{n}}
\ \stackrel{H_{0}}{\sim}\ 
\begin{cases}
t(n-1) & \text{for}\quad n < 50 \\
 & \\
N(0;1) & \text{for}\quad n \geq 50
\end{cases} \ ,
$}
\ee
%
which, under $H_{0}$, follows a $\boldsymbol{t}$\textbf{--test
distribution} with $df=n-1$ degrees of freedom (cf.
Sec.~\ref{sec:tverteil}) resp.~a \textbf{standard normal
test distribution} (cf. Sec.~\ref{sec:normverteil}). 

\medskip
\noindent
\textbf{Test decision:} Depending on the kind of test to be 
performed, the rejection region for $H_{0}$ at significance level 
$\alpha$ is given by
%
\begin{center}
\begin{tabular}[h]{c|c|c|c}
 & & & \\
\textbf{Kind of test} & $\boldsymbol{H_{0}}$ &
$\boldsymbol{H_{1}}$ &
\textbf{Rejection region for} $\boldsymbol{H_{0}}$ \\
 & & & \\
\hline
 & & & \\
(a)~two-sided & $\mu=\mu_{0}$ & $\mu\neq\mu_{0}$ &
$|t_{n}|>
\begin{cases}
t_{n-1;1-\alpha/2} & (t\text{--test}) \\
z_{1-\alpha/2} & (Z\text{--test})
\end{cases}$ \\
 & & & \\
\hline
 & & & \\
(b)~left-sided & $\mu\geq\mu_{0}$ & $\mu<\mu_{0}$ &
$t_{n}<
\begin{cases}
t_{n-1;\alpha}=-t_{n-1;1-\alpha} & (t\text{--test}) \\
z_{\alpha}=-z_{1-\alpha} & (Z\text{--test})
\end{cases}$ \\
 & & & \\
\hline
 & & & \\
(c)~right-sided & $\mu\leq\mu_{0}$ & $\mu>\mu_{0}$ &
$t_{n}>
\begin{cases}
t_{n-1;1-\alpha} & (t\text{--test}) \\
z_{1-\alpha} & (Z\text{--test})
\end{cases}$ \\
 & & &
\end{tabular}
\end{center}
%
$p$--values associated with realisations $t_{n}$ of 
(\ref{eq:ztteststat}) can be obtained from
Eqs.~(\ref{eq:pvaluetwo})--(\ref{eq:pvalueright}), using
the relevant $\boldsymbol{t}$\textbf{--test distribution} resp. the
\textbf{standard normal test distribution}.

\medskip
\noindent
\underline{\R:} \texttt{t.test(\textit{variable}, mu = $\mu_{0}$)},
\\
\texttt{t.test(\textit{variable}, mu = $\mu_{0}$,
alternative = "less")}, \\
\texttt{t.test(\textit{variable}, mu = $\mu_{0}$,
alternative = "greater")} \\
\underline{GDC:}
mode \texttt{STAT} $\rightarrow$ \texttt{TESTS} $\rightarrow$
\texttt{T-Test\ldots} when $n < 50$,
resp.~mode \texttt{STAT} $\rightarrow$ \texttt{TESTS}
$\rightarrow$ \texttt{Z-Test\ldots} when $n \geq 50$. \\
\underline{SPSS:} Analyze $\rightarrow$ Compare Means
$\rightarrow$ One-Sample T Test \ldots

\medskip
\noindent
\underline{Note:} Regrettably, SPSS provides no option for 
selecting between a ``one-tailed'' (left-/right-sided) and a 
``two-tailed'' (two-sided) $t$--test. The default setting is for a 
two-sided test. For the purpose of one-sided tests the $p$--value 
output of SPSS needs to be divided by $2$.

\medskip
\noindent
\textbf{Effect size:} The practical significance of the phenomenon investigated can be estimated from the sample mean~$\bar{x}_{n}$,
the sample standard deviation~$s_{n}$, and the reference
value~$\mu_{0}$ by the scale-invariant ratio
%
\be
\fbox{$\displaystyle
d := \frac{\left|\bar{x}_{n}-\mu_{0}\right|}{s_{n}} \ .
$}
\ee
%
For the interpretation of its strength Cohen 
(1992)~\ct[Tab.~1]{coh1992} recommends the

\medskip
\noindent
\underline{\textbf{Rule of thumb:}}\\
$0.20 \leq d < 0.50$: small effect\\
$0.50 \leq d < 0.80$: medium effect\\
$0.80 \leq d$: large effect.

\medskip
\noindent
We remark that the statistical software package~\R\ holds 
available a routine \texttt{power.t.test(power, sig.level, delta,
sd, $n$, alternative, type = "one.sample")} for the purpose of
calculating any one of the parameters \texttt{power},
\texttt{delta} or $n$ (provided all remaining parameters have been
specified) in the context of empirical investigations employing the
one-sample $t$--test for a population mean. One-sided tests are
specified via the parameter setting
\texttt{alternative = "one.sided"}.

%%%%%%%%%%%%%%%%%%%%%%%%%%%%%%%%%%%%%%%%%%%%%%%%%%%%%%%%%%%%%%%%%%%
\section[One-sample $\chi^{2}$--test for a population 
variance]{One-sample $\boldsymbol{\chi^{2}}$--test for a population
variance}
\lb{sec:onesampchi2vartest}
%%%%%%%%%%%%%%%%%%%%%%%%%%%%%%%%%%%%%%%%%%%%%%%%%%%%%%%%%%%%%%%%%%%
In analogy to the statistical significance test described in the 
previous section \ref{sec:onesampttest}, one may likewise test 
hypotheses on the value of an unknown population variance 
$\sigma^{2}$ with respect to a reference value $\sigma_{0}^{2}$ 
for a continuous one-dimensional statistical variable~$X$ which 
satisfies in $\boldsymbol{\Omega}$ a \textbf{Gau\ss ian normal 
distribution}, i.e., $X \sim N(\mu;\sigma^{2})$; cf. 
Sec.~\ref{sec:normverteil}. The hypotheses may also be formulated 
in a non-directed or directed fashion according to

\medskip
\noindent
\textbf{Hypotheses:}
%
\be
\begin{cases}
H_{0}: \sigma^{2}=\sigma_{0}^{2}
\quad\text{or}\quad
\sigma^{2} \geq \sigma_{0}^{2}
\quad\text{or}\quad
\sigma^{2} \leq \sigma_{0}^{2} \\
H_{1}: \sigma^{2} \neq \sigma_{0}^{2}
\quad\text{or}\quad
\sigma^{2} < \sigma_{0}^{2}
\quad\text{or}\quad
\sigma^{2} > \sigma_{0}^{2}
\end{cases} \ .
\ee
%
In the \textbf{one-sample $\boldsymbol{\chi^{2}}$--test for a 
population variance}, the underlying $\sigma_{0}^{2}$--dependent

\medskip
\noindent
\textbf{Test statistic:}
%
\be
\lb{eq:chisqteststat}
\fbox{$\displaystyle
T_{n}(X_{1}, \ldots, X_{n})
= \frac{(n-1)S_{n}^{2}}{\sigma_{0}^{2}}
\ \stackrel{H_{0}}{\sim}\ \chi^{2}(n-1)
$}
\ee
%
is chosen to be proportional to the sample variance defined by 
Eq.~(\ref{eq:sampvar}), and so, under $H_{0}$, follows a
$\boldsymbol{\chi^{2}}$\textbf{--test distribution} with
$df=n-1$ degrees of freedom; cf. Sec.~\ref{sec:chi2verteil}.

\medskip
\noindent
\textbf{Test decision:} Depending on the kind of test to be 
performed, the rejection region for $H_{0}$ at significance level 
$\alpha$ is given by
%
\begin{center}
\begin{tabular}[h]{c|c|c|c}
 & & & \\
\textbf{Kind of test} & $\boldsymbol{H_{0}}$ &
$\boldsymbol{H_{1}}$ &
\textbf{Rejection region for} $\boldsymbol{H_{0}}$ \\
 & & & \\
\hline
 & & & \\
(a)~two-sided & $\sigma^{2}=\sigma_{0}^{2}$ &
$\sigma^{2}\neq\sigma_{0}^{2}$ &
$t_{n}\begin{cases}<\chi^{2}_{n-1;\alpha/2} \\
>\chi^{2}_{n-1;1-\alpha/2}\end{cases}$ \\
 & & & \\
\hline
 & & & \\
(b)~left-sided & $\sigma^{2}\geq\sigma_{0}^{2}$ &
$\sigma^{2}<\sigma_{0}^{2}$ &
$t_{n}<\chi^{2}_{n-1;\alpha}$ \\
 & & & \\
\hline
 & & & \\
(c)~right-sided & $\sigma^{2}\leq\sigma_{0}^{2}$ &
$\sigma^{2}>\sigma_{0}^{2}$ &
$t_{n}>\chi^{2}_{n-1;1-\alpha}$ \\
 & & &
\end{tabular}
\end{center}
%
$p$--values associated with realisations $t_{n}$ of 
(\ref{eq:chisqteststat}), which are to be calculated from the
$\boldsymbol{\chi^{2}}$\textbf{--test distribution}, can be
obtained from Eqs.~(\ref{eq:pvaluetwo})--(\ref{eq:pvalueright}).

\medskip
\noindent
\underline{\R:} \texttt{varTest(\textit{variable},
sigma.squared = $\sigma_{0}^{2}$)} (package: \texttt{EnvStats},
by Millard (2013)~\ct{mil2013}), \\
\texttt{varTest(\textit{variable},
sigma.squared = $\sigma_{0}^{2}$, alternative = "less")}, \\
\texttt{varTest(\textit{variable},
sigma.squared = $\sigma_{0}^{2}$, alternative = "greater")}

\medskip
\noindent
Regrettably, the one-sample $\chi^{2}$--test for a population 
variance does not appear to have been implemented in the SPSS 
software package.

%\medskip
%\noindent
%\textbf{Effect size:} The practical significance of the phenomenon %investigated can be estimated and interpreted by means of the
%effect size measure~$w$ defined in Eq.~(\ref{eq:eschisq});
%cf. Cohen (1992)~\ct[Tab.~1]{coh1992}.

%%%%%%%%%%%%%%%%%%%%%%%%%%%%%%%%%%%%%%%%%%%%%%%%%%%%%%%%%%%%%%%%%%%
\section[Independent samples $t$--test for a mean]{\href{https://www.youtube.com/watch?v=3alSVL8oVMM}{Two independent samples
$\boldsymbol{t}$--test for a population mean}}
\lb{sec:ttestindep}
%%%%%%%%%%%%%%%%%%%%%%%%%%%%%%%%%%%%%%%%%%%%%%%%%%%%%%%%%%%%%%%%%%%
Quantitative--empirical studies are frequently interested 
in the question as to what extent there exist significant 
differences between two subgroups of some target 
population~$\boldsymbol{\Omega}$ in 
the distribution of a metrically scaled one-dimensional 
statistical variable $X$. Given that $X$ is \textit{normally 
distributed} in $\boldsymbol{\Omega}$ 
(cf. Sec.~\ref{sec:normverteil}), the parametric \textbf{two 
independent samples $\boldsymbol{t}$--test 
for a population mean} originating from work by Student [Gosset] 
(1908)~\ct{stu1908} provides an efficient and powerful 
investigative tool.

\medskip
\noindent
For independent random samples of sizes $n_{1}, n_{2} \geq 50$,
the issue of whether there exists empirical evidence in the 
samples \textit{against} the assumption of a normally 
distributed~$X$ in  $\boldsymbol{\Omega}$ can again be tested for 
by means of the \textbf{Kolmogorov--Smirnov--test}; cf. 
Sec.~\ref{sec:onesampttest}.

\medskip
\noindent
\underline{\R:} \texttt{ks.test(\textit{variable}, "pnorm")} \\
\underline{SPSS:} Analyze $\rightarrow$ Nonparametric Tests
$\rightarrow$ Legacy Dialogs $\rightarrow$ 1-Sample K-S \ldots: 
Normal

\medskip
\noindent
For $n_{1}, n_{2} < 50$, one may resort to a consideration of the 
magnitudes of the \textbf{standardised skewness and excess kurtosis
measures}, Eqs.~(\ref{eq:g1g2ratios}), to check for the validity of
the normality assumption for the $X$-distributions. 

\medskip
\noindent
In addition, prior to the $t$--test procedure, one needs to 
establish whether or not the variances of~$X$ have to be viewed as 
significantly different in the two random samples selected.
\textbf{Levene's test} provides an empirical method to test
$H_{0}:~\sigma_{1}^{2}=\sigma_{2}^{2}$
against $H_{1}:~\sigma_{1}^{2}\neq\sigma_{2}^{2}$; cf. Levene 
(1960)~\ct{lev1960}.

\medskip
\noindent
\underline{\R:}
\texttt{leveneTest(\textit{variable}, \textit{group
variable})} (package: \texttt{car}, by Fox and Weisberg
(2011)~\ct{foxwei2011})

\medskip
\noindent
The hypotheses of a $t$--test may be formulated in a non-directed 
fashion or in a directed one. Hence, the different kinds of 
possible conjectures are

\medskip
\noindent
\textbf{Hypotheses:} \hfill (test for differences)
%
\be
\begin{cases}
H_{0}: \mu_{1}-\mu_{2}=0
\quad\text{or}\quad
\mu_{1}-\mu_{2} \geq 0
\quad\text{or}\quad
\mu_{1}-\mu_{2} \leq 0 \\
H_{1}: \mu_{1}-\mu_{2} \neq 0
\quad\text{or}\quad
\mu_{1}-\mu_{2} < 0
\quad\text{or}\quad
\mu_{1}-\mu_{2} > 0
\end{cases} \ .
\ee
%
A test statistic is constructed from the difference of sample 
means, $\bar{X}_{n_{1}}-\bar{X}_{n_{2}}$,
%and the difference of unknown population means  
%$(\mu_{1}-\mu_{2})$ in the two subgroups, 
standardised by the \textbf{standard error}
%
\be
\text{SE}(\bar{X}_{n_{1}}-\bar{X}_{n_{2}})
:= \sqrt{\frac{S_{n_{1}}^{2}}{n_{1}}+\frac{S_{n_{2}}^{2}}{n_{2}}} 
\ ,
\ee
%
which derives from the associated theoretical \textbf{sampling 
distribution} for $\bar{X}_{n_{1}}-\bar{X}_{n_{2}}$. Thus, one
obtains the

\medskip
\noindent
\textbf{Test statistic:} 
%
\be
\lb{eq:indeptteststat}
\fbox{$\displaystyle
T_{n_{1},n_{2}} := \frac{\bar{X}_{n_{1}}-\bar{X}_{n_{2}}
}{\text{SE}(\bar{X}_{n_{1}}-\bar{X}_{n_{2}})}
\ \stackrel{H_{0}}{\sim}\ t(df) \ ,
$}
\ee
%
which, under $H_{0}$, satisfies a
$\boldsymbol{t}$\textbf{--test distribution} (cf. 
Sec.~\ref{sec:tverteil}) with a number of degrees of freedom 
determined by the relations
%
\be
\displaystyle
df := \begin{cases}
n_{1}+n_{2}-2\ , &
\text{when}\quad \sigma_{1}^{2}=\sigma_{2}^{2} \\
 & \\
{\displaystyle\frac{\left(\frac{S_{n_{1}}^{2}}{n_{1}}
+\frac{S_{n_{2}}^{2}}{n_{2}}\right)^{2}}{\frac{(S_{n_{1}}^{2}/n_{1})^{2}
}{n_{1}-1}+\frac{(S_{n_{2}}^{2}/n_{2})^{2}}{n_{2}-1}}} \ ,
& \text{when}\quad \sigma_{1}^{2} \neq \sigma_{2}^{2}
\end{cases} \ .
\ee
%

\medskip
\noindent
\textbf{Test decision:} Depending on the kind of test to be 
performed, the rejection region for $H_{0}$ at significance level 
$\alpha$ is given by
%
\begin{center}
\begin{tabular}[h!]{c|c|c|c}
 & & & \\
\textbf{Kind of test} & $\boldsymbol{H_{0}}$ &
$\boldsymbol{H_{1}}$ &
\textbf{Rejection region for} $\boldsymbol{H_{0}}$ \\
 & & & \\
\hline
 & & & \\
(a)~two-sided & $\mu_{1}-\mu_{2}=0$ & $\mu_{1}-\mu_{2}\neq 0$ &
$|t_{n_{1},n_{2}}|>t_{df;1-\alpha/2}$ \\
 & & & \\
\hline
 & & & \\
(b)~left-sided & $\mu_{1}-\mu_{2}\geq 0$ & $\mu_{1}-\mu_{2}<0$ &
$t_{n_{1},n_{2}}<t_{df;\alpha}=-t_{df;1-\alpha}$ \\
 & & & \\
\hline
 & & & \\
(c)~right-sided & $\mu_{1}-\mu_{2}\leq 0$ & $\mu_{1}-\mu_{2}>0$ &
$t_{n_{1},n_{2}}>t_{df;1-\alpha}$ \\
 & & &
\end{tabular}
\end{center}
%
$p$--values associated with realisations $t_{n_{1},n_{2}}$ of 
(\ref{eq:indeptteststat}), which are to be calculated from the
$\boldsymbol{t}$\textbf{--test distribution}, can be obtained from 
Eqs.~(\ref{eq:pvaluetwo})--(\ref{eq:pvalueright}).

\medskip
\noindent
\underline{\R:}
\texttt{t.test(\textit{variable}\texttildelow\textit{group
variable})}, \\
\texttt{t.test(\textit{variable}\texttildelow\textit{group
variable}, alternative = "less")}, \\
\texttt{t.test(\textit{variable}\texttildelow\textit{group
variable}, alternative = "greater")} \\
\underline{GDC:} mode \texttt{STAT} $\rightarrow$ \texttt{TESTS}
$\rightarrow$ \texttt{2-SampTTest\ldots} \\
\underline{SPSS:} Analyze $\rightarrow$ Compare Means
$\rightarrow$ Independent-Samples T Test \ldots

\medskip
\noindent
\underline{Note:} Regrettably, SPSS provides no option for 
selecting between a one-sided and a two-sided $t$--test. The 
default setting is for a two-sided test. For the purpose of 
one-sided tests the $p$--value output of SPSS needs to be divided 
by $2$.

\medskip
\noindent
\textbf{Effect size:} The practical significance of the phenomenon investigated can be estimated from the sample
means~$\bar{x}_{n_{1}}$ and $\bar{x}_{n_{2}}$ and the pooled sample
standard deviation
%
\be
s_{\mathrm{pooled}} := \sqrt{\frac{(n_{1}-1)s_{n_{1}}^{2}
+(n_{2}-1)s_{n_{2}}^{2}}{n_{1}+n_{2}-2}}
\ee
%
by the scale-invariant ratio
%
\be
\fbox{$\displaystyle
d := \frac{\left|\bar{x}_{n_{1}}-\bar{x}_{n_{2}}
\right|}{s_{\mathrm{pooled}}} \ .
$}
\ee
%
For the interpretation of its strength Cohen 
(1992)~\ct[Tab.~1]{coh1992} recommends the

\medskip
\noindent
\underline{\textbf{Rule of thumb:}}\\
$0.20 \leq d < 0.50$: small effect\\
$0.50 \leq d < 0.80$: medium effect\\
$0.80 \leq d$: large effect.

\medskip
\noindent
\underline{\R:}
\texttt{cohen.d(\textit{variable}, \textit{group
variable}, pooled = TRUE)} (package: \texttt{effsize}, by
Torchiano (2018)~\ct{tor2018})

\medskip
\noindent
We remark that the statistical software package~\R\ holds 
available a routine \texttt{power.t.test(power, sig.level, delta,
sd, $n$, alternative)} for the purpose of calculation of any one 
of the parameters \texttt{power}, \texttt{delta} or $n$ (provided 
all remaining parameters have been specified) in the context of 
empirical investigations employing the independent samples 
$t$--test for a population mean. Equal values of $n$ are 
required here. One-sided tests are addressed via the parameter 
setting \texttt{alternative = "one.sided"}.

\medskip
\noindent
When the necessary conditions for the application of the
independent samples $t$--test are \textit{not} satisfied, the 
following alternative test procedures (typically of a weaker test 
power, though) for comparing two subgroups of 
$\boldsymbol{\Omega}$ with respect to the distribution of a 
metrically scaled variable~$X$ exist:
%
\begin{itemize}
\item[(i)] at the \textbf{nominal} scale level, provided $E_{ij}
\geq 5$ for all $i,j$, the \textbf{$\boldsymbol{\chi}^{2}$--test
for homogeneity}; cf.  Sec.~\ref{sec:chisqhomo} below, and
\item[(ii)] at the \textbf{ordinal} scale level, provided $n_{1}, 
n_{2} \geq 8$, the two independent samples 
\textbf{Mann--Whitney--$\boldsymbol{U}$--test} for a  median; cf. 
the following Sec.~\ref{sec:utest}.
\end{itemize}
%

%%%%%%%%%%%%%%%%%%%%%%%%%%%%%%%%%%%%%%%%%%%%%%%%%%%%%%%%%%%%%%%%%%
\section[Independent samples Mann--Whitney--$U$--test]{\href{https://www.youtube.com/watch?v=bG6xXoyEgB8}{Two independent samples
Mann--Whitney--$\boldsymbol{U}$--test for a population median}}
\lb{sec:utest}
%%%%%%%%%%%%%%%%%%%%%%%%%%%%%%%%%%%%%%%%%%%%%%%%%%%%%%%%%%%%%%%%%%%
The non-parametric \textbf{two independent samples 
Mann--Whitney--$\boldsymbol{U}$--test for a population median},
devised by the Austrian--US-American mathematician and statistician
\href{http://en.wikipedia.org/wiki/Henry_Mann}{Henry Berthold
Mann~(1905--2000)} and the US-American statistician
Donald Ransom Whitney~(1915--2001) in 1947~\ct{manwhi1947}, can be 
applied to random sample data for ordinally scaled one-dimensional 
statistical variables~$X$, or for metrically scaled 
one-dimensional statistical variables~$X$ which may \textit{not} be 
reasonably assumed to be normally distributed in the target 
population~$\boldsymbol{\Omega}$. In both situations, the method 
employs \textbf{rank number data} (cf. Sec.~\ref{sec:2Dord}), which 
faithfully represents the original random sample data, to 
effectively compare the medians of~$X$ (or, rather, the mean 
rank numbers) between two independent groups. It aims to test 
empirically the null hypothesis $H_{0}$ of one of the following 
pairs of non-directed or directed

\medskip
\noindent
\textbf{Hypotheses:} \hfill (test for differences)
%
\be
\begin{cases}
H_{0}: \tilde{x}_{0.5}(1) = \tilde{x}_{0.5}(2)
\quad\text{or}\quad
\tilde{x}_{0.5}(1) \geq \tilde{x}_{0.5}(2)
\quad\text{or}\quad
\tilde{x}_{0.5}(1) \leq \tilde{x}_{0.5}(2) \\
H_{1}: \tilde{x}_{0.5}(1) \neq \tilde{x}_{0.5}(2)
\quad\text{or}\quad
\tilde{x}_{0.5}(1) < \tilde{x}_{0.5}(2)
\quad\text{or}\quad
\tilde{x}_{0.5}(1) > \tilde{x}_{0.5}(2)
\end{cases} \ .
\ee
%

\medskip
\noindent
Given two independent sets of random sample data for $X$,
\textbf{ranks} are being introduced on the basis of an ordered
\textbf{joint random sample} of size $n=n_{1}+n_{2}$ according to
$x_{i}(1) \mapsto R[x_{i}(1)]$ and $x_{i}(2) \mapsto R[x_{i}(2)]$. 
From the ranks thus assigned to the elements of each of the two 
sets of data, one computes the

\medskip
\noindent
$\boldsymbol{U}$\textbf{--values:}
%
\begin{eqnarray}
U_{1} & := & n_{1}n_{2} + \frac{n_{1}(n_{1}+1)}{2}
- \sum_{i=1}^{n_{1}}R[x_{i}(1)] \\
%
U_{2} & := & n_{1}n_{2} + \frac{n_{2}(n_{2}+1)}{2}
- \sum_{i=1}^{n_{2}}R[x_{i}(2)] \ ,
\end{eqnarray}
%
for which the identity $U_{1}+U_{2}=n_{1}n_{2}$ applies. Choose 
$U:=\min(U_{1},U_{2})$.\footnote{Since the $U$--values are tied to 
each other by the identity $U_{1}+U_{2}=n_{1}n_{2}$, it makes no 
difference to this method when one chooses $U:=\max(U_{1},U_{2})$ 
instead.} For independent random samples of sizes $n_{1}, n_{2} 
\geq 8$ (see, e.g., Bortz (2005) \ct[p~151]{bor2005}), the 
standardised $U$--value serves as the

\medskip
\noindent
\textbf{Test statistic:} 
%
\be
\lb{eq:uteststat}
\fbox{$\displaystyle
T_{n_{1},n_{2}} := \frac{U-\mu_{U}}{\text{SE}U}
\ \stackrel{H_{0}}{\approx}\ N(0;1) \ ,
$}
\ee
%
which, under $H_{0}$, approximately satisfies a \textbf{standard
normal test distribution}; cf. Sec.~\ref{sec:normverteil}. Here,
$\mu_{U}$ denotes the mean of the $U$--value expected under
$H_{0}$; it is defined in terms of the sample sizes by
%
\be
\mu_{U}:=\frac{n_{1}n_{2}}{2} \ ;
\ee
%
$\text{SE}U$ denotes the \textbf{standard error} of the $U$--value 
and can be obtained, e.g., from Bortz (2005) 
\ct[Eq.~(5.49)]{bor2005}.

\medskip
\noindent
\textbf{Test decision:} Depending on the kind of test to be 
performed, the rejection region for $H_{0}$ at significance level 
$\alpha$ is given by
%
\begin{center}
\begin{tabular}[h]{c|c|c|c}
 & & & \\
\textbf{Kind of test} & $\boldsymbol{H_{0}}$ &
$\boldsymbol{H_{1}}$ &
\textbf{Rejection region for} $\boldsymbol{H_{0}}$ \\
 & & & \\
\hline
 & & & \\
(a)~two-sided & $\tilde{x}_{0.5}(1) = \tilde{x}_{0.5}(2)$ & 
$\tilde{x}_{0.5}(1) \neq \tilde{x}_{0.5}(2)$ &
$|t_{n_{1},n_{2}}|>z_{1-\alpha/2}$ \\
 & & & \\
\hline
 & & & \\
(b)~left-sided & $\tilde{x}_{0.5}(1) \geq \tilde{x}_{0.5}(2)$ & 
$\tilde{x}_{0.5}(1) < \tilde{x}_{0.5}(2)$ &
$t_{n_{1},n_{2}}<z_{\alpha}=-z_{1-\alpha}$ \\
 & & & \\
\hline
 & & & \\
(c)~right-sided & $\tilde{x}_{0.5}(1) \leq \tilde{x}_{0.5}(2)$ & 
$\tilde{x}_{0.5}(1) > \tilde{x}_{0.5}(2)$ &
$t_{n_{1},n_{2}}>z_{1-\alpha}$ \\
 & & &
\end{tabular}
\end{center}
%
$p$--values associated with realisations $t_{n_{1},n_{2}}$ of 
(\ref{eq:uteststat}), which are to be calculated from the
\textbf{standard normal test distribution}, can be obtained from 
Eqs.~(\ref{eq:pvaluetwo})--(\ref{eq:pvalueright}).

\medskip
\noindent
\underline{\R:}
\texttt{wilcox.test(\textit{variable}~\texttildelow~\textit{group
variable})}, \\
\texttt{wilcox.test(\textit{variable}~\texttildelow~\textit{group
variable}, alternative = "less")}, \\
\texttt{wilcox.test(\textit{variable}~\texttildelow~\textit{group
variable}, alternative = "greater")} \\
\underline{SPSS:} Analyze $\rightarrow$ Nonparametric Tests
$\rightarrow$ Legacy Dialogs $\rightarrow$ 2 Independent
Samples \ldots: Mann-Whitney U

\medskip
\noindent
\underline{Note:} Regrettably, SPSS provides no option for 
selecting between a one-sided and a two-sided $U$--test. The 
default setting is for a two-sided test. For the purpose of 
one-sided tests the $p$--value output of SPSS needs to be divided 
by $2$.

%%%%%%%%%%%%%%%%%%%%%%%%%%%%%%%%%%%%%%%%%%%%%%%%%%%%%%%%%%%%%%%%%%%
\section[Independent samples $F$--test for a variance]{Two
independent samples $\boldsymbol{F}$--test for a population
variance}
\lb{sec:Ftestindep}
%%%%%%%%%%%%%%%%%%%%%%%%%%%%%%%%%%%%%%%%%%%%%%%%%%%%%%%%%%%%%%%%%%%
In analogy to the independent samples $t$--test for a 
population mean of Sec.~\ref{sec:ttestindep}, one may likewise 
investigate for a metrically scaled one-dimensional statistical 
variable~$X$, which can be assumed to satisfy a Gau\ss ian normal 
distribution in $\boldsymbol{\Omega}$ (cf. 
Sec.~\ref{sec:normverteil}), whether there exists a significant 
difference in the values of the population variance between two 
independent random samples.\footnote{Run the 
Kolmogorov--Smirnov--test to check 
whether the assumption of normality of the distribution of $X$ in 
the two random samples drawn needs to be rejected.} The parametric 
\textbf{two independent samples $\boldsymbol{F}$--test for a 
population variance} empirically evaluates the plausibility of the 
null hypothesis $H_{0}$ in the non-directed resp.~directed pairs of

\medskip
\noindent
\textbf{Hypotheses:} \hfill (test for differences)
%
\be
\begin{cases}
H_{0}: \sigma_{1}^{2}=\sigma_{2}^{2}
\quad\text{or}\quad
\sigma_{1}^{2} \geq \sigma_{2}^{2}
\quad\text{or}\quad
\sigma_{1}^{2} \leq \sigma_{2}^{2} \\
H_{1}: \sigma_{1}^{2} \neq \sigma_{2}^{2}
\quad\text{or}\quad
\sigma_{1}^{2} < \sigma_{2}^{2}
\quad\text{or}\quad
\sigma_{1}^{2} > \sigma_{2}^{2}
\end{cases} \ .
\ee
%
Dealing with independent random samples of sizes $n_{1}$ and 
$n_{2}$, the ratio of the corresponding sample variances serves as 
a

\medskip
\noindent
\textbf{Test statistic:}
%
\be
\lb{eq:fteststat}
\fbox{$\displaystyle
T_{n_{1},n_{2}} := \frac{S_{n_{1}}^{2}}{S_{n_{2}}^{2}}
\ \stackrel{H_{0}}{\sim}\ F(n_{1}-1,n_{2}-1) \ ,
$}
\ee
%
which, under $H_{0}$, satisfies an $\boldsymbol{F}$\textbf{--test
distribution} with $df_{1}=n_{1}-1$ and $df_{2}=n_{2}-1$ degrees of
freedom; cf. Sec.~\ref{sec:fverteil}.

\medskip
\noindent
\textbf{Test decision:} Depending on the kind of test to be 
performed, the rejection region for $H_{0}$ at significance level 
$\alpha$ is given by
%
\begin{center}
\begin{tabular}[h]{c|c|c|c}
 & & & \\
\textbf{Kind of test} & $\boldsymbol{H_{0}}$ &
$\boldsymbol{H_{1}}$ &
\textbf{Rejection region for} $\boldsymbol{H_{0}}$ \\
 & & & \\
\hline
 & & & \\
(a)~two-sided & $\sigma_{1}^{2}=\sigma_{2}^{2}$ & $\sigma_{1}^{2} 
\neq \sigma_{2}^{2}$ &
$t_{n_{1},n_{2}}\begin{cases} < 1/f_{n_{2}-1,n_{1}-1;1-\alpha/2} \\
> f_{n_{1}-1,n_{2}-1;1-\alpha/2} \end{cases}$ \\
 & & & \\
\hline
 & & & \\
(b)~left-sided & $\sigma_{1}^{2} \geq \sigma_{2}^{2}$ & 
$\sigma_{1}^{2} < \sigma_{2}^{2}$ &
$t_{n_{1},n_{2}} < 1/f_{n_{2}-1,n_{1}-1;1-\alpha}$ \\
 & & & \\
\hline
 & & & \\
(c)~right-sided & $\sigma_{1}^{2} \leq \sigma_{2}^{2}$ & 
$\sigma_{1}^{2} > \sigma_{2}^{2}$ &
$t_{n_{1},n_{2}} > f_{n_{1}-1,n_{2}-1;1-\alpha}$ \\
 & & &
\end{tabular}
\end{center}
%
$p$--values associated with realisations $t_{n_{1},n_{2}}$ of 
(\ref{eq:fteststat}), which are to be calculated from the
$\boldsymbol{F}$\textbf{--test distribution}, can be obtained from 
Eqs.~(\ref{eq:pvaluetwo})--(\ref{eq:pvalueright}).

\medskip
\noindent
\underline{\R:}
\texttt{var.test(\textit{variable}~\texttildelow~\textit{group
variable})}, \\
\texttt{var.test(\textit{variable}~\texttildelow~\textit{group
variable}, alternative = "less")}, \\
\texttt{var.test(\textit{variable}~\texttildelow~\textit{group
variable}, alternative = "greater")} \\
\underline{GDC:} mode \texttt{STAT} $\rightarrow$ \texttt{TESTS}
$\rightarrow$ \texttt{2-SampFTest\ldots}

\medskip
\noindent
Regrettably, the two-sample $F$--test for a population 
variance does not appear to have been implemented in the SPSS 
software package. Instead, to address quantitative issues of the 
kind raised here, one may resort to \textbf{Levene's test}; cf. 
Sec.~\ref{sec:ttestindep}.

%%%%%%%%%%%%%%%%%%%%%%%%%%%%%%%%%%%%%%%%%%%%%%%%%%%%%%%%%%%%%%%%%%%
\section[Dependent samples $t$--test for a mean]{\href{https://www.youtube.com/watch?v=XqaTOsBSvg0}{Two dependent samples $\boldsymbol{t}$--test
for a population mean}}
\lb{sec:ttestdep}
%%%%%%%%%%%%%%%%%%%%%%%%%%%%%%%%%%%%%%%%%%%%%%%%%%%%%%%%%%%%%%%%%%%
Besides investigating for significant differences in the 
distribution of a single one-dimensional statistical variable~$X$ 
in two or more independent subgroups of some target 
population~$\boldsymbol{\Omega}$, many research projects are 
interested in finding out (i)~how the distributional properties of 
a one-dimensional statistical variable $X$ have changed within one 
and the same random sample of $\boldsymbol{\Omega}$ in an 
experimental before--after situation, or (ii)~how the distribution 
of a one-dimensional statistical variable~$X$ differs between two 
subgroups of $\boldsymbol{\Omega}$, the sample units of which 
co-exist in a natural pairwise one-to-one correspondence to one 
another.

\medskip
\noindent
When the one-dimensional statistical variable~$X$ in question is 
metrically scaled and can be assumed to satisfy a Gau\ss ian 
normal distribution in $\boldsymbol{\Omega}$, significant 
differences can be tested for by means of the parametric
\textbf{two dependent samples $\boldsymbol{t}$--test for a
population mean}. Denoting by $A$ and $B$ either temporal before
and after instants, or partners in a set of natural pairs $(A,B)$,
define for $X$ the metrically scaled \textbf{difference variable}
%
\be
D:=X(A)-X(B) \ .
\ee
%
An \textit{important test prerequisite} demands that $D$ itself may 
be assumed \textit{normally distributed} in $\boldsymbol{\Omega}$; 
cf. Sec.~\ref{sec:normverteil}. Whether this property holds true, 
can be checked for $n \geq 50$ via the
\textbf{Kolmogorov--Smirnov--test}; cf.
Sec.~\ref{sec:onesampttest}. When $n < 50$, one may resort to a
consideration of the magnitudes of the \textbf{standardised
skewness and excess kurtosis measures}, Eqs.~(\ref{eq:g1g2ratios}). 

\medskip
\noindent
With $\mu_{D}$ denoting the population mean of the difference 
variable $D$, the

\medskip
\noindent
\textbf{Hypotheses:} \hfill (test for differences)
%
\be
\begin{cases}
H_{0}: \mu_{D}=0
\quad\text{or}\quad
\mu_{D} \geq 0
\quad\text{or}\quad
\mu_{D} \leq 0 \\
H_{1}: \mu_{D} \neq 0
\quad\text{or}\quad
\mu_{D} < 0
\quad\text{or}\quad
\mu_{D} > 0
\end{cases}
\ee
%
can be given in a non-directed or a directed formulation. From the 
sample mean $\bar{D}$ and its associated \textbf{standard error},
%
\be
\text{SE}\bar{D} := \frac{S_{D}}{\sqrt{n}} \ ,
\ee
%
which derives from the theoretical \textbf{sampling distribution}
for $\bar{D}$, one obtains by means of standardisation according
to Eq.~(\ref{eq:standardisation}) the

\medskip
\noindent
\textbf{Test statistic:}
%
\be
\lb{eq:deptteststat}
\fbox{$\displaystyle
T_{n} := \frac{\bar{D}}{\text{SE}\bar{D}}
\ \stackrel{H_{0}}{\sim}\ t(n-1) \ ,
$}
\ee
%
which, under $H_{0}$, satisfies a $\boldsymbol{t}$\textbf{--test
distribution} with $df=n-1$ degrees of freedom; cf.
Sec.~\ref{sec:tverteil}.

\medskip
\noindent
\textbf{Test decision:} Depending on the kind of test to be 
performed, the rejection region for $H_{0}$ at significance level 
$\alpha$ is given by
%
\begin{center}
\begin{tabular}[h]{c|c|c|c}
 & & & \\
\textbf{Kind of test} & $\boldsymbol{H_{0}}$ &
$\boldsymbol{H_{1}}$ &
\textbf{Rejection region for} $\boldsymbol{H_{0}}$ \\
 & & & \\
\hline
 & & & \\
(a)~two-sided & $\mu_{D}=0$ & $\mu_{D}\neq 0$ &
$|t_{n}|>t_{n-1;1-\alpha/2}$ \\
 & & & \\
\hline
 & & & \\
(b)~left-sided & $\mu_{D} \geq 0$ & $\mu_{D}<0$ &
$t_{n}<t_{n-1;\alpha}=-t_{n-1;1-\alpha}$ \\
 & & & \\
\hline
 & & & \\
(c)~right-sided & $\mu_{D} \leq 0$ & $\mu_{D}>0$ &
$t_{n}>t_{n-1;1-\alpha}$ \\
 & & &
\end{tabular}
\end{center}
%
$p$--values associated with realisations $t_{n}$ of 
(\ref{eq:deptteststat}), which are to be calculated from the
$\boldsymbol{t}$\textbf{--test distribution}, can be obtained from 
Eqs.~(\ref{eq:pvaluetwo})--(\ref{eq:pvalueright}).

\medskip
\noindent
\underline{\R:}
\texttt{t.test(\textit{variableA}, \textit{variableB}, 
paired = "T")}, \\
\texttt{t.test(\textit{variableA}, \textit{variableB},
paired = "T", alternative = "less")}, \\
\texttt{t.test(\textit{variableA}, \textit{variableB},
paired = "T", alternative = "greater")} \\
\underline{SPSS:} Analyze $\rightarrow$ Compare Means
$\rightarrow$ Paired-Samples T Test \ldots

\medskip
\noindent
\underline{Note:} Regrettably, SPSS provides no option for 
selecting between a one-sided and a two-sided $t$--test. The 
default setting is for a two-sided test. For the purpose of 
one-sided tests the $p$--value output of SPSS needs to be divided 
by $2$.

\medskip
\noindent
\textbf{Effect size:} The practical significance of the phenomenon investigated can be estimated from the sample mean~$\bar{D}$ and
the sample standard deviation~$s_{D}$ by the scale-invariant ratio
%
\be
\fbox{$\displaystyle
d := \frac{\left|\bar{D}\right|}{s_{D}} \ .
$}
\ee
%
For the interpretation of its strength Cohen 
(1992)~\ct[Tab.~1]{coh1992} recommends the

\medskip
\noindent
\underline{\textbf{Rule of thumb:}}\\
$0.20 \leq d < 0.50$: small effect\\
$0.50 \leq d < 0.80$: medium effect\\
$0.80 \leq d$: large effect.

\medskip
\noindent
\underline{\R:}
\texttt{cohen.d(\textit{variable}, \textit{group
variable}, paired = TRUE)} (package: \texttt{effsize}, by
Torchiano (2018)~\ct{tor2018})

\medskip
\noindent
We remark that the statistical software package~\R\ holds 
available a routine \texttt{power.t.test(power, sig.level, delta,
sd, $n$, alternative, type = "paired")} for the purpose of
calculation of any one of the parameters \texttt{power},
\texttt{delta} or $n$ (provided all remaining parameters have been
specified) in the context of empirical investigations employing the
dependent samples $t$--test for a population mean. One-sided tests
are addressed via the parameter setting
\texttt{alternative = "one.sided"}.

%%%%%%%%%%%%%%%%%%%%%%%%%%%%%%%%%%%%%%%%%%%%%%%%%%%%%%%%%%%%%%%%%%%
\section[Dependent samples Wilcoxon--test]{Two dependent
samples Wilcoxon--test for a population median}
%%%%%%%%%%%%%%%%%%%%%%%%%%%%%%%%%%%%%%%%%%%%%%%%%%%%%%%%%%%%%%%%%%%
When the test prerequisites of the dependent samples $t$--test 
\textit{cannot} be met, i.e., a given metrically scaled 
one-dimensional statistical variable~$X$ cannot be assumed to 
satisfy a Gau\ss ian normal distribution in $\boldsymbol{\Omega}$, 
or $X$ is an ordinally scaled one-dimensional statistical variable 
in the first place, the non-parametric \textbf{signed ranks test} 
published by the US-American chemist and statistician 
\href{http://en.wikipedia.org/wiki/Frank_Wilcoxon}{Frank Wilcoxon
(1892--1965)} in 1945 \ct{wil1945} constitutes a 
quantitative--empirical tool for comparing the distributional 
properties of $X$ between two dependent random samples drawn from 
$\boldsymbol{\Omega}$. Like Mann and Whitney's $U$--test discussed 
in Sec.~\ref{sec:utest}, it is built around the idea of \textbf{rank 
number data} faithfully representing the original random sample 
data; cf. Sec.~\ref{sec:2Dord}. Defining again a variable
%
\be
D:=X(A)-X(B) \ ,
\ee
%
with associated median $\tilde{x}_{0.5}(D)$, the null hypothesis 
$H_{0}$ in the non-directed or directed pairs of

\medskip
\noindent
\textbf{Hypotheses:} \hfill (test for differences)
%
\be
\begin{cases}
H_{0}: \tilde{x}_{0.5}(D)=0
\quad\text{or}\quad
\tilde{x}_{0.5}(D) \geq 0
\quad\text{or}\quad
\tilde{x}_{0.5}(D) \leq 0 \\
H_{1}: \tilde{x}_{0.5}(D) \neq 0
\quad\text{or}\quad
\tilde{x}_{0.5}(D) < 0
\quad\text{or}\quad
\tilde{x}_{0.5}(D) > 0
\end{cases}
\ee
%
needs to be subjected to a suitable significance test.

\medskip
\noindent
For realisations $d_{i}$ ($i=1,\ldots,n$) of $D$, introduce
\textbf{rank numbers} according to $d_{i} \mapsto R[|d_{i}|]$  for
the ordered \textbf{absolute values} $|d_{i}|$, while keeping a 
record of the \textbf{sign} of each $d_{i}$. Exclude from the data
set all null differences $d_{i}=0$, leading to a sample of reduced
size $n \mapsto n_\mathrm{red}$ . Then form the \textbf{sums of
rank numbers} $W^{+}$ for the $d_{i}>0$ and $W^{-}$ for the
$d_{i}<0$, respectively, which are linked to one another by the
identity $W^{+}+W^{-}=n_\mathrm{red}(n_\mathrm{red}+1)/2$. 
Choose~$W^{+}$.\footnote{Due to the identity
$W^{+}+W^{-}=n_\mathrm{red}(n_\mathrm{red}+1)/2$, choosing 
instead $W^{-}$ would make no qualitative difference to the 
subsequent test procedure.} For reduced sample sizes $n_\mathrm{
red}>20$ (see, e.g., Rinne (2008)~\ct[p~552]{rin2008}), one 
employs the

\medskip
\noindent
\textbf{Test statistic:}
%
\be
\lb{eq:wilcteststat}
\fbox{$\displaystyle
T_{n_\mathrm{red}} := \frac{W^{+}-\mu_{W^{+}}}{\text{SE}W^{+}}
\ \stackrel{H_{0}}{\approx}\ N(0;1) \ ,
$}
\ee
%
which, under $H_{0}$, approximately satisfies a \textbf{standard
normal test distribution}; cf. Sec.~\ref{sec:normverteil}. Here,
the mean $\mu_{W^{+}}$ expected under $H_{0}$ is defined in terms
of $n_\mathrm{red}$ by
%
\be
\mu_{W^{+}}:=\frac{n_\mathrm{red}(n_\mathrm{red}+1)}{4} \ ,
\ee
%
while the \textbf{standard error} $\text{SE}W^{+}$ can be computed 
from, e.g., Bortz (2005) \ct[Eq.~(5.52)]{bor2005}.

\medskip
\noindent
\textbf{Test decision:} Depending on the kind of test to be 
performed, the rejection region for $H_{0}$ at significance level 
$\alpha$ is given by
%
\begin{center}
\begin{tabular}[h]{c|c|c|c}
 & & & \\
\textbf{Kind of test} & $\boldsymbol{H_{0}}$ &
$\boldsymbol{H_{1}}$ &
\textbf{Rejection region for} $\boldsymbol{H_{0}}$ \\
 & & & \\
\hline
 & & & \\
(a)~two-sided & $\tilde{x}_{0.5}(D) = 0$ & $\tilde{x}_{0.5}(D)
\neq 0$ & $|t_{n_\mathrm{red}}|>z_{1-\alpha/2}$ \\
 & & & \\
\hline
 & & & \\
(b)~left-sided & $\tilde{x}_{0.5}(D) \geq 0$ & $\tilde{x}_{0.5}(D)
< 0$ & $t_{n_\mathrm{red}}<z_{\alpha}=-z_{1-\alpha}$ \\
 & & & \\
\hline
 & & & \\
(c)~right-sided & $\tilde{x}_{0.5}(D) \leq 0$ & $\tilde{x}_{0.5}(D)
> $ & $t_{n_\mathrm{red}}>z_{1-\alpha}$ \\
 & & &
\end{tabular}
\end{center}
%
$p$--values associated with realisations $t_{n_\mathrm{red}}$ of 
(\ref{eq:wilcteststat}), which are to be calculated from the
\textbf{standard normal test distribution}, can be obtained from 
Eqs.~(\ref{eq:pvaluetwo})--(\ref{eq:pvalueright}).

\medskip
\noindent
\underline{\R:}
\texttt{wilcox.test(\textit{variableA}, \textit{variableB},
paired = "T")}, \\
\texttt{wilcox.test(\textit{variableA}, \textit{variableB},
paired = "T", alternative = "less")}, \\
\texttt{wilcox.test(\textit{variableA}, \textit{variableB},
paired = "T", alternative = "greater")} \\
\underline{SPSS:} Analyze $\rightarrow$ Nonparametric Tests
$\rightarrow$ Legacy Dialogs $\rightarrow$ 2 Related
Samples \ldots: Wilcoxon


\medskip
\noindent
\underline{Note:} Regrettably, SPSS provides no option for 
selecting between a one-sided and a two-sided Wilcoxon--test. The 
default setting is for a two-sided test. For the purpose of 
one-sided tests the $p$--value output of SPSS needs to be divided 
by $2$.

%%%%%%%%%%%%%%%%%%%%%%%%%%%%%%%%%%%%%%%%%%%%%%%%%%%%%%%%%%%%%%%%%%%
\section[$\chi^{2}$--test for homogeneity]{\href{https://www.youtube.com/watch?v=Lp2M_0_OhWM}{$\boldsymbol{\chi}^{2}$--test
for homogeneity}}
\lb{sec:chisqhomo}
%%%%%%%%%%%%%%%%%%%%%%%%%%%%%%%%%%%%%%%%%%%%%%%%%%%%%%%%%%%%%%%%%%%
Due to its independence of scale levels of measurement, the 
non-parametric $\boldsymbol{\chi^{2}}$\textbf{--test for
homogeneity} constitutes the most generally applicable statistical
test for significant differences in the distributional properties
of a particular one-dimensional statistical variable~$X$ between
$k \in \mathbb{N}$ different independent subgroups of some
population $\boldsymbol{\Omega}$. By assumption, the
one-dimensional variable~$X$ may take values in a total of $l \in
\mathbb{N}$ different \textbf{categories} $a_{j}$ ($j=1,\ldots,l$).
Begin by formulating the

\medskip
\noindent
\textbf{Hypotheses:} \hfill (test for differences)
%
\be
\begin{cases}
H_{0}: X\ \text{satisfies the same distribution
in all}\ k\ \text{subgroups of}\ \boldsymbol{\Omega} \\
H_{1}: X\ \text{satisfies a different distribution
in at least one subgroup of}\ \boldsymbol{\Omega}
\end{cases} \ .
\ee
%
With $O_{ij}$ denoting the \textbf{observed frequency} of category 
$a_{j}$ in subgroup~$i$ ($i=1,\ldots,k$), and $E_{ij}$ the, under 
$H_{0}$, \textbf{expected frequency} of category $a_{j}$ in 
subgroup~$i$, the sum of rescaled squared \textbf{residuals} 
$\displaystyle\frac{(O_{ij}-E_{ij})^{2}}{E_{ij}}$ provides a useful

\medskip
\noindent
\textbf{Test statistic:}
%
\be
\lb{eq:chisqhomteststat}
\fbox{$\displaystyle
T_{n} := \sum_{i=1}^{k}\sum_{j=1}^{l}\frac{(O_{ij}-E_{ij}
)^{2}}{E_{ij}}
\ \stackrel{H_{0}}{\approx}\ \chi^{2}[(k-1)\times(l-1)] \ .
$}
\ee
%
Under $H_{0}$, this test statistic satisfies approximately a 
$\boldsymbol{\chi^{2}}$\textbf{--test distribution} with $df=(k-1)
\times (l-1)$ degrees of 
freedom; cf. Sec.~\ref{sec:chi2verteil}. The $E_{ij}$ are defined 
as \textbf{projections} of the \textbf{observed 
proportions}~$\displaystyle\frac{O_{+j}}{n}$ in the total sample  
of size~$n:=O_{1+}+\ldots+O_{k+}$ of each of the $l$~categories 
$a_{j}$ of $X$ into each of the $k$~subgroups of size $O_{i+}$ 
by [cf. Eqs.~(\ref{eq:margfreq1}) and (\ref{eq:margfreq2})]
%
\be
E_{ij} := O_{i+}\,\frac{O_{+j}}{n} \ .
\ee
%
Note the \textit{important (!) test prerequisite} that the total 
sample size~$n$ be such that
%
\be
E_{ij} \stackrel{!}{\geq} 5
\ee
% 
applies for all categories $a_{j}$ and subgroups~$i$.

\medskip
\noindent
\textbf{Test decision:} The rejection region for $H_{0}$ at 
significance level $\alpha$ is given by (right-sided test)
%
\be
t_{n}>\chi^{2}_{(k-1)\times(l-1);1-\alpha} \ .
\ee
%
By Eq.~(\ref{eq:pvalueright}), the $p$--value associated with a 
realisation $t_{n}$ of (\ref{eq:chisqhomteststat}), which is to
be calculated from the $\boldsymbol{\chi^{2}}$\textbf{--test
distribution}, amounts to
%
\be
p = P(T_{n}>t_{n}|H_{0}) = 1-P(T_{n}\leq t_{n}|H_{0})
= 1-\chi^{2}\texttt{cdf}\left(0,t_{n},(k-1)\times(l-1)\right) \ .
\ee
%

\medskip
\noindent
\underline{\R:} \texttt{chisq.test(\textit{group variable},
\textit{variable})} \\
\underline{GDC:} mode \texttt{STAT} $\rightarrow$ \texttt{TESTS}
$\rightarrow$ \texttt{$\chi^{2}$-Test\ldots}\\
\underline{SPSS:} Analyze $\rightarrow$ Descriptive Statistics
$\rightarrow$ Crosstabs \ldots $\rightarrow$ Statistics \ldots:
Chi-square

\medskip
\noindent
Typically the power of a $\chi^{2}$--test for homogeneity is 
weaker than for the related two procedures of comparing three or
more independent subgroups of $\boldsymbol{\Omega}$, which will be 
discussed in the subsequent Secs.~\ref{sec:anova} and 
\ref{sec:kruskalwallis}.

\medskip
\noindent
\textbf{Effect size:} The practical significance of the phenomenon investigated can be estimated and interpreted by means of the
effect size measure~$w$ defined in Eq.~(\ref{eq:eschisq});
cf. Cohen (1992)~\ct[Tab.~1]{coh1992}.

%%%%%%%%%%%%%%%%%%%%%%%%%%%%%%%%%%%%%%%%%%%%%%%%%%%%%%%%%%%%%%%%%%%
\section[One-way analysis of variance (ANOVA)]{\href{https://www.youtube.com/watch?v=viiFi8zfLX0}{One-way analysis of variance (ANOVA)}}
\lb{sec:anova}
%%%%%%%%%%%%%%%%%%%%%%%%%%%%%%%%%%%%%%%%%%%%%%%%%%%%%%%%%%%%%%%%%%%
This powerful quantitative--analytical tool has been developed in 
the context of investigations on biometrical genetics by 
the English statistician
\href{http://www-history.mcs.st-and.ac.uk/Biographies/Fisher.html}{Sir
Ronald Aylmer Fisher FRS (1890--1962)} (see Fisher 
(1918)~\ct{fis1918}), and later extended by the
US-American statistician
\href{http://en.wikipedia.org/wiki/Henry_Scheffe}{Henry
Scheff\'{e} (1907--1977)} (see Scheff\'{e} (1959)~\ct{sch1959}). 
It is of a parametric nature and can be interpreted alternatively 
as a method for\footnote{Only experimental designs with fixed 
effects are considered here.}
%
\begin{itemize}
\item[(i)] investigating the influence of a qualitative 
one-dimensional statistical variable~$Y$ with $k \geq 3$ 
categories $a_{i}$ ($i=1,\ldots,k$), generally referred to as a 
``factor,'' on a quantitative one-dimensional statistical 
variable~$X$, or\\[-0.5cm]
\item[(ii)] testing for differences of the mean of a quantitative 
one-dimensional statistical variable~$X$ between $k \geq 3$ 
different subgroups of some target 
population~$\boldsymbol{\Omega}$.\\[-0.5cm]
\end{itemize}
%
A necessary condition for the application of the \textbf{one-way 
analysis of variance (ANOVA)} test procedure is that the 
quantitative one-dimensional statistical variable~$X$ to be 
investigated may be reasonably assumed to be (a)~\textit{normally 
distributed} (cf. Sec.~\ref{sec:normverteil}) in the $k \geq 3$ 
subgroups of the target population~$\boldsymbol{\Omega}$ 
considered, with, in addition, (b)~\textit{equal variances}. Both
of these conditions also have to hold for each of a set of $k$ 
mutually stochastically independent random variables $X_{1}, 
\ldots, X_{k}$ representing $k$ random samples drawn independently 
from the identified $k$ subgroups of $\boldsymbol{\Omega}$, of 
sizes $n_{1}, \ldots, n_{k} \in \mathbb{N}$, respectively. In the 
following, the element $X_{ij}$ of the underlying $(n \times 2)$ 
data matrix $\boldsymbol{X}$ represents the $j$th value of $X$ in 
the random sample drawn from the $i$th subgroup of 
$\boldsymbol{\Omega}$, with $\bar{X}_{i}$ the corresponding
\textbf{subgroup sample mean}. The $k$ independent random samples
can be understood to form a \textbf{total random sample} of size 
$\displaystyle n:=n_{1}+\ldots+n_{k} =\sum_{i=1}^{k}n_{i}$, with 
\textbf{total sample mean} $\bar{X}_{n}$; cf.
Eq.~(\ref{eq:sampmean}).

\medskip
\noindent
The intention of the ANOVA procedure in the variant~(ii) stated 
above is to empirically test the null hypothesis $H_{0}$ in the 
set of

\medskip
\noindent
\textbf{Hypotheses:} \hfill (test for differences)
%
\be
\lb{anovahyp1}
\begin{cases}
H_{0}: \mu_{1}=\ldots=\mu_{k}=\mu_{0} \\
H_{1}: \mu_{i} \neq \mu_{0}\ \text{at least for one}
\ i=1,\ldots,k
\end{cases} \ .
\ee
%
The necessary test prerequisites can be checked by (a)~the 
\textbf{Kolmogorov--Smirnov--test} for normality of the 
$X$-distribution in each of the $k$ subgroups of 
$\boldsymbol{\Omega}$ (cf. Sec.~\ref{sec:onesampttest}) when 
$n_{i} \geq 50$, or, when $n_{i} < 50$, by a consideration of the 
magnitudes of the \textbf{standardised skewness and excess kurtosis
measures}, Eqs.~(\ref{eq:g1g2ratios}), and likewise by
(b)~\textbf{Levene's test} for $H_{0}: \sigma_{1}^{2} = 
\ldots=\sigma_{k}^{2}=\sigma_{0}^{2}$ against $H_{1}$: 
``$\sigma_{i}^{2} \neq \sigma_{0}^{2}$ at least for one 
$i=1,\ldots,k$'' to test for equality of the variances in these
$k$ subgroups (cf. Sec.~\ref{sec:ttestindep}).

\medskip
\noindent
\underline{\R:}
\texttt{leveneTest(\textit{variable}, \textit{group
variable})} (package: \texttt{car}, by Fox and Weisberg
(2011)~\ct{foxwei2011})

\medskip
\noindent
The starting point of the ANOVA procedure is a simple
algebraic decomposition of the \textbf{random sample values}
$X_{ij}$ into three additive components according to
%
\be
\lb{eq:xijdecompo}
X_{ij} = \bar{X}_{n} + (\bar{X}_{i}-\bar{X}_{n})
+ (X_{ij}-\bar{X}_{i}) \ .
\ee
%
This expresses the $X_{ij}$ in terms of the sum of the total 
sample mean, $\bar{X}_{n}$, the deviation of the subgroup sample 
means from the total sample mean, $(\bar{X}_{i}-\bar{X}_{n})$, and 
the residual deviation of the sample values from their respective 
subgroup sample means, $(X_{ij}-\bar{X}_{i})$. The decomposition 
of the $X_{ij}$ motivates a \textbf{linear stochastic model} for
the target population~$\boldsymbol{\Omega}$ of the 
form\footnote{Formulated in the context of this linear stochastic 
model, the null and research hypotheses are $H_{0}: 
\alpha_{1}=\ldots=\alpha_{k}=0$ and $H_{1}$: at least one 
$\alpha_{i}\neq 0$, respectively.}
%
\be
\text{in}\ \boldsymbol{\Omega}: \quad
X_{ij}=\mu_{0}+\alpha_{i}+\varepsilon_{ij}
\ee
%
in order to quantify, via the $\alpha_{i}$ ($i=1,\ldots,k$), the 
potential influence of the qualitative one-dimensional 
variable~$Y$ on the quantitative one-dimensional variable~$X$. 
Here $\mu_{0}$ is the \textbf{population mean} of $X$, it holds
that $\sum_{i=1}^{k}n_{i}\alpha_{i}=0$, and it is assumed for the
\textbf{random errors} $\varepsilon_{ij}$ that $\varepsilon_{ij} 
\stackrel{\text{i.i.d.}}{\sim} N(0;\sigma_{0}^{2})$, i.e., that 
they are identically normally distributed and mutually 
stochastically independent.

\medskip
\noindent
Having established the decomposition (\ref{eq:xijdecompo}), one 
next turns to consider the associated set of \textbf{sums of
squared deviations}, defined by
%
\bea
\lb{eq:bss}
\text{BSS} & := &
\sum_{i=1}^{k}\sum_{j=1}^{n_{i}}\left(\bar{X}_{i}-\bar{X}_{n}
\right)^{2}
\ = \ \sum_{i=1}^{k}n_{i}\left(\bar{X}_{i}-\bar{X}_{n}\right)^{2} 
\\
%
\lb{eq:rss}
\text{RSS} & := &
\sum_{i=1}^{k}\sum_{j=1}^{n_{i}}\left(X_{ij}-\bar{X}_{i}\right)^{2}
\\
%
\lb{eq:tss}
\text{TSS} & := &
\sum_{i=1}^{k}\sum_{j=1}^{n_{i}}\left(X_{ij}-\bar{X}_{n}\right)^{2}
\ ,
\eea
%
where the summations are (i)~over all $n_{i}$ sample units within 
a subgroup, and (ii)~over all of the $k$ subgroups themselves.
The sums are referred to as, resp., (a)~the sum of squared 
deviations between the subgroup samples (BSS), (b)~the residual 
sum of squared deviations within the subgroup samples (RSS), and 
(c)~the total sum of squared deviations (TSS) of the individual 
$X_{ij}$ from the total sample mean $\bar{X}_{n}$. It is a fairly 
elaborate though straightforward algebraic exercise to show that 
these three squared deviation terms relate to one another 
according to the strikingly simple and elegant identity (cf. Bosch 
(1999)~\ct[p~220f]{bos1999})
%
\be
\text{TSS} = \text{BSS} + \text{RSS} \ .
\ee
%

\medskip
\noindent
Now, from the sums of squared deviations 
(\ref{eq:bss})--(\ref{eq:tss}), one defines, resp., the 
\textbf{total sample variance},
%
\be
S_\mathrm{total}^{2} := 
\frac{1}{n-1}\,\sum_{i=1}^{k}\sum_{j=1}^{n_{i}}
\left(X_{ij}-\bar{X}_{n}\right)^{2}
= \frac{\text{TSS}}{n-1} \ ,
\ee
%
involving $df=n-1$ degrees of freedom, the \textbf{sample variance 
between subgroups},
%
\be
S_\mathrm{between}^{2} := 
\frac{1}{k-1}\,\sum_{i=1}^{k}n_{i}\left(\bar{X}_{i}-\bar{X}_{n}
\right)^{2}
= \frac{\text{BSS}}{k-1} \ ,
\ee
%
with $df=k-1$, and the \textbf{mean sample variance within
subgroups},
%
\be
S_\mathrm{within}^{2} := 
\frac{1}{n-k}\,\sum_{i=1}^{k}\sum_{j=1}^{n_{i}}
\left(X_{ij}-\bar{X}_{i}\right)^{2}
= \frac{\text{RSS}}{n-k} \ ,
\ee
%
for which $df=n-k$.

\medskip
\noindent
Employing the latter two subgroup-specific dispersion measures, 
the set of hypotheses~(\ref{anovahyp1}) may be recast into the 
alternative form

\medskip
\noindent
\textbf{Hypotheses:} \hfill (test for differences)
%
\be
\lb{anovahyp2}
\begin{cases}
H_{0}: {\displaystyle\frac{S_\mathrm{between}^{2}}{S_\mathrm{
within}^{2}}} \leq 1 \\ \\
H_{1}: {\displaystyle\frac{S_\mathrm{between}^{2}}{S_\mathrm{
within}^{2}}} > 1 
\end{cases} \ .
\ee
%

\medskip
\noindent
Finally, as a test statistic for the ANOVA procedure one chooses 
this very ratio of variances\footnote{This ratio is sometimes 
given as $\displaystyle T_{n,k} := \frac{(\text{explained 
variance})}{(\text{unexplained variance})}$, in analogy to 
expression~(\ref{eq:linregcoeffdet}) below. Occasionally, one also 
considers the coefficient $\displaystyle\eta^{2} := 
\frac{\text{BSS}}{\text{TSS}}$, which, however, does not account 
for the degrees of freedom involved. In this respect, the modified 
coefficient $\displaystyle\tilde{\eta}{}^{2}:=\frac{S_\mathrm{
between}^{2}}{S_\mathrm{total}^{2}}$ would constitute a more 
sophisticated measure.} we just employed,
%
\[
T_{n,k} := \frac{(\text{sample variance between 
subgroups})}{(\text{mean sample variance within subgroups})}
= \frac{\text{BSS}/(k-1)}{\text{RSS}/(n-k)}\ ,
\]
%
expressing the size of the ``sample variance between 
subgroups'' in terms of multiples of the ``mean sample variance 
within subgroups''; it thus constitutes a relative measure.
A real effect of difference between subgroups is thus given when 
the non-negative numerator turns out to be significantly larger 
than the non-negative denominator. Mathematically, this 
statistical measure of deviations between the data and the null 
hypothesis is captured by the

\medskip
\noindent
\textbf{Test statistic:}\footnote{Note the one-to-one
correspondence to the test statistic (\ref{eq:fteststat}) employed
in the independent samples $F$--test for a population variance.}
%
\be
\lb{eq:anovateststat}
\fbox{$\displaystyle
T_{n,k} := 
\frac{S_\mathrm{between}^{2}}{S_\mathrm{within}^{2}}
\ \stackrel{H_{0}}{\sim}\ F(k-1,n-k) \ .
$}
\ee
%
Under $H_{0}$, it satisfies an $\boldsymbol{F}$\textbf{--test
distribution} with $df_{1}=k-1$ 
and $df_{2}=n-k$ degrees of freedom; cf. Sec.~\ref{sec:fverteil}.

\medskip
\noindent
It is a well-established standard in practical applications of the 
one-way ANOVA procedure to display the results of the data 
analysis in the form of a \textbf{summary table}, here given in 
Tab.~\ref{tab:anovatab}.
%
\begin{table}

%
\begin{center}
\begin{tabular}[h]{c|c|c|c|c}
\underline{\textbf{ANOVA}} & sum of & df & mean & test \\
variability & squares &   & square & statistic \\
\hline
between groups & $\text{BSS}$ & $k-1$ &
$S_\mathrm{between}^{2}$ & $t_{n,k}$ \\
within groups & $\text{RSS}$ & $n-k$ & $S_\mathrm{within}^{2}$ & \\
\hline
total & \text{TSS} & $n-1$ & &
\end{tabular}
\end{center}
\caption{ANOVA summary table.}
\lb{tab:anovatab}
\end{table}
%

\medskip
\noindent
\textbf{Test decision:} The rejection region for $H_{0}$ at 
significance level $\alpha$ is given by (right-sided test)
%
\be
t_{n,k} > f_{k-1,n-k;1-\alpha} \ .
\ee
%
With Eq.~(\ref{eq:pvalueright}), the $p$--value associated with a 
specific realisation $t_{n,k}$ of (\ref{eq:anovateststat}), which
is to be calculated from the $\boldsymbol{F}$\textbf{--test
distribution}, amounts to
%
\be
p = P(T_{n,k}>t_{n,k}|H_{0}) = 1-P(T_{n,k}\leq t_{n,k}|H_{0})
= 1-F\texttt{cdf}(0,t_{n,k},k-1,n-k) \ .
\ee
%

\medskip
\noindent
\underline{\R:}
\texttt{anova( lm(\textit{variable}~\texttildelow~\textit{group
variable}) )} (variances equal), \\
\texttt{oneway.test(\textit{variable}~\texttildelow~\textit{group
variable})} (variances not equal) \\
\underline{GDC:} mode \texttt{STAT} $\rightarrow$ \texttt{TESTS}
$\rightarrow$ \texttt{ANOVA(} \\
\underline{SPSS:} Analyze $\rightarrow$ Compare Means
$\rightarrow$ One-Way ANOVA \ldots

\medskip
\noindent
\textbf{Effect size:} The practical significance of the phenomenon investigated can be estimated from the sample sums of squared 
deviations $\text{BSS}$ and $\text{RSS}$ according to
%
\be
\fbox{$\displaystyle
f := \sqrt{\frac{\text{BSS}}{\text{RSS}}} \ .
$}
\ee
%
For the interpretation of its strength Cohen 
(1992)~\ct[Tab.~1]{coh1992} recommends the

\medskip
\noindent
\underline{\textbf{Rule of thumb:}}\\
$0.10 \leq f < 0.25$: small effect\\
$0.25 \leq f < 0.40$: medium effect\\
$0.40 \leq f$: large effect.

\medskip
\noindent
We remark that the statistical software package~\R\ holds 
available a routine \texttt{power.anova.test(groups, $n$, 
between.var, within.var, sig.level, power)} for the purpose of 
calculation of any one of the parameters \texttt{power} or $n$
(provided all remaining parameters have been specified) in the 
context of empirical investigations employing the one-way ANOVA. 
Equal values of $n$ are required here.

\medskip
\noindent
When a one-way ANOVA yields a statistically significant result, 
so-called \textbf{post-hoc tests} need to be run subsequently in 
order to identify those subgroups $i$ whose means $\mu_{i}$ differ 
most drastically from the reference value $\mu_{0}$. The
\textbf{Student--Newman--Keuls--test} (Newman (1939)~\ct{new1939}
and Keuls (1952)~\ct{keu1952}), e.g., successively subjects the
pairs of subgroups with the largest differences in sample means to 
independent samples $t$--tests; cf. Sec.~\ref{sec:ttestindep}. 
Other useful post-hoc tests are those developed by \textbf{
Holm--Bonferroni} (Holm (1979)~\ct{hol1979}), \textbf{Tukey} (Tukey 
(1977)~\ct{tuk1977}), or by \textbf{Schef\-f\'{e}} (Schef\-f\'{e} 
(1959)~\ct{sch1959}).

\medskip
\noindent
\underline{\R:} \texttt{pairwise.t.test(\textit{variable},
\textit{group variable}, p.adj = "bonferroni")} \\
\underline{SPSS:} Analyze $\rightarrow$ Compare Means
$\rightarrow$ One-Way ANOVA \ldots $\rightarrow$
Post Hoc \ldots

%%%%%%%%%%%%%%%%%%%%%%%%%%%%%%%%%%%%%%%%%%%%%%%%%%%%%%%%%%%%%%%%%%%
\section[Kruskal--Wallis--test]{Kruskal--Wallis--test for a
population median}
\lb{sec:kruskalwallis}
%%%%%%%%%%%%%%%%%%%%%%%%%%%%%%%%%%%%%%%%%%%%%%%%%%%%%%%%%%%%%%%%%%%
Finally, a feasible alternative to the one-way ANOVA, when the 
conditions for the latter's legitimate application cannot be met,
or one is interested in the distributional properties of a specific 
ordinally scaled one-dimensional statistical variable~$X$, is 
given by the non-parametric significance test devised by the 
US-American mathematician and statistician
\href{http://www-history.mcs.st-and.ac.uk/Biographies/Kruskal_William.html}{William Henry Kruskal (1919--2005)} and the
US-American economist and statistician
\href{http://en.wikipedia.org/wiki/W._Allen_Wallis}{Wilson
Allen Wallis (1912--1998)} in 1952 \ct{kruwal1952}. The \textbf{
Kruskal--Wallis--test} effectively serves to detect significant 
differences for a population median of an ordinally or metrically 
scaled one-dimensional statistical variable~$X$ between $k \geq 3$ 
independent subgroups of some target 
population~$\boldsymbol{\Omega}$. To be investigated empirically 
is the null hypothesis $H_{0}$ in the pair of mutually exclusive

\medskip
\noindent
\textbf{Hypotheses:} \hfill (test for differences)
%
\be
\begin{cases}
H_{0}: \tilde{x}_{0.5}(1) =
\ldots = \tilde{x}_{0.5}(k) \\
H_{1}: \text{at least one}\ \tilde{x}_{0.5}(i)\ 
(i=1,\ldots,k)\ \text{is different from
the other group medians}
\end{cases} \ .
\ee
%
Introduce \textbf{rank numbers} according to $x_{j}(1) \mapsto 
R[x_{j}(1)]$, \ldots, and $x_{j}(k) \mapsto R[x_{j}(k)]$ within 
the random samples drawn independently from each of the $k \geq 3$ 
subgroups of $\boldsymbol{\Omega}$ on the basis of an ordered
\textbf{joint random sample} of size $\displaystyle 
n:=n_{1}+\ldots+n_{k}=\sum_{i=1}^{k}n_{i}$; cf. 
Sec.~\ref{sec:2Dord}. Then form the \textbf{sum of rank numbers}
for each random sample separately, i.e.,
%
\be
R_{+i} := \sum_{j=1}^{n_{i}}R[x_{j}(i)]
\qquad (i=1,\ldots,k) \ .
\ee
%
Provided the sample sizes satisfy the condition $n_{i} \geq 5$ for 
all $k \geq 3$ independent random samples (hence, $n \geq 
15$), the test procedure can be based on the

\medskip
\noindent
\textbf{Test statistic:}
%
\be
\lb{eq:kruwalteststat}
\fbox{$\displaystyle
T_{n,k} := \left[\frac{12}{n(n+1)}
\sum_{i=1}^{k}\frac{R_{+i}^{2}}{n_{i}}\right]
- 3(n+1)
\ \stackrel{H_{0}}{\approx}\ \chi^{2}(k-1) \ ,
$}
\ee
%
which, under $H_{0}$, approximately satisfies a 
$\boldsymbol{\chi^{2}}$\textbf{--test distribution} with $df=k-1$
degrees of freedom (cf. Sec.~\ref{sec:chi2verteil}); see, e.g.,
Rinne (2008)~\ct[p~553]{rin2008}.

\medskip
\noindent
\textbf{Test decision:}  The rejection region for $H_{0}$ at 
significance level $\alpha$ is given by (right-sided test)
%
\be
t_{n,k}>\chi^{2}_{k-1;1-\alpha} \ .
\ee
%
By Eq.~(\ref{eq:pvalueright}), the $p$--value associated with a
realisation $t_{n,k}$ of (\ref{eq:kruwalteststat}), which is to be
calculated from the $\boldsymbol{\chi^{2}}$\textbf{--test
distribution}, amounts to
%
\be
p = P(T_{n,k}>t_{n,k}|H_{0}) = 1-P(T_{n,k}\leq t_{n,k}|H_{0})
= 1-\chi^{2}\texttt{cdf}(0,t_{n,k},k-1) \ .
\ee
%

\medskip
\noindent
\underline{\R:}
\texttt{kruskal.test(\textit{variable}~\texttildelow~\textit{group
variable})} \\
\underline{SPSS:} Analyze $\rightarrow$ Nonparametric Tests
$\rightarrow$ Legacy Dialogs $\rightarrow$ K Independent Samples
\ldots: Kruskal-Wallis H

%%%%%%%%%%%%%%%%%%%%%%%%%%%%%%%%%%%%%%%%%%%%%%%%%%%%%%%%%%%%%%%%%%%
%%%%%%%%%%%%%%%%%%%%%%%%%%%%%%%%%%%%%%%%%%%%%%%%%%%%%%%%%%%%%%%%%%%
