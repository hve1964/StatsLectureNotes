%%%%%%%%%%%%%%%%%%%%%%%%%%%%%%%%%%%%%%%%%%%%%%%%%%%%%%%%%%%%%%%%%%%
%  File name: abstr-Stats.tex
%  Title:
%  Version: 08.06.2019 (hve)
%%%%%%%%%%%%%%%%%%%%%%%%%%%%%%%%%%%%%%%%%%%%%%%%%%%%%%%%%%%%%%%%%%%
\addcontentsline{toc}{chapter}{Abstract}
%%%%%%%%%%%%%%%%%%%%%%%%%%%%%%%%%%%%%%%%%%%%%%%%%%%%%%%%%%%%%%%%%%%
\chapter*{}
\vspace{-8ex}
\section*{Abstract}
%%%%%%%%%%%%%%%%%%%%%%%%%%%%%%%%%%%%%%%%%%%%%%%%%%%%%%%%%%%%%%%%%%%
{\small These lecture notes were written with the aim to provide 
an accessible though technically solid introduction to the logic 
of systematical analyses of statistical data to undergraduate 
and to postgraduate students, in particular in the Social Sciences 
and in Economics. They may also serve as a general reference for 
the application of quantitative--empirical research methods. In an 
attempt to encourage the adoption of an interdisciplinary 
perspective on quantitative problems arising in practice, the 
notes cover the four broad topics (i)~descriptive statistical 
processing of raw data, (ii)~elementary probability theory,
(iii)~the operationalisation of one-dimensional latent statistical
variables according to Likert's widely used scaling approach, and
(iv)~null hypothesis significance testing within the
frequentist approach to probability theory concerning 
(a)~distributional differences of variables between subgroups of a 
target population, and (b)~statistical associations between two 
variables. The relevance of effect sizes for making inferences is
being emphasised. These lecture notes are fully hyperlinked, thus
providing a direct route to original scientific papers as well as
to interesting biographical information. They also list many
commands for activating statistical functions and data analysis
routines in  the software packages \R{}, SPSS, EXCEL and
OpenOffice.}

\vspace{10mm}
\noindent
\underline{Cite as:} 
\href{http://arxiv.org/abs/1302.2525}{arXiv:1302.2525v4 [stat.AP]}
\vfill

\medskip
\noindent
These lecture notes were typeset in \LaTeXe.

%%%%%%%%%%%%%%%%%%%%%%%%%%%%%%%%%%%%%%%%%%%%%%%%%%%%%%%%%%%%%%%%%%%
%%%%%%%%%%%%%%%%%%%%%%%%%%%%%%%%%%%%%%%%%%%%%%%%%%%%%%%%%%%%%%%%%%%